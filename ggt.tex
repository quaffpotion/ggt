\documentclass[12pt,parskip=full]{report}

% This first part of the file is called the PREAMBLE. It includes
% customizations and command definitions. The preamble is everything
% between \documentclass and \begin{document}.

\usepackage[margin=1in]{geometry}  % set the margins to 1in on all sides
\usepackage{graphicx}              % to include figures
\usepackage{amsmath}               % great math stuff
\usepackage{amsfonts}              % for blackboard bold, etc
\usepackage{amsthm}                % better theorem environments
\usepackage{amssymb}
\usepackage{verbatim}
\usepackage{adjustbox}
\usepackage{parskip} %vertical space for new paragaph, and no indent
\usepackage{mathabx} %for wide check
\usepackage{hyperref} %for indexing and links
%\usepackage[nottoc,numbib]{tocbibind} %make bibliography appear in TOC
\usepackage{fancyhdr}
\usepackage{float}
%\usepackage{mathtools}



\usepackage{sansmathfonts}
\usepackage[T1]{fontenc}
%\usepackage{geometry}
%\geometry{legalpaper, portrait, margin=1in}


\usepackage{etoolbox}% http://ctan.org/pkg/etoolbox
\makeatletter
\patchcmd{\@makechapterhead}{\vspace*{50\p@}}{}{}{}% Removes space above \chapter head
\patchcmd{\@makeschapterhead}{\vspace*{50\p@}}{}{}{}% Removes space above \chapter* head
\makeatother

%%%%%%%%%%%%%%%%%%%%%%%%%%%%%%%%%%%%%%%%%%%%%%%%%%%%%%%%%%%%%%%%%%%%%%%%%%%%
%% line spacing
\usepackage{setspace}
    %\singlespacing
    %\onehalfspacing
    \doublespacing

\renewcommand*\familydefault{\sfdefault}

%%%%%%%%%%% FANCY HEADER STUFF %%%%%%%%%%%%


\pagestyle{fancy} 
%\fancyhead[LE,LO]{\textbf{\textsf{\scriptsize Ben Stucky}}}
%\fancyhead[CE,CO]{\textbf{\textsf{\footnotesize Cubulating one-relator products with torsion}}}
%\fancyhead[RE,RO]{\textbf{\textsf{\scriptsize \thepage}}} % of \pageref{numpages}

\fancyhead[LE,LO]{}
\fancyhead[CO,CE]{}
\fancyhead[RE,RO]{}

\fancyfoot[LE,LO]{}
\fancyfoot[CO,CE]{\thepage}
\fancyfoot[RE,RO]{}

%\fancyhead[LE]{\scriptsize \thepage}
%\fancyhead[CE]{\scriptsize \thepage}
%\fancyhead[RE]{} % of \pageref{numpages}

  %For "Page \thepage\ of \pageref{numpages}" to work, 
  %need to have "\label{numpages}" just before "\end{document}" at bottom.
%\fancyfoot[LE,LO]{\textbf{\textsf{\scriptsize University of Oklahoma}}}
%\fancyfoot[CE,CO]{\textbf{\textsf{\scriptsize \href{http://benstuc.ky}{http://benstuc.ky}}}}
%\fancyfoot[RE,RO]{\textbf{\textsf{\scriptsize \href{mailto:bwstucky@ou.edu}{bwstucky@ou.edu}}}}
\renewcommand{\headrulewidth}{0pt}
%\renewcommand{\footrulewidth}{0.4pt}
%\setlength{\headwidth}{6.3in}
%Get rid of headers, and the line, on the first page.
%\fancypagestyle{plain}{\fancyhead{}\renewcommand{\headrulewidth}{0pt}}

%Custom page margins.
%\setlength{\topmargin}{1in}
%\setlength{\bottommargin}{1in}
%\setlength{\oddsidemargin}{1in} 
%\setlength{\evensidemargin}{1in}

%\setlength{\topmargin}{-20pt}
%\setlength{\voffset}{-10pt}
%\setlength{\headsep}{20pt}
%\setlength{\textwidth}{6.3in}
%\setlength{\hoffset}{-20pt}
%\setlength{\footskip}{35pt}
%\setlength{\textheight}{8.9in}
%\setlength{\oddsidemargin}{25pt} 
%\setlength{\evensidemargin}{25pt}



%\newcommand{\vstr}[1][3]{\rule{0ex}{#1ex}} %vertical strut (spacer)
%\newcommand{\hstr}[1][3]{\rule{#1ex}{0ex}} %horizontal strut (spacer)
%\newcommand{\pgap}[1][0.3cm]{\vspace{#1 plus 0.1cm minus 0.1cm}} %paragraph spacer
%\newcommand{\negsp}[1][20]{\mspace{-#1mu}} %negative space (moves left)

%Adjusts spacing between lines.
%\linespread{1.3}

%For footnotes that use symbols instead of numbers.
%\long\def\symbolfootnote[#1]#2{\begingroup%
%\def\thefootnote{\fnsymbol{footnote}}\footnote[#1]{#2}\endgroup}

%%%%%%%%%%% END FANCY HEADER STUFF %%%%%%%%%%%%

% various theorems, numbered by section

%\renewcommand{\familydefault}{\sfdefault} %sans serif font
%\setlength{\parskip}{\baselineskip} %vertical space for new paragraph
%\setlength{\parindent}{0pt} %don't indent new paragraphs

\theoremstyle{plain}
\newtheorem{thm}{Theorem}[section]
\newtheorem{lem}[thm]{Lemma}
\newtheorem{prop}[thm]{Proposition}
\newtheorem{cor}[thm]{Corollary}
\newtheorem{conj}[thm]{Conjecture}

\theoremstyle{definition}
\newtheorem{rmk}[thm]{Remark}
\newtheorem{conv}[thm]{Convention}
\newtheorem{dfn}[thm]{Definition}
\newtheorem{qst}[thm]{Question}
\newtheorem*{claim*}{Claim}
\newtheorem{exa}[thm]{Example}

\DeclareMathOperator{\id}{id}

\newcommand{\bd}[1]{\mathbf{#1}}  % for bolding symbols
\newcommand{\RR}{\mathbb{R}}      % for Real numbers
\newcommand{\ZZ}{\mathbb{Z}}      % for Integers
\newcommand{\NN}{\mathbb{N}}      % for Naturals
\newcommand{\PP}{\mathbb{P}} %for peripheral subgroups
\newcommand{\PPP}{\mathcal{P}} %for collection of peripheral subgroups
\newcommand{\HH}{\mathbb{H}} %for hyperbolic space
\newcommand{\col}[1]{\left[\begin{matrix} #1 \end{matrix} \right]}
\newcommand{\comb}[2]{\binom{#1^2 + #2^2}{#1+#2}}
\newcommand{\lfh}{1.3cm} %figure height
\newcommand{\sfh}{0.3cm} %smaller figure height
\newcommand{\spa}{\hspace{0.1cm}} %add hspace
\newcommand{\nc}[1]{\langle\langle#1\rangle\rangle} %normal closure
\newcommand{\pres}[1]{\langle#1\rangle} %presentation
\newcommand{\cz}{\text{CAT}(0)}
\newcommand{\cl}[1]{\overline{#1}} %closure
\newcommand{\intr}[1]{\text{int}(#1)} %interior
\newcommand{\abs}[1]{\lvert#1\rvert}

\newcommand{\aux}[1]{\widecheck{#1}} % images of sets in auxiliary diagram
\newcommand{\ucc}[1]{\bar{#1}} % (U)niversal (C)over (C)ollapse (collapse essential 2-cells with the same boundary
\newcommand{\pl}[1]{#1_{\#}} % (P)atching
\newcommand{\os}[1]{#1^{(1)}} %(O)ne-(S)keleton (for a CW complex)
\newcommand{\zs}[1]{#1^{(0)}} %(Z)ero-(S)keleton (for a CW complex)
\newcommand{\ggp}[1]{#1_1} %one-skeleton of a staggered generalized 2-complex
\newcommand{\gos}[1]{#1_{\text{tot}}} %notation for underlaying (G)raph (O)f (S)paces
\newcommand{\rl}{\ell_r} %notation for (R)elative (l)ength
\newcommand{\bsl}[1]{\ell(#1)} %notation for (B)ass-(S)erre (l)ength
\newcommand{\frgsl}[1]{L(#1)} %notation for (f)inite (r)elative (g)en (s)et (l)ength
\newcommand{\da}[1]{\mathcal{A}(#1)} %notatation for (d)iagram (a)rea
\newcommand{\stab}[1]{\text{stab}(#1)} %stabilizer
\newcommand{\nbhd}[2]{\mathcal{N}_{#2}(#1)}%notation for #2-nbhd of #1
\newcommand{\diam}[1]{\text{diam}(#1)} %diameter of a set
\newcommand{\smcan}{C'(\frac{1}{6})} %(s)mall (c)ancellation
\newcommand{\horo}{\mathcal{H}} %horoball
\newcommand{\aug}[1]{A(#1)} %notation for augmentation
\newcommand{\coll}[1]{{#1}_c} %space obtained by collapsing the spanning tree
\newcommand{\ceil}[1]{\lceil#1\rceil} %ceiling

%\newcommand{\simpcon}{simply connected} %simply connected

\renewcommand\labelitemii{\textbullet}

\renewcommand\labelitemiii{\textbullet}

%\pagestyle{fancy}
%\fancyhead{} % clear all header fields
%\renewcommand{\headrulewidth}{0pt} % no line in header area
%\fancyfoot{} %clear all footer fields
%\fancyfoot[RO, LE]{Revision date: \today}   




\newcommand{\sxyz}{((K_x)_y)_z}
\newcommand{\rxy}{(R_x)_y}
\newcommand{\ryx}{(R_y)_x}
\newcommand{\Sxyz}[3]{((K_#1)_#2)_#3}

\begin{document}

%\nocite{*}
%titlepage

%%%%%%%%%%%%%%%%%%%%%%%%%%%%%%%%%%%%%%%%%%%%%%%%%%%%%%%%%%%%%%%%%%%%%%%%%%%%%


\pagenumbering{roman}
{%% PAGE NUMBERING ROMAN

%%%%%%%%%%%%%%%%%%%%%%%%%%%%%%%%%%%%%%%%%%%%%%%%%%%%%%%%%%%%%%%%%%%%%%%%%%%%
%% TITLE PAGE  %%%%%%%%%%%%%%%%%%%%%%%%%%%%%%%%%%%%%%%%%%%%%%%%%%%%%%%%%%%%%%%%%%%%%%%%%%
%%%%%%%%%%%%%%%%%%%%%%%%%%%%%%%%%%%%%%%%%%%%%%%%%%%%%%%%%%%%%%%%%%%%%%%%%%%%
{\singlespacing

\newpage
\thispagestyle{empty}
\begin{center}
{ %\large
\uppercase{UNIVERSITY OF OKLAHOMA}
\par
\vspace{0.16in}
\uppercase{GRADUATE COLLEGE}
\par
\vspace{1.2in}
%%%%%%%%%%%%%%%%%%%%%%%%%%%%%%5
%COMBINATION OF QUASICONVEX SUBGROUPS IN
% If \(G\) has cohomological dimension 2 then there are at most two non-trivial finite type \ref{defn:finitetype} pairwise transverse \ref{defn:transverse} \(G\)-trees up to deformation.
\uppercase{locally finite tree actions for groups of dimension two} 
\par
\vspace{0.17in}
%RIGHT-ANGLED ARTIN GROUPS
\par
\vspace{1.2in}
%%%%%%%%%%%%%%%%%%%%%%%%%%%%%%
\uppercase{A dissertation}
\par
\vspace{0.17in}
\uppercase{Submitted to the graduate faculty}
\par
\vspace{0.17in}
in partial fulfillment of the requirements for the
\par
\vspace{0.17in}
Degree of
\par
\vspace{0.17in}
\uppercase{Doctor of Philosophy}
\par
\vfill
%%%%%%%%%%%%%%%%%%%%%%%%%%%%%
By
\par
\vspace{0.17in}
\uppercase{Anthony Michael Martino}
\par
%\vspace{0.05in}
Norman, Oklahoma
\par
%\vspace{0.05in}
2022
}
\end{center}


%%%%%%%%%%%%%%%%%%%%%%%%%%%%%%%%%%%%%%%%%%%%%%%%%%%%%%%%%%%%%%%%%%%%%%%%%%%%
%% SIGNATURE PAGE  %%%%%%%%%%%%%%%%%%%%%%%%%%%%%%%%%%%%%%%%%%%%%%%%%%%%%%%%%%%%%%%%%%%%%%%%%%
%%%%%%%%%%%%%%%%%%%%%%%%%%%%%%%%%%%%%%%%%%%%%%%%%%%%%%%%%%%%%%%%%%%%%%%%%%%%
\newpage
\thispagestyle{empty}
\ \vspace{0.25in}
\begin{center}
{%\large
\uppercase{locally finite tree actions for groups of dimension two}
\par
\vspace{0.17in}

\par
\vspace{0.5in}

\uppercase{A dissertation approved for the}
\par
%\vspace{0.17in}
\uppercase{Department of Mathematics}
\par
\vspace{1in}
\uppercase{by}
\par
%\vfill
\vspace{2in}
\begin{flushright}
\begin{tabular}{cr}
%\hline
\  \  \  \  \  \  \  & Dr. Max Forester, Chair\\
 %\\ &  \\
  \\ &  \\
%\hline
  & Dr. Noel Brady \\
%\\ & \\
 \\ &  \\
%\hline
 & Dr. Tomasz Przebinda \\
%\\   & \\
 \\ &  \\
%\hline
 & Dr. Jing Tao \\
% \\ &  \\
  \\ &  \\
%\hline
   & Dr. Bruno Uchoa \\
\end{tabular}
\end{flushright}
} 
\end{center}


%%%%%%%%%%%%%%%%%%%%%%%%%%%%%%%%%%%%%%%%%%%%%%%%%%%%%%%%%%%%%%%%%%%%%%%%%%%%
%% COPYRIGHT PAGE  %%%%%%%%%%%%%%%%%%%%%%%%%%%%%%%%%%%%%%%%%%%%%%%%%%%%%%%%%%%%%%%%%%%%%%%%%%
%%%%%%%%%%%%%%%%%%%%%%%%%%%%%%%%%%%%%%%%%%%%%%%%%%%%%%%%%%%%%%%%%%%%%%%%%%%%
\newpage
\thispagestyle{empty}
\   \
\par
\vfill
\begin{center}
\copyright \    Copyright by \uppercase{Anthony Michael Martino} \   2022

All Rights Reserved.
\end{center}
}

\setcounter{page}{3}
%%%%%%%%%%%%%%% DEDICATION
\newpage
%\thispagestyle{empty} %comment this line to make numbering start right after the copyright page
{\singlespacing
\begin{center}
{
\par
\vspace{1.2in}
%\large
DEDICATION
\par
\vspace{0.57in}
to
\par
\vspace{1.2in}
%%%%%%%%%%%%%%%%%%%%%%%%%%%%%%5
<PLACEHOLDER>
%	My Grandma, Marjorie Stucky, and
	\par
	\vspace{0.17in}
%	the memory of
	\par
	\vspace{0.17in}
%	my grandparents Marvin Stucky, LaNae Waltner, and LaVerne Waltner
	\par
	\vspace{0.17in}
%	all of whom showed me the importance of creativity
	\par
	\vspace{0.17in}
	\par
	\vspace{1.2in}
%%%%%%%%%%%%%%%%%%%%%%%%%%%%%%
%
\par
\vspace{0.17in}
%
\par
\vfill
%%%%%%%%%%%%%%%%%%%%%%%%%%%%%
}
\end{center}
}


\newpage

%%%%%%%%%%%%%%%%%%%%%%%%%%%%%%%%%%%%%%%%%%%%%%%%%%%%%%%%%%%%%%%%%%%%%%%%%%%%
%% ACKNOWLEDGEMENTS PAGE  %%%%%%%%%%%%%%%%%%%%%%%%%%%%%%%%%%%%%%%%%%%%%%%%%%%%%%%%%%%%%%%%%%%%%%%%%%
%%%%%%%%%%%%%%%%%%%%%%%%%%%%%%%%%%%%%%%%%%%%%%%%%%%%%%%%%%%%%%%%%%%%%%%%%%%%
\chapter*{Acknowledgments} %TODO: expand acknowledgments

<PLACEHOLDER>
% First and foremost, I wish to thank my adviser, Dr. Max Forester, for his invaluable guidance through the duration of my time as a graduate student and without whom this work would not have been possible. For introducing me to the beautiful world of geometric group theory, for his great generosity with his time, energy, and ideas, and for his thoughtful and lucid teaching style which has been an inpiration to me, I am extremely grateful. 

% I also wish to thank the rest of my advisory committee. Thanks to Dr. Noel Brady for the helpful feedback and probing questions about the thesis. Thanks to Dr. Michael Jablonski for serving on the committee in addition to his many other duties and for all of the professional advice he has given me. Thanks to Dr. Scott Greene for providing such thoughtful feedback about the dissertation and for suggesting the conclusion. Thanks to Dr. Jing Tao for helpful discussions about relative hyperbolicity and relative quasiconvexity as they pertain to the thesis. 

% I wish to thank Drs. Dave Futer, Keri Kornelson, Murad \"{O}zaydin, Ralf Schmidt, Krishnan Shankar, and Sepideh Stewart for their mentorship at various stages during my time as a graduate student. I wish to thank my Bethel College professors Dr. Karl Friesen and Dr. Tim Frye for their teaching and for their continued support. I wish to thank Dr. Lisa Thimm for yelling at me for sleeping during her linear algebra class and for pushing me to attend Budapest Semesters in Mathematics, where I first became interested in graduate school.

% I cannot attempt to list all of the friends to whom I owe thanks for supporting and believing in me throughout graduate school. Whether or not they supported me directly in my studies, I feel strongly that I could not have completed my degree at OU without them. The following list is far from complete.

% I wish to thank Paul Plummer, whose imaginative and inquisitive approach to research helped me to fully make the shift from a view of mathematics as something linear and static which needs to be understood from the ground up to a view of the subject as something more dynamic, the frontiers of which can be probed at a high level in order to search for tractable projects to work on and new connections to make and exploit. Thank you as well for providing feedback about this project and for the TREE$(3)-\varepsilon$ Hanabi games we played throughout graduate school.

% I wish to thank Ignat Soroko, whose work ethic and fearless question-asking and professional connection-making has been an inspiration to me.

% Thanks to all of my other former and current graduate student friends for your support. There are so many names I could list here, and I will get into trouble if I try to list them all, so I will just mention two: Thank you Long Tran and Shaoyun Yi for your friendship and constant support since Day 1 of this journey. We've been through so much and it's a pleasure to be graduating together.

% %Additional thanks to fellow former and current graduate students Shaoyun Yi, Chun-Hsien Lu, Alok Shukla, Siddhesh Wagh, Long Tran, Jieru Zhu, Dania Sheaib, Thomas Lane, Thomas Morgan, Connor Davis, Tony Martino, Michael Tilles, Jonathan Merlini, Jon Epstein, Ryan Reynolds, Jordan Wiebe, Manami Roy, Dustin Gaskins, and Paul Regier for their friendship.

% I wish to thank the faculty and graduate students of Temple University for their hospitality and generosity in providing a place for me to work and discuss mathematics during the 2018 -- 2019 academic year when I was ABD and living in Philadelphia -- you made me feel so welcome in a new place. Thanks in particular to Dave Futer, Thomas Ng, and Kyle Rhoads. Thanks to my high school buddy Jon Hess for being a great friend to me in Philadelphia as well.

% %Dave Futer, Matthew Stover, Edgar Bering, Bach Nguyen, Thomas Ng, Kh\'{a}nh L\^{e}, Tim Morris, Rebekah Palmer, Rose Kaplan-Kelly, Tantrik Mukerji, Brandi Henry, and Kyle Rhoads

% I wish to thank William Lonn for being particularly welcoming to me during my first year at OU, for generously introducing me to all of his friends and for being a great friend himself.

% I wish to thank Ken and Bre Ward for their friendship and for serving as role models to me in my ongoing journey into adulthood. Through their example, they have paved the way for most of the milestones and intellectual shifts in my life since starting college -- from questioning authority and finding my identity to getting married to adopting a pet to pursuing a career in academia -- and I don't know where I would be without them.

% I wish to thank Fred Frances for his friendship since high school, for teaching me to believe in myself, showing me how to better listen to and affirm others, and for sharing his sense of humor with me and with the world.

% I apologize to the friends I've omitted (including but not limited to my other friends from Kansas, Oklahoma, Pennsylvania, South Dakota, BSM, RTCS, PMC, my time in Kenya, math conferences, jazz bands, hockey and ultimate frisbee teams, and music festivals) -- all of you are so important to me and I hope you know who you are.

% I wish to thank my in-laws Pam Bracken, Buddy Johnson, and Nell Johnson for their love and hospitality. Thank you for letting me eat all of your food and also for the many rides to the bus station to get to Norman during trips to Oklahoma in my last year of graduate school.

% I wish to thank my brothers Abe and Sam Stucky for their love and support. Seeing you both persevere to achieve your dreams has been a huge inspiration to me.

% I wish to thank my parents Lynda and Max Stucky for everything that I have. In particular, thank you Mom for teaching me about the powers of positivity and creative problem-solving, and thank you Dad for teaching me to take responsibility for my actions and to communicate clearly and directly.

% Finally, I wish to thank my wife Madeleine for being a constant source of support and inspiration, for her sense of humor, for her practical advice and organizational solutions for achieving my professional goals, and for continually reminding me how important it is to believe in other people. We're so alike in many ways and I guess I'm your beau, huh. Madeleine, I am so grateful to you for always pushing me to be the best person that I can be, for making sacrifices of your time so that I could finish graduate school, and for taking care of me in every way. You have enriched my life tremendously, and I am so glad that I met you. I can't wait to see what else we will accomplish together!


%%%%%%%%%%%%%%%%%%%%%%%%%%%%%%%%%%%%%%%%%%%%%%%%%%%%%%%%%%%%%%%%%%%%%%%%%%%%
%% TABLE OF CONTENTS PAGE  %%%%%%%%%%%%%%%%%%%%%%%%%%%%%%%%%%%%%%%%%%%%%%%%%
%%%%%%%%%%%%%%%%%%%%%%%%%%%%%%%%%%%%%%%%%%%%%%%%%%%%%%%%%%%%%%%%%%%%%%%%%%%%

\newpage
%{\singlespacing
\tableofcontents{}

\newpage
\listoffigures
%\newpage
%}

\newpage


%%%%%%%%%%%%%%%%%%%%%%%%%%%%%%%%%%%%%%%%%%%%%%%%%%%%%%%%%%%%%%%%%%%%%%%%%%%%
%% ABSTRACT  %%%%%%%%%%%%%%%%%%%%%%%%%%%%%%%%%%%%%%%%%%%%%%%%%%%%%%%%%%%%%%%
%%%%%%%%%%%%%%%%%%%%%%%%%%%%%%%%%%%%%%%%%%%%%%%%%%%%%%%%%%%%%%%%%%%%%%%%%%%%

\chapter*{Abstract} %TODO: expand abstract

We investigate finitely generated groups of cohomological dimension 2 and certain actions on locally finite trees. Our setting includes examples of Daniel Wise, which possess two canonical splittings. We show that under suitable hypotheses such a group cannot admit more than two finite-index splittings, up to deformation.

% Since the resolution of the virtual Haken conjecture in the theory of hyperbolic $3$-manifolds, there has been much attention devoted to $\cz$ cube complexes. These non-positively curved metric spaces are powerful tools for understanding infinite, finitely generated groups in part because of their ``cubical'' combinatorics. Simply knowing that a group is cubulable (acts geometrically -- properly and cocompactly by isometries --  on a $\cz$ cube complex) is sufficient to unlock a good deal of structural information about it, and cubulating groups has become an important goal of modern geometric group theory.

% In 2013, Lauer and Wise showed that a one-relator group with torsion whose defining relator has exponent at least $4$ is cubulable. To achieve this, they build a system of nicely-behaved codimension-$1$ subspaces (``walls'') in the universal cover and invoke a construction due to Sageev.

% In this thesis, we achieve a generalization of this result to one-relator products with torsion, namely, that a one-relator product of locally indicable groups whose defining relator has exponent at least $4$ admits a geometric action on a $\cz$ cube complex if the factors do. Our results are framed in the more general context of ``staggered'' quotients of free products of finitely many locally indicable and cubulable groups. The main tools are geometric small-cancellation results for van Kampen diagrams over these groups, which allow us to argue that walls are plentiful and geometrically well-behaved in the universal cover. Relative hyperbolicity of these one-relator products and relative quasiconvexity of wall stabilizers both play a central role.

% Using Agol's theorem that a hyperbolic, cubulable group is virtually special, we obtain as a corollary that the one-relator products we consider are virtually special provided that the factors are hyperbolic in addition to the other assumptions.

\newpage

} %% END PAGE NUMBERING ROMAN

%%%%%%%%%%%%%%%%%%%%%%%%%%%%%%%%%%%%%%%%%%%%%%%%%%%%%%%%%%%%%%%%%%%%%%%%%%%%
%%%%%%%%%%%%%%%%%%%%%%%%%%%%%%%%%%%%%%%%%%%%%%%%%%%%%%%%%%%%%%%%%%%%%%%%%%%%

\pagenumbering{arabic}

\chapter{Introduction}

We are interested in groups acting on locally finite trees. By Bass-Serre theory this corresponds to groups that split as a finite index graph of groups. When working with splittings there is already a notion of similarity given by elementary deformations. We say two graphs of groups lie in the same deformation space if they are related by a sequence of moves that transform edges of the form \(A*_CC\) to a vertex \(A\) or the reverse.

In order to have access to results from homological algebra and make the problem tractable we restrict ourselves to cocompact actions where the stabilizers are groups of type FP. Property FP along with \(FP_n\), \(FP_\infty\), and \(F\) are all related finiteness properties. We will only need Property FP; for examples take groups \(G\) with cocompact finite dimensional \(K(G,1)\) spaces. Our setting then is finite index graphs of groups where the vertex and edge groups are FP and the quotient graph is finite. (Note, finite index implies that if any stabilizer is FP then they all are)

One basic question to ask of a group is its cohomological dimension. Trivial groups have dimension zero, non-abelian free groups have dimension one, and so on. Even at dimension two there are interesting examples that fit within our setup. Daniel Wise invented a kind of 2-complex called a VH-complex and used them to construct groups with no finite quotients \cite{wisethesis}. Burger and Mozes produced infinite, finitely presented, torsion-free simple groups in the same class \cite{burgermozes}.

Our goal is to fix a group \(G\) of dimension two (as this is the first interesting case after free groups) and give a uniqueness result on the number of splittings. Already, Wise has shown that \(G\) comes with two splittings but there is no comment on uniqueness and indeed the torus shows that \(\mathbb{Z}\times\mathbb{Z}\) has an infinite number of splittings all lying in different deformation spaces. We will show that for certain actions there are exactly two splittings up to deformation.

The proof proceeds by encoding two tree actions into a single square complex called Guirardel's core. Then, using a supposed third action we are able to construct a three dimensional complex that can be built up from lower dimensional complexes in such a way that results of Bieri on cohomological dimension apply. The result is an extended core with dimension three that \(G\) acts on; this contradicts the original assumption that \(G\) only had dimension two. 

By analogy with VH-complexes that come with a vertical and horizontal splitting thanks to the work of Wise, our extended core  will satisfy the VHD property (see \ref{dfn:vhd}, here ``D'' stands for ``Depth'') that we introduce below. Not all VHD complexes will have a third splitting but we show our extended core does. Lastly, it's possible to have a VHD-complex where the splittings are in a sense degenerate for our purposes; essentially the splitting process can be iterated until only graphs remain and we require at least one graph of positive rank. We call these primitive \ref{dfn:primitive} VHD-complexes and account for them.

The main result is the following.
\begin{thm}
    [Main Theorem]
    \label{thm:martino}
    If \(G\) is finitely generated and has dimension 2 then either \(G\) is the fundamental group of a primitive VHD complex or there are at most two pairwise transverse deformation spaces of locally finite, finite-type \(G\)-trees.
\end{thm}

% 

We conjecture that primitive VHD complexes all have a fundamental group of \(\mathbb{Z}\times \mathbb{Z}\) so the main conjecture is the following.

\begin{conj}
    [Main Conjecture]
    If \(G\) is finitely generated, dimension 2, and not \(\mathbb{Z}\times \mathbb{Z}\), then there are at most two pairwise transverse deformation spaces of locally finite, finite-type \(G\)-trees.
\end{conj}

\chapter{Preliminaries}

    This section collects general results and fixes terminology.

% \section{Examples}

% (??? work on examples)
% \begin{exa}
% Need an example of a collapse expand move / reduced not reduced / etc.
% \end{exa}

% \begin{exa}
% give the standard Bass-Serre example about SL2Z etc
% \end{exa}

% \begin{exa}
% Spell out the standard picture where FH, FL, FP sort of correspond to the module situation with ZG ... \(ZG^k\) where k is the number of cells and the slot is the dimension of cells

% ...has a cellular chain complex ... free Z modules infinite rank
% ...group into copies of ZG ...



% ===






% \end{exa}





\section{Actions on Trees}

In this section we introduce \(G\)-trees, their equivalence with graphs of groups, and related results.

A \emph{graph} \(\Gamma\) is a set of vertices \(V(\Gamma)\) and edges \(E(\Gamma)\) together with a fixed point free involution on \(E(\Gamma)\) denoted by \(e\mapsto \overline{e}\) and a map \(\partial_0: E(\Gamma)\to V(\Gamma)\). Define \(\partial_1\) via \(\partial_1(e):=\partial_0(\overline{e})\). An edge \(e\) for which \(\partial_0e=\partial_1e\) is called a \emph{loop}. A finite sequence of edges \(e_1, \ldots, e_n\) such that \(\partial_1(e_{k}) = \partial_0(e_{k+1})\) for \(1\leq k\leq n-1\) is called a \emph{path}. If additionally, \(e_{i+1}\neq \overline{e}_i\) then it is a \emph{path without reversals}. A \emph{circuit} is a path without reversals where the starting and ending vertex are the same. A \emph{tree} is a connected graph with no circuits.

A \emph{\(G\)-tree} is an action of a group \(G\) on a simplicial tree without inversions. An element \(g\in G\) is \emph{elliptic} if \(g\) fixes a vertex and \emph{hyperbolic} otherwise. If \(g\in G\) is hyperbolic then there is a line called the \emph{axis} that it acts on by translation.  A subgroup \(H\leq G\) is \emph{elliptic} if there is a vertex that is fixed by all elements of \(H\). We denote the set of all elliptic subgroups by \(\mathcal{E}(G)\).

A \emph{graph of groups} consists of a connected graph \(\Gamma\), groups \(G_v\) indexed by \(V(\Gamma)\), groups indexed by \(E(\Gamma)\) satisfying \(G_e=G_\overline{e}\), and injective homomorphisms \(\phi_e: G_e\to G_{\partial_0(e)}\) indexed by edges. Given a vertex \(v\in V(\Gamma)\) there is a notion of a \emph{fundamental group} \(\pi_1(\Gamma, v)\) as defined in \cite{serretrees}.

A \emph{graph of spaces} consists of a connected graph \(\Gamma\), vertex spaces \(X_v\) indexed by \(V(\Gamma)\), edge spaces \(X_e\) indexed by \(E(\Gamma)\) that satisfy  \(X_e=X_\overline{e}\), and \(\pi_1\)-injective maps \(f_e: X_e\to X_{\partial_0(e)}\) indexed by edges. The \emph{total space} of a graph of spaces is the quotient \[ \left. \left(\bigsqcup_{v\in V(\Gamma)} X_v \sqcup \bigsqcup_{e\in E(\Gamma)} X_e\times [0,1]\right) \middle/ \sim \right.} \] where \(\sim\) is given by identifying the following: \[X_e\times [0,1] \to X_{\overline{e}}\times [0,1] \text{ by } (x,t)\mapsto (x, 1-t)\]\[ X_e\times 0\to X_{\partial_0(e)}\) \text{ by } \((x,0)\mapsto f_e(x).\]

These definitions work well together in the following sense. Given a graph of groups with fundamental group \(G\) it is possible to construct a graph of spaces where the fundamental group of the total space is also \(G\). We also recover two special cases of interest. A graph of groups with exactly two vertices and one edge corresponds to an amalgamated free product \(A*_CB\); should the single edge be a loop then it corresponds to an HNN extension \(A*_C\). These special cases can be seen as applications of Van Kampen's theorem.

\begin{dfn}
    [Minimal Action]
    \label{defn:minimal}
    We say a $G$-tree is \emph{minimal} if there is no proper invariant subtree.
\end{dfn}

Under the correspondence between \(G\)-trees and graphs of groups, minimality is equivalent to requiring that every proper subgraph corresponds to a proper subgroup of \(G\).

From \cite{boundingcomplexity} we have the following classification of \(G\)-trees. Note, any tree action can be made to act without inversions by subdividing once.
\begin{thm}
    [\(G\)-tree Classification]
    \label{thm:classification}
    Let \(G\) be a finitely generated group and \(T\) a minimal \(G\)-tree acting without inversions. Then \(T\) satisfies exactly one of the following:
    
    \begin{enumerate}
        \item Elliptic: \(T\) is a point or equivalently every element of \(G\) is elliptic.
        \item Dihedral: \(T\) is a line and there exists an epimorphism from \(G\) to the infinite dihedral group.
        \item Parabolic: \(T\) has one end fixed by \(G\). Additionally, there is an epimorphism from \(G\) to \(\mathbb{Z}\).
        \item Hyperbolic (or irreducible): There exists two axes in \(T\) with compact intersection. When this happens, \(G\) contains a non-abelian free group.
    \end{enumerate}
\end{thm}

Given a \(G\)-tree, or equivalently a graph of groups, a \emph{collapse move} produces a new graph where an edge corresponding to \(A*_CC\) is collapsed to a vertex \(A\). The reverse is called an \emph{expansion move}. An \emph{elementary deformation} is a finite sequence of such moves. We say a \(G\)-tree is \emph{reduced} when one can no longer perform any collapse moves. For details and examples see \cite{foresterdeformationrigidity}.

\begin{dfn}
    [Deformation Space]
    We say two \(G\)-trees \(T_1\) and \(T_2\) are in the same \emph{deformation space} if there is an elementary deformation from \(T_1\) to \(T_2\).
\end{dfn}




% "Note that if a G–tree (or a graph of groups) is reduced then it is minimal." \cite{foresterdeformationrigidity}

% In the lemma below, do we need cocompactness? (no) -- but it's easier and we can cite a nice  Bass Covering Theory paper stuff

% Update: Still looking for this result from the bass paper

% )
\begin{lem}
    \label{reducedcocompact}
    If \(X\) is a \(G\)-tree that is reduced then it is minimal
    \begin{proof}
    One can show that if \(e\) is an edge outside of a \(G\)-invariant subtree, then it admits an elementary collapse move. For details see \cite{foresterdeformationrigidity}.
    \end{proof}
\end{lem}

The following is a corollary of the main theorem from \cite{foresterdeformationrigidity} that relates elliptic subgroups and deformations.

\begin{thm} [Forester]
    \label{thm:forester}
    Let \(G\) be a group and let \(X\) and \(Y\) be cocompact \(G\)–trees. The following conditions are equivalent.
    \begin{enumerate}
        \item \(X\) and \(Y\) are related by an elementary deformation. 
        \item \(X\) and \(Y\) have the same elliptic subgroups.
    \end{enumerate}
If all vertex stabilizers are finitely generated then we may also include:
\begin{enumerate}
    \setcounter{enumi}{2}
    \item \(X\) and \(Y\) have the same elliptic elements.
\end{enumerate}
\end{thm}

\begin{exa}

An important example: The Torus has infinitely many graph of spaces decompositions each lying in a different deformation space. To see this, note that simple closed curves on the torus correspond to graph of spaces decompositions with the curve as the edge space. Up to homotopy, such curves correspond to extended rationals e.g. r/s, 1/0, and 0/1 each telling you how many times a curve wraps around both directions of the torus. The wrapping numbers correspond to coordinates of \(\mathbb{Z}\times \mathbb{Z}\), so r/s corresponds to the cyclic subgroup \(\langle (r,s)\rangle\leq \mathbb{Z}\times \mathbb{Z}\). With this perspective, \(\mathbb{Z}\times \mathbb{Z}\) can be viewed as an HNN extension (i.e. a graph of groups with exactly one edge and one vertex) where the vertex and edge groups are \(\langle (r,s)\rangle\). The elliptic subgroups are exactly the vertex group and its subgroups therefore by  \ref{thm:forester} each splitting is in a different deformation space.




% Non-example: The Torus has infinitely many decompositions all in different deformation spaces

% The torus is Z x Z, or if you like it's Z with the generator glued to itself via the identity (the HNN version). 
% From the Z x Z perspective, a pair of extended (i.e. include infinity so we allow 1/0 and 0/1) rationals p/q and r/s correspond to a basis for Z x Z if [[p,r],[q,s]] has determinant 1 so that < (p,q), (r,s) > contained in Z x Z is a basis.

% Z x Z isom pi1(torus) isom HNN of < (p,q) > over < (p,q) >

% Simple closed curves on the torus correspond to graph of space decompositions of the torus with the curve as the edge space.

% It's easy to see that the one vertex group is different because it's just powers of a single element in ZxZ; the (p,q).

% Conjecturally, this is sort of the only non-example


\end{exa}


If one \(G\)-tree contains the elliptic subgroups of another it is possible to produce a \(G\)-map between the two trees. This map in turn can be analyzed using Stallings folds as in \ref{lem:gmapfactor}.
\begin{prop}
[Equivariant Map equivalent to elliptic subgroup containment]
\label{pro:gmapfromsubset}
Suppose $X$ and $Y$ are simplicial \(G\)-trees. Then the following are equivalent:
\begin{enumerate}
    \item \(\mathcal{E}(X) \subseteq \mathcal{E}(Y)\)
    \item There exists a simplicial $G$-map from a subdivision of $X$ to $Y$.
\end{enumerate}
\end{prop}
\begin{proof}
    For (1)$\Rightarrow$(2) we will construct a $G$-map. As in \ref{lem:affineequivariantmap} we start by defining a map on representatives of vertex orbits and extending equivariantly. Pick a vertex \(x\in X\), by (1) \(G_x\) is also elliptic in \(Y\) and so fixes at least one vertex, say \(y\). Define \(f(x)=y\) and \(f(gx) = gy\) for all \(g\in G\). Next, we check that the resulting map is well-defined.
    
    Indeed, if \(gv = hv\) then \(g^{-1}h = v\) which means \(g^{-1}h \in G_v\). Then by the definition of \(f\) the element \(g^{-1}h\) fixes \(f(v)\in Y\). Hence,
    
    \[f(gv) = gf(v) = g(g^{-1}h)f(v) = hf(v) = f(hv).\]
    
    Once defined on vertices, define on edges by sending the edge \(vw\) to the unique path in \(Y\) from \(f(v)\) to \(f(w)\). To make this definition work on edges it may be necessary to subdivide the edge \(vw \in X\). If the endpoints of an edge go to the same vertex, send the entire edge to that vertex. 
    
    For (2)$\Rightarrow$(1), suppose $f:X\to Y$ was a simplicial $G$-map. Consider a vertex group of $X$, say $G_x$. Let $g\in G_x$, then $g\cdot f(x)=f(g\cdot x)=f(x)$ a vertex in $Y$ since the map is simplicial. Hence, $G_x$ also fixes the vertex $f(x)\in Y$. Hence, every vertex group for $X$ fixes a vertex of $Y$. Therefore, every elliptic subgroup for $X$ is also elliptic for $Y$.
\end{proof}

Bass-Serre Theory is responsible for the correspondence between \(G\)-trees and graph of groups decompositions of \(G\). The original construction creates a tree from cosets but the following definition is inspired by the topological reformulation given by Scott and Wall in \cite{scottwall}.
\begin{dfn}
    [Bass-Serre Map]
    \label{defn:bsmap}
    The universal cover of a graph of spaces is a tree of spaces. Define a map to the underlying tree taking $X_v$ to the vertex \(v\) and $X_e \times I$ to the edge \(e\) by collapsing onto the second factor. This map is called a Bass-Serre map.
\end{dfn}

Here we list some technical definitions that are not part of the standard background for \(G\)-trees but are needed here or appear in results we cite.

To start, we are most interested in \(G\)-trees of the following type.
\begin{dfn}
    [Finite Type]
    \label{defn:finitetype} 
    We say a $G$-tree is of \emph{finite type} if the tree is locally finite, the vertex stabilizers have property FP, and the quotient is a finite graph.
\end{dfn}

Next, we fix terminology for discussing Guirardel's work from \cite{guirardelcorepaper}.

\begin{dfn}
    [Directions and Halfspaces]
    \label{dfn:directionhalfspaces}
    A \emph{direction} based at a point \(p\in T\) where \(T\) is a tree is a connected component of \(T\smallsetminus p\). If \(p\) is a midpoint of an edge then we call the resulting directions \emph{halfspaces}.
    
    A \emph{halfspace in a product of trees} is a preimage of a direction of a tree under projection. A \emph{quadrant} in \(T_1\times T_2\) based at \((x,y)\) is the product of two directions at \(x\) and \(y\) respectively or equivalently the intersection of two halfspaces, one from each factor. An \emph{orthant} in \(T_1\times T_2\times T_3\) based at \((x,y,z)\) is the product of three directions at \(x\) and \(y\) and \(z\) respectively or equivalently the intersection of three halfspaces, one from each factor.
\end{dfn}
 
 Finally, there is a notion of a geometric action on an \(\mathbb{R}\)-tree. The full details do not concern us but \cite{levitt} gives a condition \ref{lem:simpgeo} that applies when specialized to our setup.
 
\begin{lem}
    [Geometric Condition]
    \label{lem:simpgeo} 
    (Theorem 0.6 in \cite{levitt})
    A minimal simplicial action of a finitely generated group is geometric if and only if all edge groups are finitely generated.
\end{lem}

\section{Folding} 

Here we define a special kind of map between \(G\)-trees called a fold, first introduced by Stallings in \cite{stallingsfolds}. Later we will factor \(G\)-maps into a sequence of folds and analyze each fold separately.

\begin{dfn}
    [Folding]
    \label{defn:folding}
    Folding is an operation on \(G\)-trees without inversions. Essentially, two edges sharing a vertex are folded together followed by folding their translates together equivariently. After folding a new \(G\)-tree is produced, denote it by \(T/{\sim}\). It may happen that \(T/{\sim}\) inverts some edge but this can be remedied by first subdividing the original \(G\)-tree \(T\).
    
    Let \(e_1\) and \(e_2\) be edges with a common vertex \(v\) and denote the vertices opposite to \(v\) by \(u_1\) and \(u_2\) respectively. Suppose that in the quotient, \(e_1\) and \(e_2\) are not loops, that is \(v\not\in Gu_1\cup Gu_2\). Then folding \(e_1\) and \(e_2\) together will be called a \emph{type A} fold. The other types of folds will not concern us as they are all compositions of type A folds and subdivisions. For details see \cite{boundingcomplexity} and \ref{exa:folding}.
\end{dfn}

\begin{exa}
[Type IA]
\label{exa:folding}
Suppose the orbit map embeds \(e_1\cup e_2\). Then the fold will force a new identification. See figure \ref{fig:typeIA}.
\begin{figure}[!h]
    \label{fig:typeIA}
    \centering
    \includegraphics[width=10cm]{typeIA.png}
    \caption{The left side shows \(T\) and \(T/G\). The right side shows the new \(G\)-tree \(T/{\sim}\) and the resulting quotient \( (T/{\sim})/G\) after a type IA fold. Stabilizers are written in blue.}
\end{figure}
\end{exa}

\begin{exa}
[Type IIIC]
\label{exa:folding2}
Suppose there exists some \(g\in G\) that identifies \(e_1\) with the reverse of \(e_2\), then the new edge \(e_1/{\sim} =  e_2/{\sim} \) is inverted by \(G\).  This can be avoided by first subdividing.
\end{exa}

\newpage
\begin{dfn}
    [Morphism]
    A \emph{morphism} is a map between graphs where vertices go to vertices and edges go to edges. (Alternatively, this is a simplicial map where no edges are collapsed)
\end{dfn}

\begin{dfn}
    [Collapse Map]
    A \emph{collapse map} is a map between two trees obtained by quotienting each connected component of a union of subtrees to a point. (Alternatively, it's a simplicial map where the preimage of every vertex is connected)
\end{dfn}

We need a result from Bestvina and Feighn on factoring morphisms.
\begin{thm}[Bestvina-Feighn \cite{boundingcomplexity}]
    \label{thm:folds} 
    Let $G$ be a finitely generated group. Suppose that $\alpha: T'\to T$ is a simplicial equivariant map from a $G$-tree $T'$ to a minimal $G$-tree $T$ such that no edge in $T'$ is mapped to a point by $\alpha$. If all edge stabilizers of $T$ are finitely generated and if $T'/G$ is finite, then $\alpha$ can be represented as a finite composition of folds.
\end{thm}

The following lemma lets us fully factor our \(G\)-maps.
\begin{lem}
    \label{lem:gmapfactor}
    Every surjective \(G\)-map between trees factors as a collapse map, followed by a morphism.
    \begin{proof}
        Consider all edges that map to vertices. The connected components of this set of edges constitutes a forest which gives the collapse map.
    \end{proof}
\end{lem}
% really it's collapse, morphism, immersion but who cares we just need the onto case



% (??? Found a reference about minimality:


% ... proof is easier if cocompact due to result of bass

% ... bass: minimal <=> condition on quotient graph

% ... if you're not minimal then in the quotient you are a core graph + collapsible trees (where every edge is collapsible but also maybe infinite)


% ..... can also prove using ... invariant tree and think about what a random edge has to do

% ===


\section{Cohomological Dimension and Groups of Type FP}


We write \(A\) to mean an arbitrary ring with a unit. Take \(R\) to be a non-zero commutative ring with a unit. A resolution of a module \(M\) is a sequence of modules \((P_k)_{0\leq k\leq \infty}\) and homomorphisms \((\phi_k)_{0\leq k\leq \infty}\) where \[\cdots \to P_0 \overset{\phi_0}{\to} M\] is exact. We say a resolution \[\cdots 0\to P_n\to \cdots\to P_0 \to M\] has length \(n\) if \(n\) is the greatest integer such that \(P_n\) is a non-zero module.

\begin{dfn}
    [Projective Module]
    A module \(P\) is projective if for every surjective module homomorphism \(f: N\to M\) and for every module homomorphism \(g: P\to M\) there exists a module homomorphism \(h: P\to N\) such that \(fh = g\).
\end{dfn}

\begin{dfn}
    [Projective Resolution]
    A \emph{projective resolution} \((P_k)_{0\leq k\leq \infty}\) of a module \(M\) is a resolution of \(M\) where all the \(P_k\) are projective modules.
\end{dfn}

\begin{dfn}
[Cohomological Dimension]
The cohomological dimension of a group \(G\) is given by:
\[
    \text{cd}(G) = \text{inf}\{n\mid\text{The trivial }\mathbb{Z}G\text{-module }\mathbb{Z}\text{ admits a projective resolution of length \(n\) over \(\mathbb{Z}\)G}\}
\]
\end{dfn}

\begin{dfn}
[Property \(FP_n\) for a Module]

A module \(M\) over \(A\) has property \(FP_n\) with \(0\leq n\leq \infty\) if there is a projective resolution \((P_k)_{0\leq k\leq \infty}\) of \(M\) where the \(P_k\) are finitely generated for all \(0\leq k\leq n\).
\end{dfn}

\begin{rmk}
The property \(FP_n\) generalizes familiar properties. For a module \(M\), being finitely generated is equivalent to having property \(FP_0\), being finitely presented is equivalent to having property \(FP_1\).
\end{rmk}

\begin{dfn}
[Property \(FP_n\) for a Group]

A group \(G\) has property \(FP_n\) over a ring \(R\) with \(0\leq n\leq \infty\) if the trivial \(RG\)-module \(R\) is of type \(FP_n\) as an \(RG\)-module.
\end{dfn}

For our purposes we will only need the case where \(R\) is the integers \(\mathbb{Z}\) and will therefore omit the ring. Our resolutions will then be taken over the group ring \(\mathbb{Z}G\).

\begin{prop}
% (Bieri p19)
A group \(G\) is finitely generated if and only if \(G\) is of type \(FP_1\) over \(R\).
\end{prop}

\begin{dfn}
[Property FP]
% (Bieri p55)
A module is of type \(FP\) if there exists a finitely generated projective resolution of finite length. This is equivalent to being of type \(FP_\infty\) and having finite cohomological dimension.
\end{dfn}

\section{VH-complexes}

A \emph{cube complex} is a CW-complex consisting of cubes glued together via isometries between their faces. Here an \emph{\(n\)-cube} is a copy of \([-1,1]^n\) with the Euclidean metric, a \emph{\(k\)-face} is a subset of an \(n\)-cube where all but \(k\) coordinates are restricted to \(-1\) or \(1\), a \emph{midcube} of a cube \([-1,1]^n\) is the result of restricting exactly one coordinate to zero. A \emph{hyperplane} of a cube complex comes from taking a single midcube and extending it along glued faces of the cube complex to include other midcubes. More specifically, a hyperplane is a minimal closed subset whose intersection with any closed cube is a union of midcubes. For details see \cite{manning}, \cite{haglundwise}.

There are several notions of what it means for a space to have non-positive curvature. For a cube complex there is a local combinatorial condition.

\begin{dfn}
    [Link of a vertex]
    Let \(X\) be a cube complex. The \emph{link of a vertex \(v\)} is the complex formed by taking a small spherical neighborhood of the vertex \(v\).
\end{dfn}

\begin{dfn}
    [Flag simplicial complex]
    A simplicial complex \(K\) satisfies the \emph{flag simplicial} property if every complete subgraph of the 1-skeleton of \(K\) bounds a simplex 
\end{dfn}

\begin{dfn}
    [Non-positively curved (NPC)]
    A cube complex \(X\) is \emph{non-positively curved (NPC)} if the link of every vertex is simplicial and satisfies the flag condition. If \(X\) is also simply connected then we say it is \emph{CAT(0)}.
\end{dfn}


\begin{dfn}
    [Square Complex]
    A \emph{square complex} is a 2-complex whose 2-cells are attached by combinatorial paths of length 4.
\end{dfn}

\begin{dfn}
    [VH-complex]
    A square complex $X$ is a \emph{VH-complex} if the following hold,
    \begin{enumerate}
        \item the link at each vertex is simplicial;
        \item each edge is labelled vertical or horizontal;
        \item each hyperplane is two-sided;
        \item attaching maps alternate between horizontal and vertical edges.
    \end{enumerate}
    Let $V$ and $H$ denote the set of vertical and horizontal edges. Take $V_X = V \cup X^{(0)}$ and $H_X = H \cup X^{(0)}$ to be the \emph{vertical} and \emph{horizontal} skeletons. 
\end{dfn}

\begin{rmk}
Note that the link of every VH-complex is a bipartite graph; this follows from conditions (2) and (3), see also remark 1.4 from Wise \cite{wisethesis}. In dimension 2 the NPC property for a cube complex is equivalent to requiring that all links be simplicial graphs i.e. no double edges or loops. Wise does not require his VH-complexes to be NPC. However, for convenience, we assume property (1) and so our VH-complexes will be NPC. The link condition also generalizes to an object we need later called a VHD-complex (see \ref{dfn:vhd}).
\end{rmk}

\begin{dfn}
    [Two-sided Hyperplanes]
    A hyperplane is \emph{two-sided} if there is a choice of orientation on each edge dual to the hyperplane that is consistent across squares.
\end{dfn}

\begin{rmk}
    An embedded hyperplane \(H\) is \emph{two-sided} if and only if the open cubical neighborhood is homeomorphic to \(H\times (-1,1)\). This orientation condition can always be achieved by subdividing a cube complex twice which allows us to avoid any twisting (e.g. Mobius bands) and ensures our cube complex will decompose easily into a graph of spaces.
\end{rmk}

The following is necessary to state an important result of Wise about the existence of splittings for VH-complexes. The notion of a  foliation will appear in Guirardel's work as well.
\begin{dfn}
    [Decomposition Graph]
    \label{dfn:decompositiongraph}
    
    Given a VH-complex \(X\) we define a map \(\rho: X\to \Gamma_X\) from \(X\) to a graph. The vertices of \(\Gamma_X\) are defined to be the connected components of \(V_X\) which we call vertex spaces and the edges are given by the connected components of \(X\smallsetminus V_X\). After foliating each square with vertical segments, given \(x\in X\) define \(V_x\) to be the smallest subset of \(X\) that contains \(x\) and any vertical segment that it intersects. Because \(X\) has two-sided hyperplanes there are no singular leaves and \(V_a\) and \(V_b\) are parallel to each other if \(a\) and \(b\) lie in the interior of the same horizontal edge. The set of all \(V_x\) then gives a foliation of \(X\). Collapsing each leaf of the foliation is enough to define a map to \(\Gamma_X\) viewed as a 1-dimensional cell complex. 
\end{dfn}

The following is a restatement of theorem 2.16 \cite{wisethesis} from Wise:
\begin{thm}[Wise Graph Decomposition]
\label{thm:wisegraph}
Suppose \(X\) is a VH-complex. Then the vertical and horizontal decomposition graphs \ref{dfn:decompositiongraph} each determine a splitting of \(\pi_1(X)\) as a graph of free groups. 
\end{thm}

\chapter{Locally finite trees and elliptic subgroups}

In this section we prove various utility lemmas related to locally finite trees. We also give a key sufficient condition for when elliptic subgroup containment can be upgraded to equality.

\begin{prop}
    [Hyperbolic gives unique minimal tree]
    \label{pro:uniquemintree}
    
    If $G$ is acting on a tree $X$ and contains a hyperbolic element then there is a unique minimal subtree equal to the union of all hyperbolic axes. In particular it is non-empty.
\end{prop}
\begin{proof}
See Proposition 3.1 from \cite{hymanbass}.
\end{proof}
\begin{prop}
    [Commensurable groups have the same minimal tree]
    \label{pro:commintree}
    Suppose $G$ acts on a tree $X$ and $H$ and $K$ are commensurable subgroups. If $H$ contains a hyperbolic element, then so does $K$ and the minimal subtrees for $H$ and $K$ are equal.
\end{prop}
\begin{proof}
    This follows from \ref{pro:uniquemintree}, for details see Corollary 7.7 from \cite{hymanbass}.
\end{proof}

The reader should think about a sequence of folds from \ref{thm:folds} while reading the statements of \ref{lem:foldingpreserveshyp}, \ref{lem:pullbacklocallyfinite}, and \ref{lem:preservehyperbolicity}. Combined these lemmas allow us to inductively push properties across folds.

\begin{prop}
    [Folding preserves properties]
    \label{lem:foldingpreserveshyp}
    Suppose $X$ and $Y$ are $G$-trees. In addition suppose $Y$ has a locally finite minimal subtree that is not a point. Let $\phi:X\to Y$ be a type A fold. Then if an element of $G$ is hyperbolic for $X$ it is also hyperbolic for $Y$.
    
    \begin{proof}
        For the fold $\phi$ pick two edges $e$ and $e'$ that are folded and adjacent to some $v$. The element $g$ is hyperbolic and so comes with an axis that has a positive translation length, hence $Gv$, the orbit of $v$ is infinite. Because $\phi$ is a type A fold, $G v$ is taken injectively to $Y$. That is, $G(\phi (v))$ is infinite. If $G(\phi (v))$ intersects $Y_\text{min}$ then $G(\phi v)$ is entirely contained in $Y_\text{min}$ because minimal trees are $G$-invariant. However, $g$ is elliptic in $Y$ and we are acting by isometries so the infinte set $G(\phi (v))$ lies within a bounded distance of $\phi (v)$ in $Y_\text{min}$; but this is impossible becauase $Y_\text{min}$ is locally finite. That is, $G(\phi (v))\subset Y\smallsetminus Y_\text{min}$. 
        
        Consider the images of the folded edges $e$ and $e'$ along with their orbits. 
        Folds preserve adjacency so every edge in $G(\phi (e))$ (which is equal to $G(\phi (e'))$) is adjacent to a vertex in $G( \phi (v))$. In particular, this means at least one vertex of each edge in $G(\phi (e))$ and $G(\phi (e'))$ is outside of $Y_\text{min}$. 
        Therefore, $G(\phi (e))$ and $G(\phi (e'))$ are not contained in $Y_\text{min}$. 
        Said differently, $\phi( Ge\cup Ge')$ is disjoint from $Y_\text{min}$.
        
        Finally, since $Y_\text{min}$ is not a point, it contains an edge, call it $f$. Consider an edge $\hat{f}$ that maps to $f$. Since $f$ is contained in $Y_\text{min}$ it is disjoint from $\phi( Ge\cup Ge')$ and therefore $\hat{f}$ is not part of a fold and therefore \(G\hat{f}\) goes injectively. But this is impossible because then $Gf$ is an infinite set within a bounded distance of a single point in a locally finite tree.
    \end{proof}
\end{prop}

\begin{lem}
    
    
    
    Let $X$ and $Y$ be $G$--trees in the same deformation space and
suppose that $X$ contains a locally finite subtree that is
$G$--invariant. Then $Y$ also contains a $G$--invariant locally finite
subtree. 
    \begin{proof}
        It suffices to consider a single elementary collapse move \(q\colon X \to Y\)
along the edge $e \in E(X)$ with $G_{\partial_0 (e)} = G_e$. First
suppose that $X$ has a $G$--invariant subtree $X'$. Let $Y' = q(X')$,
a $G$--invariant subtree of $Y$. If $e \not\in X'$ then $X'$ maps
isomorphically to $Y'$, and so $Y'$ is locally finite. Otherwise, 
the restriction $q\colon X' \to Y'$ is an elementary collapse. In the
proof of \cite[Theorem 7.3]{foresterdeformationrigidity} it was observed that $q$ is a
$(3, 2/3)$--quasi-isometry, and that $\frac{1}{3}(d(x,x') - 2) \leq
d(q(x),q(x'))$ for all $x,x'\in X'$. It follows that the pre-image of
a ball of radius $1$ is contained in a ball of radius $5$ in $X'$. The
latter is finite, and so every ball of radius $1$ in $Y'$ is finite. 

Next consider the same collapse move $q\colon X \to Y$ but suppose that
$Y$ contains a locally finite $G$--invariant subtree
$Y'$. We wish to find the same in $X$. Let $Z \subset X$ be the
$G$--invariant subgraph whose edges are $\{ e' \in E(X) - (Ge \cup
G\overline{e}) \mid q(e') \in E(Y')\}$. If $Z$ is connected then let
$X' = Z$; it maps isomorphically to $Y'$ by $q$ and hence is locally
finite. 


Otherwise, $\partial_0(e)$ is in $Z$. We define $X'$ to be
$Z \cup (Ge \cup G\overline{e}) = q^{-1}(Y')$. For any vertex
$v\in V(X')$ consider the ball $B_v(1)$ of radius $1$ at $v$. Edges of
$Z$ in $B_v(1)$ map injectively to $Y'$, so there are finitely
many. It remains to bound the number of edges of $Ge \cup
G\overline{e}$ in $B_v(1)$. 
Note that each component of $Ge \cup G\overline{e}$ is a cone on some
subset $S \subset G \partial_0(e)$ (with cone point in $G
\partial_1(e)$). Each vertex of $S$ is incident to an edge of $Z$
and these edges are all distinct. Hence $S$ is finite, because
collapsing $e$ results in a locally finite tree. Hence $Ge \cup
G\overline{e}$ is locally finite, and therefore $B_v(1)$ is finite. 
    \end{proof}
\end{lem}

From the above lemma we get the following corollary.
\begin{cor}
    \label{lem:pullbacklocallyfinite}
    If \(X\) and \(Y\) are in the same deformation space and \(Y_\text{min}\) is is locally finite then \(X_\text{min}\) is also locally finite.
\end{cor}

\begin{lem}
    [Collapse map preserves hyperbolic elements]
    \label{lem:preservehyperbolicity}
    Suppose \(X \to Y\) is a collapse map with \(Y\) locally finite. Suppose \(Y\) is not a single point. Then, if an element is hyperbolic for \(X\) it is also hyperbolic for \(Y\).
    \begin{proof}
        Suppose for sake of a contradiction that \(g\in G\) were hyperbolic for \(X\) and elliptic for \(Y\). Let \(y \in Y\) be some vertex fixed by the elliptic element \(g\) and \(G_y\) it's stabilizer. Since \(Y\) is not a single point, there is another vertex \(y\neq z \in Y\). Because \(Y\) is locally finite, \(G_y\) and \(G_z\) are commensurable. For \(G\)-maps, pre-images are invariant. By the construction of a collapse map, the preimage of vertices are connected and non-empty. Putting these together we have that the preimages of vertices are invariant trees. This means that the minimal subtrees of \(G_y\) and \(G_z\) acting on \(X\) are contained in the disjoint preimages of \(y\) and \(z\) respectively. However, since they are commensurable and \(G_y\) contains the hyperbolic element \(g\), these minimal trees are non-empty and equal by \ref{pro:commintree}. This is a contradiction.
    \end{proof}
    % \begin{proof}
    %     (Proof in Style 2)
    %     For sake of a contradiction suppose $g\in G$ acts hyperbolically on $X$ but elliptically on $Y$.  Then $g$ fixes some vertex $y\in Y$.  Consider the stabilizer $G_y$ acting on $X$. The tree $f^{-1}(y)$ is stabilized by $G_y$ because the map is a $G$-map. This means the minimal tree for the $G_y$ action on $X$ is contained in $f^{-1}(y)$. It's non-empty because $g\in G_y$ is hyperbolic for $X$.
        
        
    %     Since $Y$ is not a single point, there exists some vertex $z\in Y$ with $z\neq y$. As before, $G_z$ stabilizes the tree $f^{-1}(z)$ so the minimal tree for the $G_z$ action on $X$ is contained in $f^{-1}(z)$. By local finiteness and the orbit-stabilizer theorem applied to the set of edges adjacent to a given vertex, $G_y$ and $G_z$ are commensurable. By \ref{pro:commintree} the minimal tree for $G_y$ is non-empty and equal to the minimal tree for $G_z$. Except now the minimal tree for $G_y$ acting on $X$ is contained in two non-empty disjoint sets $f^{-1}(y)$ and $f^{-1}(z)$ a contradiction.
    % \end{proof}
\end{lem}


We already saw that being in the same deformation space is equivalent to having the same elliptic subgroups. This further reduces the problem to having the same elliptic elements. 

\begin{thm}
    [Elliptic elements determine elliptic subgroups]
    \label{thm:ellelesubgroups} 
    Let \(X\) and \(Y\) be cocompact \(G\)-trees with finitely generated vertex groups. Then the following are equivalent:
    \begin{enumerate}
        \item \(X\) and \(Y\) define the same partition of \(G\) into elliptic and hyperbolic elements.
        \item \(X\) and \(Y\) have the same elliptic subgroups.
    \end{enumerate}

    \begin{proof}
    By Proposition 2.6, Theorem 4.2, and Corollary 4.3 of \cite{foresterdeformationrigidity}.
    \end{proof}
\end{thm}

\begin{rmk}
    For an interesting example of how the conclusion of \ref{thm:ellelesubgroups} may not hold when a group fails to be finitely generated consider a particular subgroup of the Baumslag-Solitar group \(BS(1,2)\) acting on a trivalent tree with a fixed end. One can produce a group isomorphic to the dyadic rationals along with two tree actions where every element is elliptic but the elliptic subgroups differ. For details see \cite{foresterdeformationrigidity}. 
\end{rmk}

The following is used whenever we need to apply Bieri's results on dimension iteratively. This happens once during the transverse construction \ref{pro:transverseconstruction} as well as part of the final argument \ref{lem:iteratedsplitting} in the proof of the main theorem. The proof works by factoring a tree map into a sequence of folds and showing that the decomposition of the group into elliptic and hyperbolic elements is preserved across folds. In our setting, this decomposition is enough to force the elliptic subgroups to be preserved as well.

\begin{thm}
    [Elliptic containment implies equality]
    \label{thm:ellipticimpliesequality} 
    If \(X\) and \(Y\) are locally finite cocompact \(G\)-trees with finitely generated vertex and edge stabilizers and \(Y\) has no global fixed point then \(\mathcal{E}(X) \subseteq \mathcal{E}(Y) \Longrightarrow \mathcal{E}(X) = \mathcal{E}(Y) \). 
\begin{proof}

    Let \(X\) and \(Y\) as in the hypotheses. Without loss of generality we may assume that \(X\) and \(Y\) are reduced and therefore minimal by \ref{reducedcocompact}. By \ref{pro:gmapfromsubset} there exists a \(G\)-map from \(X\) to \(Y\). This \(G\)-map is onto because \(Y\) is minimal and the image of a \(G\)-map is an invariant set. Using \ref{lem:gmapfactor} we can factor the \(G\)-map into a collapse map followed by a morphism. Apply \ref{thm:folds} to factor the morphism into a finite sequence of folds. From a remark in \cite{boundingcomplexity} after possibly subdividing we can take all of the folds to be of type A. Collapse maps and folds preserve the finite generation of vertex and edge stabilizers. In the next step we repeatedly apply \ref{lem:pullbacklocallyfinite}. Using \ref{lem:preservehyperbolicity} and \ref{lem:foldingpreserveshyp} we see each stage of the composition starting from the right preserves the property of hyperbolicity for an element \(g\in G\).  Hence, \(X\) and \(Y\) partition \(G\) into the same elliptic and hyperbolic elements. Under our setup \ref{thm:ellelesubgroups} applies, therefore \(X\) and \(Y\) have the same elliptic subgroups as needed.
\end{proof}
\end{thm}

\chapter{Transverse deformation spaces}

In this section, we introduce the transverse property and use it to  construct a VH-complex with two predetermined splittings. This will be the first step in bootstrapping the construction of the extended core.

\begin{dfn}
    [Transverse]
    \label{defn:transverse} 
    We say that two locally finite \(G\)-trees \(X\) and \(Y\) are \emph{transverse} if they are not in the same deformation space and there exist two vertex stabilizers, one for each tree, such that their intersection has Property FP.
\end{dfn}

\begin{rmk}
    The definition of transverse does not depend on the vertices chosen and remains unchanged up to deformation. \new{From the definition of transverse one gets that \(G_x\cap G_y\) is FP for a specific \(x\) and \(y\). Consider \(G_{x'}\). Local finiteness of \(X\) implies that \(G_{x'}\) and \(G_x\) are commensurable so they share a finite index subgroup. Taking the subgroup diagram and intersecting everything with \(G_y\) and checking the inclusions are still of finite index gives that \(G_{x'}\cap G_y\) is commensurable with \(G_x\cap G_y\) and so also must be FP. A similar argument works for vertices \(y'\) in the second tree.
    An elementary deformation either pulls a subgroup out into a new vertex group or pushes one back in. In either case, the tree remains transverse to the same trees.}
\end{rmk}

Given two tree actions one can take their product and use the diagonal action to obtain a new object that can fail to be cocompact. Guirardel's theorem below gives a way to find a cocompact subset in this situation. It's worth noting that the problem of finding a compact subset that carries the fundamental group of a space isn't always solvable. For a graph with a finitely generated fundamental group  one can find a compact core but Wise was able to find a 2-complex in \cite{wisecompactcore} where no compact subcomplex carried the fundamental group.

\begin{thm}
    [Guirardel Core Theorem]
    \label{thm:guirardelcore}
    Let \(T_1\), \(T_2\) be two minimal actions of \(G\) on \(\mathbb{R}\)-trees having non-homothetic length functions, or being irreducible (see \ref{thm:classification}). Assume that \(T_1\) and \(T_2\) are not the refinement of a common simplicial non-trivial action. Then there exists a subset \(\mathscr{C}\subseteq T_1\times T_2\) which is the smallest non-empty closed invariant connected subset of \(T_1\times T_2\) having convex fibers. Moreover, \(\mathscr{C}\) is CAT(0) for the induced path-metric, and \(T_1\times T_2\) equivariantly deformation retracts to \(\mathscr{C}\). We call \(\mathscr{C}\) the core of \(T_1\times T_2\).
\end{thm}

\begin{rmk}
    \label{guirardelcocompact}
    As Guirardel explains after Theorem 8.1 in \cite{guirardelcorepaper}, if \(T_1\) and \(T_2\) are simplicial trees then the core from \ref{thm:guirardelcore} is cocompact.
\end{rmk}

\begin{dfn}
    [Refinement]
    For \(G\)-trees \(T_1\) and \(T_2\) we say that \(T_1\) \emph{is a refinement of} \(T_2\) if there is an equivariant collapse map from \(T_1\) to \(T_2\).
\end{dfn}

\begin{lem}
    [Not refinements of a common tree]
    \label{lem:nocommonrefinement} 
    Let \(X\) and \(Y\) be two locally finite \(G\)-trees that lie in different deformation spaces. Then $X$ and $Y$ are not refinements of a common non-trivial simplicial $G$-tree.
    \begin{proof}
        Suppose the two trees were refinements of a common non-trivial simplicial $G$-tree. This would mean there is an edge in the common tree that has an edge above it in both trees. Let \(K\) be the stabilizer of this edge. It appears in all three trees. Since $X$ and $Y$ are locally finite, the vertex groups of $X$ are commensurable to each other, similarly for $Y$. But the property of fixing a point is invariant under commensurability. Therefore, all vertex groups of the first tree are elliptic in the second tree and vice versa. Hence, both actions have the same elliptic subgroups which means they are in the same deformation space which contradicts our initial assumptions.
    \end{proof}
\end{lem}

% \begin{lem}
%     [Axes are hausdorff equivalent]
%     \label{lem:axeshausdorff}
%     (??? no longer needed in proof of transverse lemma)
%     Any two axes for a given hyperbolic isometry of a geodesic metric space are Hausdorff equivalent.
% \begin{proof}

% \begin{figure}[htp]
%     \centering
%     \includegraphics[width=4cm]{axis-hausdorff.jpg}
%     \caption{To show the axes are close, we translate two closest points until they are near an arbitrary point and then apply the triangle inequality}
%     \label{fig:axis-hausdorff}
% \end{figure}

%     Let $X$ be a geodesic metric space and $g$ a hyperbolic isometry with translation length $a$. Suppose $\ell_1$ and $\ell_2$ are axes for $g$. Let $D:= d(\ell_1,\ell_2)$ denote the distance between the closed sets $\ell_1$ and $\ell_2$. Suppose that $x_0\in\ell_1$ and $y_0\in\ell_2$ realize that distance. Let $x\in\ell_1$ be arbitrary. Because the translation length of $g$ is $a$ there is some integer $m$ such that $g^mx_0$ is within $a$ of $x$. The isometry preserves distances so $g^mx_0$ and $g^my_0$ are $D$ far apart. Hence, $$d(x, g^my_0)\leq d(x, g^mx_0) + d(g^mx_0, g^my_0)\leq a+D$$ so $\ell_1$ is within the $a+D$ neighborhood of $\ell_2$. By symmetry, the axes $\ell_1$ and $\ell_2$ are Hausdorff equivalent.
% \end{proof}

% \end{lem}

\begin{lem}
    [Bass finitely generated conditions]
    \label{lem:bassfgcon}
    Consider a group \(G\) acting on a tree \(X\)
    \begin{enumerate}
        \item If \(X/G\) is finite and \(G_x\) is finitely generated for all \(x\in X\) then \(G\) is finitely generated
        \item If \(G\) is finitely generated and acts minimally on \(X\) then \(X/G\) is finite
    \end{enumerate}
    \begin{proof}
        For details see Proposition 7.9 in \cite{hymanbass}.
    \end{proof}
\end{lem}

The following should be thought of as a prequel to the main theorem \ref{thm:martino}. In one of the implications below we take two trees, impose a condition on its vertex stabilizers, apply the Guirardel Core machinery \ref{thm:guirardelcore}, and finally produce a compact 2-dimensional complex with a prescribed fundamental group. In the proof of the sequel, we will take three trees, impose a similar condition on vertex stabilizers, carefully arrange that a key lemma of Guirardel \ref{thm:guirardelsliceconvex} applies, and finally produce a compact 3-dimensional complex.





\begin{thm}
    [Transverse Construction]
    \label{pro:transverseconstruction} 
 Let \(G\) be a group of cohomological dimension 2. If \(X\) and \(Y\) are non-trivial minimal \(G\)-trees of finite type that are in different deformation spaces then the following are equivalent:
\begin{enumerate}
    \item $X$ and $Y$ are transverse;
    \item $G_x\cap G_y = \{1\}$ for all vertices \(x \in V(X), y\in V(Y)\);
    \item There exists a compact VH-complex $K$ with $\pi_1(K) \cong G$ whose horizontal and vertical splittings are $X$ and $Y$.
\end{enumerate}
\begin{proof}

    $1\Rightarrow 2$: Fix $x_0 \in V(X)$. Let $y\in V(Y)$. Then $G_{x_0} \cap G_y = (G_{x_0})_y$. By (1) $X$ is transverse to $Y$ hence $G_{x_0}\cap G_y$ is FP. Since the choice of $y\in V(Y)$ was arbitrary, the vertex groups of the $G_{x_0}$ action on $Y$ are FP. Note, $Y$ locally finite implies its edge groups are finite index subgroups of its vertex groups. Hence the edge groups are also FP. 
    
    We claim that the action of $G_{x_0}$ on $Y$ is non-trivial. We will apply Bieri \ref{pro:bireridimension} twice. The third equality below comes from applying Bieri to the action of \(G_{x_0}\) on \(Y\) restricted to the minimal subtree. To know this action is cocompact we use \ref{lem:bassfgcon}.
    \begin{align*}
        2 &= dG\\
          &= dG_{x_0}+1\\
          &= d( G_{x_0} )_y+1+1\\
          &= d(G_{x_0}\cap G_y)+2
    \end{align*}
    The equation shows that $d(G_{x_0}\cap G_y)=0$ so $G_{x_0}\cap G_y$ is trivial.
    
    \begin{claim*}
    The action of $G_{x_0}$ on $Y$ is non-trivial.
    \begin{proof}
        Suppose the action were trivial. That is, there exists some $y\in V(Y)$ such that $(G_{x_0})_y=G_{x_0}$. Hence, $G_{x_0}$ is elliptic for the action of $G$ on $Y$. By the local finiteness of $Y$, for all $x\in V(X)$, $G_x$ acts elliptically on $Y$. Hence, $\mathcal{E}(X)\subset \mathcal{E}(Y)$. Again by local finiteness we can promote this using \ref{thm:ellipticimpliesequality} to $\mathcal{E}(X) = \mathcal{E}(Y)$ which by Theorem \ref{thm:forester} gives $X \sim Y$  contradicting the fact that $X$ and $Y$ were assumed to be in different deformation spaces.
    \end{proof}
    \end{claim*}
$2\Rightarrow 1$: Trivial groups are FP.

$2\Rightarrow 3$: Take $X \times Y$ and give it the VH-structure where $X$ and $Y$ correspond to horizontal and vertical edges respectively. We first check a few conditions.
    
    First note that $X$ and $Y$ are minimal $G$-trees by assumption.
    
    If our trees had homothetic length functions (i.e. the length functions were a constant multiple of each other) then they would vanish on the same elements, which would imply they had the same elliptic elements. Our trees are cocompact with FP vertex groups, and FP implies finitely generated so we can apply \ref{thm:forester} of \cite{foresterdeformationrigidity} which says in this case having the same elliptic elements is enough to conclude that the elliptic subgroups are also the same. Hence, both trees lie in the same deformation space, a contradiction. Therefore, \(X\) and \(Y\) have non-homothetic length functions. By \ref{lem:nocommonrefinement} our trees are not refinements of a common non-trivial simplicial $G$-tree.
    
    Apply the Guirardel Core Theorem \ref{thm:guirardelcore} to obtain $C$ a certain subset of $X \times Y$ that we call the core. The core \(C\) has convex fibers. It's also CAT(0). Our \(G\)-trees are simplicial so \(C\) is a subcomplex. Condition (2) says that $G$ acts freely on the vertices of the product $X\times Y$ and therefore also on the the vertices of the core, a subset of $X \times Y$. Since the \(G\)-trees are simplicial and the product action is free on vertices, and the VH property rules out rotating a square by 90 degrees, and the product action prevents rotations by 180 degrees (a rotation by 180 degrees would invert an edge in the projection) we get that the product action on the cell complex \(C\) will be a covering space action.
    
    We also need that $C/G$ is VH. Is it enough to observe that the product action respects the tree factors. The edge partition on the cover \(C\) descends to a well-defined edge partition on the quotient and attaching maps constructed in the standard way for the quotient alternate between vertical and horizontal edges as needed.
    
    From Guirardel, \(C\) is CAT(0) and therefore NPC. We also get compactness from \ref{guirardelcocompact}. However, NPC is a local condition and under a covering map it descends to \(C/G\). Following Wise in \cite{wisethesis} the VH-complex \(C/G\) has a decomposition into vertex and edge spaces. The NPC condition ensures the attaching maps are \(\pi_1\)-injective, hence \(C/G\) is a graph of spaces with horizontal and vertical splittings.
    
    Because $C\subseteq X\times Y$ the leaves of $C$ coming from the vertical foliation as a square complex are equal to the connected components of the fibers from projecting $C$ to the $X$ coordinate, however by Guirardel the latter are connected. Collapsing leaves then is the same as collapsing connected fibers which gives projection to the $X$ factor. On the other hand, as \(C\) is a cover of \(C/G\) we get that \(C\) is also a graph of spaces. In this case, the edge and vertex spaces of \(C\) correspond to certain fibers from the foliation. Therefore, the Bass-Serre map from \(C\) given by collapsing vertex spaces to a point and mapping edge spaces to edges is the restriction of the projection map. With the product action projection is \(G\)-invariant and \(X\) is minimal so the image of the Bass-Serre map is all of \(X\). Hence, the \(G\)-tree \(X\) matches the Bass-Serre tree for the horizontal splitting of \(C/G\) where we collapse the vertical fibers. In a similar way, \(Y\) is the Bass-Serre tree of the vertical splitting of \(C/G\).
    
    
\item $3\Rightarrow 2$: 

    We have \(\widetilde K\) and its vertical and horizontal foliations. Collapsing
leaves to points gives \(G\)-maps \(\widetilde K\) to \(X\) and \(\widetilde K\) to \(Y\). These are the
Bass-Serre maps. The product map gives a map from \(\widetilde K\) to \(X\times Y\).

\begin{claim*} This map is 1-1.
\begin{proof}

It's enough to be 1-1 on vertices. Suppose \(v\), \(v'\) go to the same
vertex of \(X\times Y\). This means \(v\), \(v'\) are on the same leaf of the vertical foliation, and are also on the same leaf of the horizontal foliation.
Call these leaves \(V_v\) and \(H_v\).
By Lemma 3.7 of Wise in \cite{wisecsc}, the intersection of \(V_v\) and \(H_v\) is at most
one point. (Remark: the lemma says exactly one point, for \(\widetilde K\) a Complete Square Complex (CSC).
But for "at most one point", it only needs that \(\widetilde K\) is VH and CAT(0),
not complete.) Hence \(v = v'\).
\end{proof}
\end{claim*}

Both \(\widetilde K\) and \(X\times Y\) are CAT(0) square complexes, and we regard \(\widetilde K\) as a
subcomplex that is invariant under the product \(G\)-action on \(X\times Y\). The
action on \(\widetilde K\) is free because it is a covering space action.

By Cor 5.2 of Haglund in \cite{haglundss} if \(G\) acts on a CAT(0) cube complex \(X\)
without inversions then every element either has a fixed vertex or a
combinatorial axis, and not both.

Now take any non-trivial \(g \in G\). Acting on \(\widetilde K\), it has a combinatorial
axis \(L\), because the action is free. But \(L\) is also a combinatorial axis
in \(X\times Y\). By the lemma, \(g\) acting on \(X\times Y\) cannot fix a vertex.
So, no non-trivial element fixes a vertex of \(X\times Y\). Hence \(G_x \cap G_y =
1\) for all vertices \(x \in X\), \(y \in Y\).
    
\end{proof}
\end{thm}


\chapter{Notions of convexity in products of trees}

% - intro qc
% - qc => sc
% - qc <=> connected fibers

% - connected slices
% - def filling, coning
% - switching

% - cocompactness

This section develops the definitions for fibers, filling, and quadrant convexity along with the tools needed to apply them during the construction of the extended core.  

\begin{dfn}
    [Quadrant Convex]
    A subset \(S \subset T_1\times T_2\) is \emph{quadrant convex} if the complement is the union of quadrants. The \emph{quadrant hull} of a set \(S\) is the smallest quadrant convex subset containing \(S\) i.e. it's the intersection of all quadrant convex subsets containing \(S\). Equivalently, it's the complement of the union of all quadrants disjoint from \(S\). 
\end{dfn}


The following proposition is the first important application of quadrant convexity. Guirardel uses it to prove that his core is simply connected and later we will use it to show that the hyperplanes in our extended core are simply connected as a step towards showing the entire space is also simply connected.

\begin{prop}
    [Quadrant Convex implies Deformation Retraction]
    Let \(T_1\) and \(T_2\) be simplicial trees, and \(S\) a connected quadrant convex subset of \(T_1\times T_2\), then there is a deformation retraction from \(p_1(S) \times p_2(S)\) onto S. 
    
    \begin{proof}
        This follows from the proof of proposition 4.17 from Guirardel \cite{guirardelcorepaper}.
    \end{proof}
\end{prop}

\begin{rmk}
    Later we will find subsets \(S\) where \(p_i(S)\) is a tree for \(i\in\{1,2\}\) and we will conclude \(S\) is simply connected.
\end{rmk}

\begin{dfn}
    [Fibers in trees]
    \label{dfn:treefibers}
    A \emph{fiber} of a product of trees is the inverse image of a point under a projection map; e.g. \( \pi_1^{-1}(x) = \{x\} \times Y\) is a fiber of \(X\times Y\). A \emph{one-dimensional fiber} of a product of three trees is the inverse image of a point under a map \(\pi_{jk}: T_1\times T_2\times T_3 \to T_j\times T_k\) given by \(p = (p_1,p_2,p_3) \mapsto (p_j, p_k)\). We use similar terminology for subsets of tree products, that is a one-dimensional fiber of a subset of a product of trees is the intersection of that subset with a one-dimensional fiber of the product.
\end{dfn}

Next, we give an equivalence between connected fibers and quadrant convexity. The forward direction gives the second important application of quadrant convexity. During the construction of the extended core we will show a set has connected fibers, is quadrant convex, and simply connected with each proving the next in turn.

\begin{lem}
	[Guirardel Lemma 5.4, Corollary 5.5]
    \label{lem:guirardel} 
	Let \(T_{1} , T_{2}\) be two \(\mathbb{R}\)-trees and let \(F\) be a nonempty connected subset of \(T_{1} \times T_{2}\) with convex fibers. Then the complement of \(\overline{F}\) is a union of quadrants. That is, \(\overline{F}\) is also nonempty, connected, and has convex fibers.
\end{lem}

\begin{thm}
    [Equivalent Quadrant Convex Condition]
    \label{thm:guirardelsliceconvex}
    Let \(X\) and \(Y\) be two simplicial trees and \(F\subset X\times Y\) a closed subset. Then \(F\) is quadrant convex if and only if all of its fibers are connected.
    \begin{proof}
        Follows immediately from \ref{lem:guirardel} and the fact that  quadrant convex implies connected fibers.
    \end{proof}
\end{thm}

Keeping the above theorem in mind, we generalize the notion of quadrant convexity to three dimensions.
\begin{dfn}
    [Orthant Convexity]
    A subset \(S\subseteq T_1\times T_2\times T_3\) is \emph{orthant convex} if \(S\) has convex one-dimensional fibers.
\end{dfn}


% DO WE STILL NEED THESE?
% \begin{dfn}
%     [Orthant Convex]
%     A subset \(S\) of a product of three trees is \emph{orthant convex} if its complement is a union of orthants.
% \end{dfn}

% \begin{dfn}
%     [Orthant Slice Convex]
%     A subset \(S\) of a product of three trees is \emph{orthant slice convex} if its fibers are quadrant convex.
% \end{dfn}

\section{Building the orthant hull}

In this section we work by analogy with Guirardel's results on connected fibers and quadrant hulls. To create a quadrant hull one can either work subtractively by taking away quadrants or additively by connecting fibers with new material. We will work additively one dimension up and build  an object by filling in fibers.


\begin{prop}
    [Slices are connected]
 \label{prop:fibershomeoplanes} 
    Put \(f =  f_{1} \times f_{2} \times f_{3}: X \to T_{1} \times T_{2} \times T_{3} \) and \(J = \text{Im}(f)\). Then \(J \cap T_{1} \times T_{2} \times \{z\} = \text{Im}_{f} ({ f_{3}}^{-1}(z))\). 
    

    \begin{proof}
            Let \(p = (p_{1}, p_{2} , p_{3}) \in T_{1} \times T_{2} \times T_{3}\). Then we have the following, 
    \begin{align*}
        p \in \text { LHS } & \Longleftrightarrow p \in \text{Im}(f) \wedge p_{3} = z \\ 
        & \Longleftrightarrow \exists x \in X (f(x)=p \wedge  f_3(x)=z)\\
        & \Longleftrightarrow \exists x \in X (f(x)=p \wedge x \in f_3^{-1} (z))\\
        & \Longleftrightarrow p \in \text{Im}_f(f_3^{-1} (z)).
    \end{align*}
    \end{proof}
\end{prop}

The following definition serves only to give a name to an object created by Guirardel during the proof of the extension lemma \ref{lem:guirardel-extension}.
\begin{dfn}
    [Subgraph Coning]
    \label{dfn:subgraphconing}
    Let \(X\) be a 2-complex with a \(G\)-action. Let \(T\) be a simplicial \(G\)-tree. Let \(f:X\to T\) be a map. Let \(K\) be a subgraph of the 1-skeleton of \(X\). Then the \emph{subgraph cone of \(f\) with \(K\)} denoted \(\Lambda(f, K)\) is the set \[X \sqcup \left(G\times C_K\right)\] modulo the relation sending \((g, (x,0))\) to \(g\cdot x\) where \(C_K\) is the set \[ K\times [0,1]\] modulo the relation that glues \((x,1)\) to \((x', 1)\) if and only if \(f(x) = x'\). 
\end{dfn}

\begin{dfn}
	[Filling]
	\label{defn:filling}
    Given \(S \subseteq X := X_1\times X_2\times X_3\) define \(S_{k}\) for \(k \in \{1,2,3\}\) via: 
    \[
        p \in S_{k} \iff \exists 
        \,q,r\in S\, \forall j\in \{1,2,3\}: j\neq k 
        \Longrightarrow ( p_{j} = q_{j} = r_{j}
        \text{ and } p_{k} \in \text{cvxhull}_k (\{q_{k} , r_{k}\}))
    \] 
    where \(\text{cvxhull}_k\) is the convex hull operation on subsets of \(X_k\).
\end{dfn}

For us, when we apply the above definition, the factor spaces will be trees and therefore the filling operation that produces, say \(S_3\) from \(S \subset X_1\times X_2\times X_3\) is ensuring the one-dimensional fibers of \(\pi_{12}: S_3\to X_1\times X_2\) are connected. In the case of a tree, fibers are connected by drawing unique geodesics between points and connectedness and convexity are equivalent properties.



\begin{dfn}
[Connected in coordinate planes (CCP)]
    \label{defn:ccp} 
    Let \(S \subseteq X_1\times X_2\times X_3\). Then \(S\) is \emph{connected in all coordinate planes (CCP)} if \(S \cap \pi_{k}^{-1} (p)\) path is connected for all \(p \in X_{k}\) for all \(k\in \{1,2,3\}\).
\end{dfn}




\begin{lem}
    [Coning Connected Fibers]
    \label{lem:confib} 
     Let $f:K\to T$ be a map from a 2-complex $K$ to a simplicial tree $T$ with connected fibers. Let $\Gamma$ be a connected subgraph of $K^{(1)}$, the 1-skeleton of $K$. Pick $t_0\in f(\Gamma)$. Define $F:\Gamma\times I\to T$ by,
    \[ F(x,s) = 
            \gamma_{f(x),t_{0}} (s ) 
    \]
    where $\gamma_{x,y}$ is the embedded path between $x,y\in T$ with domain $[0,1]$ of constant speed or a constant map if $x=y$. Then the map $F$ is continuous and points have connected preimages.
    \begin{proof}
        Let $(x,s)$ be an arbitrary point in $\Gamma \times I$. Put $t:=F(x,s)$. We will show there is a path in $\Gamma\times I$ from $(x,s)$ to some $(x',0)\in \Gamma\times I$ that stays inside of the point preimage $F^{-1}(t)$. This is enough because $f$ already has connected preimages in $K$ and $F(x,0)=f(x)$.
        
        If $s=0$ then the constant path at $(x,s)$ suffices. 
        
        If $s=1$ then \(t=F(x,s)=F(x,1)=\gamma_{f(x),t_0}(1)=t_0\). We can take a path that runs along the top of $\Gamma\times I$ and then goes down using a vertical fiber. Since $t_0\in f(\Gamma)$ there exists some $z\in \Gamma$ such that $F(z,0)=t_0=t$. Take $g$ to be the concatenation of a path from $(x,1)$ to $(z,1)$ contained in $\Gamma\times \{1\}$ and the path $(z,1)$ to $(z,0)$ given by $t\to (z,1-t)$. When restricted to a vertical fiber, $F$ is either injective or a constant map. Since $F(z,0)=F(z,1)=t_0=t$ we have the image of our path lies in $F^{-1}(t)$ as desired.
        
        Suppose $0 < s < 1$. Consider the image of $x\times I$, a fiber of the product $\Gamma \times I$, under $F$. Denote the map $\gamma_{f(x),t_0}$ by $\gamma$. From the definition of $\gamma$, the fiber $x\times I$ is either sent to a single point or goes injectively to $T$. In the former, we take $g$ to be the straight path from $(x,s)$ to $(x,0)$.
        
        Finally, suppose the fiber $x\times I$ goes injectively to $T$. In this case, $F(x\times I) = \Imm(\gamma)$, recall $\gamma$ is a reparameterized geodesic between two points in $\Imm(f)$ which by the continuity of $f$ is both connected and convex in $T$. Hence, $t \in \Imm(f)$ by convexity. This means $f^{-1}(t)$ is a non-empty closed subset of $K$. Let $x'\in f^{-1}(t)$ be a point that minimizes $d_K(x,x')$. The distance cannot be zero because our fiber goes injectively. Hence, the distance is positive, so $x\neq x'$. 
        
        Take $\eta$ to be an geodesic path in $K$ from $x$ to $x'$. The image of $\eta$ is disjoint from $f^{-1}(t)$.
        
        Hence for $r \in [0,1]$, the geodesic from $f\eta(r)$ to $t_0$ contains $t$. In fact, $d_T(f(k), t)+d_T(t, t_0) = d_T(f(k), t_0)$ for $k\in\Imm(\eta)$. Define a function $g:\Imm(\eta)\to I$ by $g(k)=d_T(f(k), t)/d_T(f(k), t_0)$. The function $g$ is continuous and so the graph of $g$ is connected, contained in $F^{-1}(t)$, and contains $(x,s)$ and $(x',0)$. That is, the path $\eta(r)\times g(\eta(r))$ connects our preimage point to a connected fiber below.
    \end{proof}
\end{lem}


\begin{lem}
    [Reduction to vertical subpath]
    \label{lem:verticalsubpath} 
    Suppose \(S \subseteq T_{1} \times T_{2} \times T_{3} \) is a subcomplex with CCP. Let \(p,q,r \in S\) satisfy
    \begin{enumerate}
        \item \(r \not\in S\) 
        \item \(p,q \in S\) 
        \item \(p_{2} = q_{2} = r_{2}\) and \(p_{3} = q_{3} =r_{3}\)
        \item \(r_1 \in \text{cvx}_{T_{1}} (\{p_1,q_1\}) \) 
    \end{enumerate}
    then there is a path \(\sigma: [0,1] \to S\) between \(p\) and \(q\) such that \(\sigma(t)\) is contained in \(S \cap (T_{1} \times \{r_{2}\} \times \delta)\) where \(\delta\) is an open direction in \(T_{3}\) at \(r_{3}\) provided \(t \neq 0,1\).

    \begin{proof}
        Property CCP implies that \(S \cap (T_{1} \times \{r_{2}\} \times T_{3} )\) is path connected. Let \(\sigma\) be a path in that set from \(p\) to \(q\). Consider the pre-image of \(T_{1} \times r_{2} \times r_{3}\) by \(\sigma\), call it \(K\). Note that the complement of \(K\) is a countable disjoint union of open intervals in \([0,1]\) -- we will choose one later.
\begin{figure}[htp]
    \centering
    \includegraphics[width=10cm]{planar_version.png}
    \caption{Possible configuration when $T_1$ and $T_3$ are both $\mathbb{R}$. Note, interval length on the LHS may not correspond to path length on the RHS. The curve may not even be rectifiable. In general, the set on the LHS may not be discrete as drawn; it may include limit points or closed intervals.}
    \label{fig:pathconfig}
\end{figure}
        Each open interval is connected so considering projection and the fact that \(r_{3}\) is separating in \(T_{3}\) we have that under \(\sigma\) each open interval is mapped so the third coordinate lies in a single direction of \(T_{3}\) at \(r_{3}\). After identifying, \(K\) maps into \(T_{1}\). Color the points of \(K\) by which direction at \(r_1\) in \(T_{1}\) they map into. (See figure \ref{fig:pathconfig} and \ref{fig:generaltree}.) Here we use the fact that \(\sigma\) is a path that is disjoint from \(r\). In fact, because \(S\) is a subcomplex it is closed and so there is an open neighborhood of \(r\) that is disjoint from \(S\) and therefore also  \(\sigma\). (As in figure \ref{fig:sigmaavoids}.) Intersecting this neighborhood with \(T_{1} \times r_{2} \times r_{3}\) gives an open neighborhood in \(T_{1}\) that is disjoint from \(\sigma\). 
\begin{figure}[!h]
    \centering
    \includegraphics[width=10cm]{path_away_r.png}
    \caption{The path $\sigma$ avoids a neighborhood of $r$.}
    \label{fig:sigmaavoids}
\end{figure}
\begin{figure}[!h]
    \centering
    \includegraphics[width=6cm]{tree_version.png}
    \caption{Sketch of a possible configuration. This generalizes the diagram from figure \ref{fig:pathconfig}.}
    \label{fig:generaltree}
\end{figure}
        The upside is that each monocolored subset of \(K\) is closed by looking at the image of \(\sigma\) in the slice and taking intersections with a closed halfspace pointing away from \(r_{1}\). Take the smallest pairwise distance between the finite number of colored closed sets. This distance is non-zero because \(\sigma\) is disjoint from a neighborhood of $r_1$ in $T_1$. Consider two points in \(K\) that achieve that distance. There cannot be any points of \(K\) between them because we chose the smallest distance. This picks out an interval with endpoints that map to different directions as needed.
     \end{proof}
\end{lem}


\begin{lem}
    [Switching]
    \label{lem:switching} 
    Let \(S \subseteq T_1\times T_2\times T_3\) be a subcomplex that is connected in all coordinate planes \ref{defn:ccp}. Then \(S_{x}, S_{y},\) and \(S_{z}\) are as well. 
    \begin{proof}
    
    
    \begin{figure}[!h]
    \centering
    \includegraphics[width=6cm]{switching.png}
    \caption{The goal is to show the vertical green interval contained in the $xy$-plane between $r$ and $r'$ lying on $\sigma$ is in $S_x$. This connects points in $yz$-planes to $S$.}
    \label{fig:switchingsketch}
\end{figure}
    
        Without loss of generality, consider \(S_{ x}\), note that \(S_{x}\) will be connected in all \(xy\) and \(xz\) planes because \(S\) was. Consider the \(yz\)-planes in \(S_{x}\), if there were no new points added then the planes are connected and we are done. Suppose that \( r \in (S_{x} \smallsetminus S )\), we need to connect \(r\) to a point in \(S\). We will show that there is a path in \(S_{x} \cap \pi_{2}^{-1} (r_{2} )\) between \(r\not\in S\) and some point \(r'\in S\) as in \ref{fig:switchingsketch}.
       
        Since \(r\) is in \(S_x \smallsetminus S\) there exist distinct points \(p\) and \(q\) in \(S\) that agree in all coordinates except the first where we have that \(r_{1} \in \text{cvxhull}_{T_{1}}  (\{p_{1} , q_{1}\})\). Now, because \(S\) is connected in all coordinate planes there is a path \(\sigma\) from \(p\) to \(q\) that lies in \(S \cap \pi_{2}^{-1} (r_{2})\). In fact, we can take \(\sigma\) to be a path that begins at \(p\) and ends at \(q\) with \(T_{3}\) coordinates lying in exactly one closed direction of \(T_{3}\) at \(r_{3}\). We have factored out this situation into claim \ref{lem:verticalsubpath}.  

        Take \(\sigma\) as in the claim \ref{lem:verticalsubpath}. Consider \(D = r_1 \times r_{2} \times \overline \delta\) a closed set. Let \(A\) be the set of points where \(\sigma\) crosses \(D\). Note, $A$ does not include the root of $\delta$. This is a closed set so we can consider the preimage under \(\sigma\) and look at the complement in \([0,1]\). Color each interval by the direction in \(T_{1}\) at \(r_{1}\) that \(\pi_{1} \sigma\) takes it to. Now, identifying \(D\) with a closed direction at \(r_3\) in \(T_3\) we say that the colors at \(x \in D\) are the set of colors of intervals that the preimage of \(x\) under \(\pi_{3} \sigma\) hits. (See figure \ref{fig:possibleconfig}.)

\begin{figure}[!h]
    \centering
    \includegraphics[width=6cm]{interval_coloring.png}
    \caption{A possible configuration}
    \label{fig:possibleconfig}
\end{figure}
\begin{figure}[!h]
    \centering
    \includegraphics[width=6cm]{separation.png}
    \caption{A finite number of monocolored points separating the root from $A$.}
    \label{fig:separation}
\end{figure}
\begin{figure}[!h]
    \centering
    \includegraphics[width=6cm]{multicolored.png}
    \caption{Continuing from \ref{fig:separation}.  Note, the left branch is not multicolored. A particular multicolored path is highlighted.}
    \label{fig:multicolored}
\end{figure}


        Our goal is to find a geodesic from the root of \(D\) to a point in \(A\) that is multicolored. (See figures \ref{fig:separation} and \ref{fig:multicolored}.) Note, because points in trees are separating, if the image of a continuous map contains two points it also contains the geodesic between those points. Suppose the claim were false. Then between every point of \(A\) and the root there is a point that is monocolored. A finite number of these points suffices to separate all of \(A\) from the root. Consider the first interval, it must cross one of these points. There must be another interval that eventually leaves that half space and enters a halfspace not entered yet. Because the geodesics have one color, and we always cross a monocolored point upon leaving, each segment has the same color. Including the last segment that goes back to the root. But this is impossible because we assumed the path begins and ends in different directions in \(T_{1}\) at \(r_{1}\).

    \end{proof}
\end{lem}

\begin{pro}
\label{prop:directionunion}
Let $T_1$ and $T_2$ be trees with $x_2\in T_2$. Then $$T_1\times x_2 = \bigcup_{\delta \in D(x_1)} \overline{\delta}\times \{x_2\}.$$
\end{pro}


\begin{prop}
\label{prop:sliceunion}
\[ \left(T_1 \times T_2\right) \smallsetminus \left(T_1\times x_2\right) = \bigcup_{\delta\in D(x_2)} T_1\times \delta\]
\end{prop}


\begin{lem}
    [Slice Switching]
    \label{lem:sliceswitching}
    If \(R \subseteq T_{1} \times T_{2}\) is connected then \( \left( R_{x} \right)_{y} = \left( R_{y} \right)_{x}\) where $x$ corresponds to $T_1$ and $y$ corresponds to $T_2$ in the definition of filling \ref{defn:filling}.
    \begin{proof}
        We first show that $\rxy$ has connected 1-dimensional fibers. The set $\rxy$ has connected $x$-fibers, this follows from the definition of filling in the $y$-direction. That is, $\left( R_{x} \right)_{y} \cap \{x_0\} \times T_2$ is connected for all $\{x_0\}$ in $T_1$. It remains to show that \(\rxy\) has connected $y$-fibers.
        
        For sake of a contradiction, suppose \(\rxy\) had a disconnected $y$-fiber at $y_0$, denoted $F$. That is, there exists $y_0$ such that \(F = \left( R_{x} \right)_{y}\cap T_1 \times \{y_0\}\) is disconnected. Because we are working in a tree we can find a point \(x_0\in T_1\)  such that \((x_0,y_0)\)  separates \(F\). Using \ref{prop:directionunion} we proceed by cases on how many directions meet $R_x$. Specifically, subtract \((x_0,y_0)\) from both sides of \ref{prop:directionunion} to obtain a disjoint union of sets and ask how many intersect $F \cap R_x$.
        
        \begin{enumerate}
            \item zero: Suppose \(F\cap R_x \cap (\delta \times y_0)=\varnothing\) for all \(\delta\in D(x_0)\) then by \ref{prop:directionunion}, $R_x \cap F=\varnothing$. However, the fiber is still disconnected so must intersect some of $\rxy$. This means there exists $p\in (F\cap \rxy)\smallsetminus R_x$. That is, $p$ was obtained by filling in the vertical ($y$, $T_2$) direction. There exist $\eta_1, \eta_2\in D(y_0)$ such that $F\cap R_x \cap (p_1 \times \eta_k) \neq \varnothing$ for $k=1,2$. By \ref{prop:sliceunion} \(F\cap R_x \cap (p_1 \times \eta_1)\subset T_1\times \eta_1\) and \(F \cap R_x \cap (p_1 \times \eta_2)\subset T_1\times \eta_2\) are disjoint. Pick out points in each, since they are in $R_x$ they are either already in $R$ or were obtained by filling in the $x$-direction. Hence, $F\cap R_x \cap (T_1\times \eta_1)$ and $F\cap R_x \cap (T_1\times \eta_2)$ both intersect $R$. However, these are disjoint and $R$ is connected a contradiction.
            \item at least two: Suppose \(F\cap R_x\cap(\delta_k \times y_0)\neq\varnothing\) for $k=1,2$ with $\delta_k\in D(x_0)$. Then because $R_x$ has connected $y$-fibers we have that $(x_0,y_0)\in R_x\subset \rxy$ which is a contradiction since $(x_0,y_0)$ separates $F$.
            \item exactly one: Suppose \(F\cap R_x\cap(\delta_0 \times y_0)\neq\varnothing\) but is  $\varnothing$ for all $\delta_k\in D(x_0)$ where $k\neq 0$. (See figure \ref{fig:square} for a diagram of this case) Since $\delta_0\times y_0$ intersects $R_x$ it also intersects $R$. Consider the sets $(x_0\times \eta)$ for $\eta\in D(y_0)$, at most one intersects $R_x$. If more than one did then because $\rxy$ has connected $x$-fibers we would have $(x_0,y_0)\in\rxy$ a contradiction since $(x_0,y_0)$ separates $F$. Because $F$ is disconnected, there exists $\delta_1\in D(x_0)$ with $\delta_1\neq d_0$ and $(\delta_1\times y_0)\cap F\neq \varnothing$. Combined with our assumption, pick a point $p\in (\delta_1\times y_0)\cap F\smallsetminus R_x$. Now, $p\in\rxy\smallsetminus R_x$ so there exist directions $\eta_1,\eta_2\in D(y_0)$ such that $(p_1\times \eta_i)\cap R_x\neq\varnothing$ for $i=1,2$. From above, take $\eta_1$ to be one of the directions such that $(x_0\times \eta_1)\cap R_x =\varnothing$. Suppose $q\in (p_1\times \eta_1)\cap R_x$. If $q\in R$ take $x'=q_1$. If $q\not\in R$ then we use the fact that $q\in R_x\smallsetminus R$ to find a point in $R$. We can take $x'\in \omega$ where $\omega$ denotes a direction in $D(p_1)=D(q_1)$ that does not contain $\delta_0$ with the property that $(x',q_2)\in R$. Then $(x', q_2)\in \delta_1\times \eta_1$ and $(x',q_2)\in R$, however the boundary of the quadrant $(\delta_1\times y_0)\cup (x_0\times \eta_1)$ does not intersect $R_x$ and so separates $(x',q_2)\in R$ from a point in $R\cap (\delta_0\times y_0)$. This is a contradiction since $R$ is connected.
        \end{enumerate}
        Hence, the set $\rxy$ has connected 1-dimensional fibers. By Guirardel \ref{lem:guirardel} then the complement of $\rxy$ is a union of quadrants and so $\text{QH}(R)\subset \rxy\subset \text{QH}(\rxy)$. Next note that $\text{QH}((R_y)_x)=\text{QH}(R_y)=\text{QH}(R)=\text{QH}(R_x)=\text{QH}((R_x)_y)$. Therefore, $\rxy =\text{QH}(R)=\ryx$.
        
        
        
        \begin{figure}[htp]
    \centering
    \includegraphics[width=8cm]{slice_lemma_case3.png}
    \caption{Illustrating case 3 of Lemma \ref{lem:sliceswitching}}
    \label{fig:square}
\end{figure}
        
        
        
                  
    \end{proof}
\end{lem}


\section{Cocompactness of orthant hulls}

\begin{lem} 
[Filling preserves cocompactness]
\label{lem:fillingcocompact}
    Let $T_1,T_2,T_3$ be simplicial $G$-trees that are pairwise transverse and $S\subset T_1\times T_2\times T_3$ an invariant cocompact subcomplex. Then $S_x$ is also cocompact.
\begin{proof}
    Observe that $S\subseteq T_1\times B$ where $B=\pi_{23}(S)$. Under the product action $B$ is an invariant set. Projection is a continuous map so $B$ is cocompact. Next, consider a one-dimensional fiber of $S$ above a point, that is $S' = S\cap (T_1\times \{b\})$ where $b\in B$ is a vertex. Because $T_2$ and $T_3$ are transverse the stabilizer of $b$ is trivial. Hence, the quotient map on the one-dimensional fiber $S'$ is an embedding into a compact set $S/G$. Hence, $S'$ is also compact. For each vertex $b\in B$ consider the one-dimensional fiber $S\cap (T_1\times \{b\})$, the action in $T_1$ is by simplicial automorphisms so distances are preserved. Hence, every one-dimensional fiber in an orbit has the same finite diameter. Because $B$ is cocompact there are a finite number of vertex orbits and therefore a universal bound on the diameters of one-dimensional fibers above vertices. After identifying a one-dimensional fiber with $T_1$ we see that each vertex of $S_x$ is obtained by filling in the convex hull of some one-dimensional fiber; more precisely, if $v\in S_x$ then for some $b\in B$ a vertex, $v_1\in \text{cvx}(\pi_1(S\cap(T_1\times \{b\})))$ where $v_1$ is the first coordinate of $v$. In particular, $S_x$ is contained within a bounded neighborhood of $S$ with the product metric. Since $S$ was a subcomplex, so is $S_x$. Hence, $S_x$ is cocompact.
    % (We are relying on $S_x$ being a subcomplex so that we can prove that it is cocompact by saying there are a finite number of vertex orbits in $B$ and each element of an orbit has a universally bounded number of vetices) ((Could also argue that this si cocompact by saying in the product metric - really just one factor - it's a subcomplex contained in a bounded neibhorhood))
\end{proof}
\end{lem}

\chapter{Proof of main theorem}

Starting with the hypotheses from the main theorem, this section breaks the proof into pieces.

\section{Setup}

Suppose for sake of a contradiction that there were three non-trivial \(G\)-trees \(T_{1}\), \(T_{2}\), and \(T_{3}\) of finite type \ref{defn:finitetype} that are pairwise transverse \ref{defn:transverse} and no two are in the same deformation space. Transversality and finite quotients are preserved by deformations. Therefore, by \ref{reducedcocompact}, without loss of generality we may assume that these are minimal \(G\)-trees after performing a sequence of elementary collapses. 

\section{Construct square complex with two splittings and given fundamental group}

Applying the transverse construction lemma \ref{pro:transverseconstruction} we obtain \(X_{12}\). For now think of $X_{12}$ as a square complex that encodes the two $G$-trees $T_1$ and $T_2$ with fundamental group $G$.

\section{Construct CCP set}

So far we have only used \(T_1\) and \(T_2\) to define \(X_{12}\). We need to include our third action. 

\begin{lem}
    [Affine Equivariant Map]
    \label{lem:affineequivariantmap} 


    Suppose that \(G\) acts freely on a simplicial complex \(K\) and acts on a simplicial tree \(T\). Then there exists an equivariant map \(f: K \to T\) where the connected components of the fibers of \(f\) are the leaves of a measured foliation and \(f\) is an isometry on edges transverse to \(\mathscr{F}\).
    \begin{proof}
        Construct an equivariant map.

        We start by defining \(f\) on \(K^{(0)}\) the 0-skeleton. It is enough to define the map on a single vertex in each vertex orbit, and extend equivariantly. These choices can be arbitrary. Next we check that the resulting map is well-defined. 
        
        Indeed, if \(gv=hv\) then \(g^{-1} h = 1\) by freeness and so
        
        \[ f(gv) = gf(v) = g(g^{-1}h)f(v) = hf(v) = f(hv). \]
        
        Next we define the map on the 1-skeleton by mapping each edge. If \(vw\) is an edge, map it to the unique geodesic \([f(v), f(w)]\) in \(T\).
        
        Lastly, for 2-cells we use the standard fibration coming from mapping triangles to tripods. There are four cases based on where the 3 vertices land in \(T\). See figure \ref{fig:triangle_fibers}.
        
      
\begin{figure}[htp]
    \centering
    \includegraphics[width=16cm]{triangle_fibers.png}
    \caption{Possible fibrations for each simplex}
    \label{fig:triangle_fibers}
\end{figure}
        
        To obtain a transverse measure, take a small arc \(\alpha\) that is transverse to the leaves of the foliation. The length of \(\alpha\) is defined to be the length of \(f \circ \alpha\) in the tree. Hence, by construction \(f\) is an isometry on edges transverse to \(F\).
    \end{proof}
\end{lem}


Let \(\widetilde {X_{12}}\) denote the universal cover of \(X_{12}\).  Applying the affine equivariant map construction \ref{lem:affineequivariantmap} to the actions on \(\widetilde {X_{12}}\) and \(T_{3}\) gives an equivariant  map and a foliation that we denote by \(f_{123}\) and \(\mathscr{F}_{123}\) respectively.

        The content of the next lemma is that a map from a 2-complex that is constant on leaves can be extended to a map with connected leaves by enlarging the 2-complex.
        \begin{lem}
            \label{lem:guirardel-extension}
            (Guirardel's Extension Lemma)
            Consider a geometric action of a finitely generated group \(G\) on an \(\mathbb{R}\)-tree \(T\), and let \(X\) be a 2-complex endowed with a free properly discontinuous cocompact action of \(G\). Let \(\mathscr{F}\) be a \(G\)-invariant measured foliation on \(X\). Consider a map \(f: X \to T\) which is constant on leaves of \(\mathscr{F}\), and isometric in restriction to transverse edges of \(X\). Then there exists a 2-complex \( X'\) containing \(X\), endowed with a free properly discontinuous cocompact action of \(G\), a measured foliation \(\mathscr{F} '\) extending \(\mathscr{F}\), and which induces an isometry between \(X'/ \mathscr{F}'\) and \(T\). Moreover, the inclusion \(X \subseteq X'\) induces an epimorphism of fundamental groups.
        \end{lem}
        
        Next, we check that the extension lemma can be used within our setting.
\begin{lem}
[Technical assumptions for Guirardel]
    \label{lem:technicalconditions}
 The following properties hold for \(\widetilde{ X_{12} }\) and the map \(f_{123}: \widetilde{ X_{12}}\to T_3\) constructed via  \ref{lem:affineequivariantmap}. The action of \(G\) on \(T_3\) is geometric. The action of \(G\) on \(X_{12}\) is free, properly discontinuous, and cocompact. (Compare to \ref{lem:guirardel-extension})
    
    \begin{proof}
        Our trees are simplicial and the actions are minimal, hence by \ref{lem:simpgeo} they are also geometric. The action on \(X_{12}\) is a covering space action from \ref{pro:transverseconstruction} and is therefore free, properly discontinuous, and cocompact.
    \end{proof}
\end{lem}

        With \ref{lem:technicalconditions} we now have everything needed to apply the Guirardel Extension Lemma \ref{lem:guirardel-extension} to \(\widetilde{X_{12}}\), \(f_{123}\) and \(\mathscr{F}_{123}\) in order to obtain \(\widetilde{X_{12}}^{+}\) and \(f_{123}^{+}\) and \(\mathscr{F}_{123}^{+}\) respectively. Having connected the fibers for \(f_{123}\) we continue with  work on the other functions.

        Next, let  \(f_{121}\) and  \(f_{122}\)  denote the Bass-Serre \ref{defn:bsmap} maps from \(\widetilde {X_{12}}\) to \(T_{1}\) and \(T_{2}\) respectively.
        
                \begin{rmk}
            
        The following is an informal sketch of what the Bass-Serre maps \(f_{122}\) and \(f_{121}\) look like. In our case \(\widetilde {X_{12}}\) is a VH-complex so edge and vertex spaces are graphs. Each square has a vertical and horizontal foliation. When $x$ is a vertex or a point on a horizontal edge, call the subset we get from extending the vertical foliation $V_x$ (recall \ref{dfn:decompositiongraph}). Then collapsing all the $V_x$ spaces to a point gives the map to $T_1$ called \(f_{121}\) defined above as the Bass-Serre map. 
        \end{rmk}
        
        Our goal is to produce a map from the extended complex \(\widetilde {X_{12}}^+\) to \(T_1\times T_2\times T_3\). The map   \(f_{123}^+\) defined on \(\widetilde {X_{12}}^+\) is already determined by the extension lemma, so it remains to extend the Bass-Serre maps \(f_{121}\) and \(f_{122}\) from \(\widetilde {X_{12}}\) to \(\widetilde {X_{12}}^+\). 
        
        By construction, \(\widetilde {X_{12}}^+\) has the form \(\Lambda(f_{123}, K)\) from \ref{dfn:subgraphconing} for some choice of graph \(K\). 
        
                
        \begin{rmk}
            The following is an explanation of how the graph \(K\) from applying \ref{lem:guirardel-extension} in Guirardel's paper is used. In short, for \(G\) finitely presented Guirardel's proof picks a specific compact subgraph of the 1-skeleton \(K\) that is large enough to make \(\widetilde {X_{12}}^+\) simply connected. Strictly speaking, we do not need to know the nature of \(K\), only that the extension has some graph coning structure so we can extend the Bass-Serre maps.
        \end{rmk}
        
        
        Once we know that the extension has this graph coning structure we can extend the Bass-Serre maps to \(\widetilde {X_{12}}^+\) by applying \ref{lem:confib}. Denote the resulting extensions by \(f_{121}^{\wedge}\) and \(f_{122}^{\wedge}\) and foliations by  \(\mathscr{F}^{\wedge}_{121}\) and \(\mathscr{F}^{\wedge}_{122}\). 
        
        Form the product map \(f:= f_{121}^{\wedge} \times f_{122}^{\wedge} \times f_{123}^{+}\). Finally, define \(J := \text{Im}(f) \subset T_1\times T_2\times T_3\). Each map is equivariant so \(G\) acts on \(J\). Since \(J\) is the image of a cocompact set under a continuous \(G\)-map it is cocompact. 
        
        Finally, by \ref{prop:fibershomeoplanes} \(J\) has property CCP.
    

    
        
        
        
        
        
        
        
        
        
    \section{Take cellular neighborhood}
        
\begin{lem}
    [Cell Respects Slices]
    \label{lem:cellrespecslice} 
    If $A\subset T_1\times T_2\times T_3$ is CCP and \(A\) is path connected then $\text{cell}(A)$ is also CCP.

    \begin{proof}
        Let \(z\in T_3\). Pick \(p,q\in \text{cell}(A)\cap (T_1\times T_2\times \{z\})\). For a product of trees, the closure of a cell is a subcomplex. For all \(p\) in \(\text{cell}(A)\) there exists a cell \(P\) and a point \(p'\in A\) such that \(p' \in P\) and \(p\) is in the topological closure of \(P\). Similarly for \(q\), \(q'\), and \(Q\). Since \(p\) and \(p'\) are contained in a closed \(n\)-cube but \(p\) in \(P\), we have that \(d_{T_3}(p_3,p'_3)<1\). Throughout, remember that if \(p'_3\) is a vertex, then \(p'_3 = z\). Similarly for \(q'_3\). This ensures that later when points are pushed along an edge the result remains within \(\text{cell}(A)\).  Note, closed cubes project to closed cubes in \(T_3\). Since \(p3=q3=z\) the projections of \(\text{cl}(P)\) and \(\text{cl}(Q)\) are closed cubes in \(T_3\) both containing \(z\).
        
        Let $\eta'$ be a path in \(A\) from \(p'\) to \(q'\). The goal is to create a modified path \(\eta\) such that \(d_{T_3}(\pi_3\eta(t), z) <1\). Points in \(\pi_3\eta'\) that are not vertices of \(T_3\) but lie in \(A\) can be pushed to a new path \(\eta\) in \(\text{cell}(A)\cap(T_1\times T_2\times \{z\})\). 
        
        Using the CCP property of \(A\), for a given point \(r\in T_3\) we can note the first and last time a path enters the slice \(T_1\times T_2\times \{r\}\) and replace that (possibly degenerate) segment by a path contained in \(A\cap T_1\times T_2\times \{r\}\). Call this a path snip. (For the degenerate case, concatenate paths instead)
        
        Consider the following cases.
        \begin{enumerate}
            \item \(p'_3=q'_3\): Use the CCP property of \(A\) to draw a path \(\eta\) contained in the slice \(A\cap (T_1\times T_2\times \{z\})\).
            \item \(p'_3\neq q'_3\) and \(z=p'_3\) or \(z=q'_3\): Snip at \(p'_3\) and at \(q'_3\). This limits the path to one edge and prevents it from wandering when \(z\) is a vertex of \(T_3\).
            \item \(p'_3\neq q'_3\) and \(z\in [p'_3,q'_3]\smallsetminus\{p'_3,q'_3\}\): As before, begin by snipping at \(p'_3\) and \(q'_3\). This case includes the situation where \(z\) is a vertex of \(T_3\) lying between \(p'_3\) and \(q'_3\). This allows a path to wander arbitrarily far from \(z\) in \(T_3\). To prevent this, we also snip at \(z\).
        \end{enumerate}

If necessary, use the open cells containing \(p'\) and \(q'\), to draw paths  to \(p\) and \(q\). Require the interior of their domains to map to \(P\) and \(Q\) respectively so their \(T_3\) coordinates remain close to \(z\). Concatenate with \(\eta'\) to create a new path from \(p\) to \(q\). Denote this path by \(\tau\) and form \(t\to (\tau_1(t), \tau_2(t), z)\) a path from p to q contained in \(\text{cell}(A)\) as \(\tau\) was contained in \(A\) and remained close to \(z\).
    
    \end{proof}
    
    
\end{lem}

        
        
        Next, define \(K := \text{cell}(J)\). As \(J\) is \(G\) invariant so is \(K\).  Because our trees are locally finite and a cellular neighborhood is contained in a bounded neighborhood we have that \(K\) is cocompact. Taking a cellular neighborhood respects slices by \ref{lem:cellrespecslice}  so \(K\) has property CCP as well.

\section{Filling}


Filling is a way to ensure an object has connected 1-dimensional fibers; at least in the direction of the filling. Each time you fill, material is added that could in principle create disconnected fibers in the other two  directions. The worry is that attempting to refill will simply create more disconnected fibers so no amount of repeated fillings will be enough. We will apply the switching lemma \ref{lem:switching} in the context of the extended core construction so this does not happen and we will obtain an object with connected 1-dimensional fibers. 



We are now ready to complete the construction of the extended core. Let \(E:= ((K_x)_y)_z\) be the extended core. Repeatedly apply lemma \ref{lem:fillingcocompact} and use the fact that filling preserves CCP by construction to see that \(E\) is cocompact and has CCP. By  the switching lemma \ref{lem:switching} \(E\) has connected 1-dimensional fibers. 

\section{Topological properties of the Extended Core}

Here we verify that \(E\) is a universal cover and that \(E/G\) has three honest graph of spaces decompositions.

\begin{dfn}
    [VHD-Complex]
    \label{dfn:vhd}
    We say \(X\) is a VHD-complex if it is a 3 dimensional cube complex with simplicial links and two-sided hyperplanes where edges are labelled one of V, H, or D, parallel edges have the same label, and incident edges of square have different labels.
\end{dfn}

\begin{dfn}
    [Primitive]
    \label{dfn:primitive}
    A VHD complex is \emph{primitive} if its 1-skeleton contains no non-trivial circuit with edges of the same label. That is, for each label, the subgraph spanned by edges with that label is a forest.
\end{dfn}

To start, \(T_1\times T_2\times T_3\) has a natural VHD-complex structure given by labelling \(T_1\) edges with \(V\), \(T_2\) edges with \(H\) and \(T_3\) edges with \(D\). The product \(T_1\times T_2\times T_3\) contains \(E\) as a \(G\)-invariant subset. The action is a product action so orbits of edges remain in the same parallelism class, therefore the quotient inherits the VHD-complex labeling. The tree actions do not have any inversions and therefore \(E\) and the quotient \(E/G\) have two-sided hyperplanes. Hence, \(E\) and \(E/G\) are VHD-complexes.


\begin{rmk}
    Our definition of VHD-Complex does not require that hyperplanes are injective on fundamental groups, only that hyperplanes are two-sided. The injectivity property is  however needed to obtain an honest graph of groups decomposition. For the VHD-Complex \(E/G\) we are considering the injectivity property follows from fact that there is a universal cover \(E\) with simply-connected hyperplanes. For comparison, in dimension two, in a VH-Complex the \(\pi_1\)-injectivty of hyperplanes follows automatically from the two-sided requirement and the vertex link requirement.
\end{rmk}

This lemma and the next one ensures \(E/G\) will have fundamental group \(G\). 


\begin{lem}

    The action of \(G\) on \(E\) is a covering space action.
    
    \begin{proof}
        Consider the product action of \(G\) on \(T_1\times T_2\times T_3\). 
        The trees \(T_1, T_2,\) and \(T_3\) are mutually transverse which tells us that the action on the vertices of a product of any two trees is free; hence free on the vertices of the product of all three trees. Acting by simplicial automorphisms with finite vertex stabilizers (in our case trivial) gives a PD action. 
        
        For a cube of any dimension, if you fix a point on it's interior then that cube is taken to itself.
        
        Being a product action rules out rotations of cubes or squares and disallowing edge inversions rules out reflections.
        
        Hence, no non-trivial element takes a cube to itself. Thus, the action is free. This implies that the product action is a covering space action.
        
        Lastly, \(E\subseteq T_1\times T_2\times T_3\) is a \(G\)-invariant subset and therefore the action of \(G\) on \(E\) is also a covering space action.
    \end{proof}
\end{lem}

\begin{lem}
    [Extended Core is simply connected]
    \label{lem:coresc} 
    The Extended core $E$ is simply connected.
    
    \begin{proof}
    
        % - It seems like we are trying to show E/G is a graph of spaces by leveraging E and then concluding that E is a graph of spaces.
        % - Is it just better to prove directly that E is a graph of spaces?
        % -- By claim 1 we know what the hyperplanes of E are, by claim 2 we know they are separating and so the underlying graph is a tree. The conceit is that cube complexes with two-sided hyperplanes have corridors that allow for a decomposition into a graph of spaces.
        
        % - Later we still need to show that E/G is a graph of spaces so perhaps it's impossible to separate
    
        We will show that \(E\) is the total space of a graph of spaces where the edge and vertex spaces are simply connected and the underlying graph is a tree. The edge spaces are hyperplanes and the vertex spaces are the connected components of the complement of the corridors of the hyperplanes. Now, $E$ is a subset of a product so hyperplanes are embedded and two-sided and come with maps to each side of their corridor. 
        
        The hyperplanes of \(E\) are the connected components of the intersections of fibers (e.g. \(\pi_1^{-1}(p)\)) with \(E\).
        By Property CCP then, the hyperplanes are path connected. Since the fibers are separating, the hyperplanes are also separating. This means the underlying graph is a tree. Because $E$ has connected 1-dimensional fibers the hyperplanes also have connected 1-dimensional fibers and so are quadrant convex by Guirardel \ref{thm:guirardelsliceconvex} and so are also simply connected. Hence, $E$ is a tree of simply connected spaces and is therefore simply connected.
        
        %Our \(E\) inherits a VHD structure from the product of three trees that it sits in. 
        
        % Because the action is diagonal we also get that the quotient \(C/G\) is VHD. Our tree actions do not invert edges so hyperplanes of \(C/G\) are two-sided; indeed a given hyperplane only touches a single parallelism class of edges. 



        % The edge spaces are hyperplanes, the vertex spaces come from subtracting the corridors of hyperplanes, and the edge maps are maps from hyperplanes to vertex spaces defined by seeing where the push map homotopy takes a hyperplane. 
        
        % We need these maps to be injective on fundamental groups in $E/G$. The push map from a hyperplane to a vertex space followed by inclusion is the same up to homotopy as globally including the hyperplane into \(E/G\). We will show that the composition is injective so that the induced map from the edge space to the vertex space is injective on fundamental groups as needed. It is enough to show that lifts of hyperplanes are simply connected.
    
        % A hyperplane of $E/G$ lifts to a hyperplane of $E$ which sits inside a slice of $T_1 \times T_2\times T_3$. Since $E$ has connected 1-dimensional fibers by Guirardel the hyperplane is quadrant convex which implies simply connected.
        \end{proof}
        \end{lem}
        
        \begin{prop}
            [Injective on fundamental groups]
            \label{lem:injpi1}
            Let \(A\subseteq X\) and \(\rho:\widetilde X\to X\) the universal cover of \(X\). Then the inclusion map of \(A\) into \(X\) is \(\pi_1\)-injective if and only if the connected components of \(\rho^{-1}(A)\) are simply connected.
        \end{prop}
        
        The final topological property we need is for the hyperplanes of \(E/G\) to be \(\pi_1\)-injective. Fortunately, during the proof of \ref{lem:coresc} we show that the hyperplanes of \(E\) are simply connected. This combined with \ref{lem:injpi1} gives the injectivity property. As a corollary, \(E/G\) has three splittings  with Bass-Serre trees \(T_1\), \(T_2\), and \(T_3\).

    \end{proof}
\end{lem}


\section{Bieri dimension argument}

At this stage the construction of \(E/G\) is complete. It's a three dimensional complex that splits in three ways over 2-dimensional VH-complexes. By Wise, VH-complexes themselves come with two splittings over 1-dimensional graphs. Lastly, graphs are themselves splittings over trivial groups. Using a result of Bieri, we show this iterated splitting pushes the cohomological dimension of \(G\) higher which gives the final contradiction needed for the proof of the main theorem.

In order to use our assumptions we need to show the first iteration of splittings coming from the VHD complex \(E\) correspond exactly to the original trees.

Earlier we proved that \(E\) is a graph of spaces, essentially because it sits within a product of trees. With this setup, after identification, the Bass-Serre maps that collapse vertex spaces to points and fibers of edge space products to points are projection maps. We're acting via a product action so projection maps are equivariant maps which means the image of \(E\) under projection is invariant. Lastly, our \(G\)-trees are minimal so the Bass-Serre maps from \(E\) map onto the original trees.

Next, we restate corollaries of Bieri's results.
\begin{lem}
(See Corollary 6.5, Bieri, p87)
Let \(G=G_1\ast_H G_2\) be an amalgamated product of groups of type \(FP_\infty\) over \(R\), with \(H\) a proper subgroup of finite index in both factors. Then for \(k=1,2\) we have: \[\text{cd}_R(G)=\text{cd}_R(G_k)+1.\]
\end{lem}

\begin{lem}
(See Corollary 6.7, Bieri, p92)
Let \(G=G_1\ast_{H,\phi}\) be an HNN extension where \(G_1\) is of type \(FP_\infty\) over \(R\) with subgroups \(H\) and \(\phi(H)\) of finite index in \(G_1\). Then \[\text{cd}_R(G) = \text{cd}_R(G_1)+1.\]
\end{lem}

The following result is not proved as a direct consequence of the previous lemmas, instead Bieri is able to run the argument again to obtain a similar result for general graphs of groups. This is the version we will be applying.
\begin{thm}
    [Bieri dimension plus one]
    \label{pro:bireridimension}
    (Exercise p.93 from Bieri \cite{bieribook})
    For a non-trivial cocompact locally finite simplicial \(G\)-tree with \(FP_\infty\) vertex and edge groups we have that the dimension of \(G\) is exactly one more than the dimension of any vertex or edge group.
\end{thm}

The following lemma shows that the second step in the iterative splitting is non-trivial.

\begin{lem}
    [Iterated Splitting]
    \label{lem:iteratedsplitting} 
    Let \(T_1\) and \(T_2\) be two non-trivial locally finite \(G\)-trees in different deformation spaces. Then the vertex groups of each tree act non-trivially on the other tree.
    \begin{proof}
        Suppose \(x\) were a vertex of \(T_1\), Let \(K\) be it's stabilizer. Now \(K\) is a subgroup of \(G\) and so also acts on \(T_2\). If \(K\) had a global fixed point in \(T_2\)  then by local finiteness of \(T_1\) every vertex group of \(T_1\) would as well. Then by \ref{thm:ellipticimpliesequality} \(T_1\) and \(T_2\) are in the same deformation space; a contradiction. Therefore, \(K\) acts non-trivially on \(T_2\) as needed.
    \end{proof}
\end{lem}


We will now iterate the splitting process and then apply Bieri. To begin, by assumption each of the original splittings \(T_1\), \(T_2\), and \(T_3\) are non-trivial. Now, without loss of generality, consider a vertex group \(G_v\) from the \(T_3\) splitting corresponding to cutting along the D edges. By lemma \ref{lem:iteratedsplitting}, \(G_v\) acts non-trivially on \(T_1\) and \(T_2\). These non-trivial actions are the second iteration of the splitting process. The vertex group \(G_v\) is the fundamental group of  the vertex space \(X_v\), a VH-complex with two splittings from \ref{thm:wisegraph}; in fact \(X_v\) is a graph of spaces where the vertex and edge spaces are also graphs. Cut along the horizontal edges of \(X_v\) to obtain a splitting where each vertex space is a graph composed entirely of edges with a vertical label. Let \(w\) be a vertex in the graph of groups splitting and \(\Gamma_w\) the corresponding vertex space. The graph \(\Gamma_w\) can be regarded as a splitting over the trivial group -- but is it a trivial splitting? It could happen that \(\Gamma_w\) has trivial fundamental group. Here we need to assume that at least one sequence of iterated splittings ends in a graph with positive rank. Suppose this were the case for \(G_v\), that is when \(X_v\) splits there is a vertex space \(\Gamma\) which is a graph of positive rank. The fundamental group of \(\Gamma\) is a non-abelian free group and therefore has dimension one. This fact can also be seen as an application of Bieri to \(\Gamma\) where \(\Gamma\) is viewed as a non-trivial splitting over the trivial group. In this case the trivial group has dimension zero, so Bieri gives that \(\pi_1(\Gamma)\) has dimension one. After a second application of Bieri we get that \(G_v\) must have dimension two. Finally, after a third application of Bieri's dimension theorem \ref{pro:bireridimension} we know \(G\) has dimension 3, a contradiction. This completes the proof of the main theorem.



\noindent

\cleardoublepage

\phantomsection

\addcontentsline{toc}{chapter}{Bibliography}

% \bibliographystyle{amsalpha}

% \bibliography{paper}


% \begin{thebibliography}{9}


% \bibitem{bridsonhaefliger}
% Martin R. Bridson Andr ́
% eHaefliger
% Metric Spaces of
% Non-Positive Curvature

% \bibitem{stallingsfolds}
%  J. R. Stallings, Topology of finite graphs

% \bibitem{scottwall}
% topological methods in group theory

% \bibitem{burgermozes}
% Lattices in product of trees
% Burger, Marc  ; Mozes, Shahar

% \bibitem{manning}
% Cubulating spaces and groups, lecture notes
% (working draft – March 3, 2020)
% Jason F. Mannin

% \bibitem{levitt}
% Gilbert Levitt.
% \textit{Geometric group actions on trees}
% American Journal of Mathematics, Volume 119, Number 1, February 1997, pp83-102

% \bibitem{draftpaper}
% Forester and Martino
% \textit{Bounding complexity}
% some journal

% \bibitem{cullerandmorgan}
% Culler and Morgan
% \textit{Group Actions on $\mathbb{R}$-trees}
% Journal, Volume, Number, Dates (???)

% \bibitem{haglundss}
% ISOMETRIES OF CAT(0) CUBE COMPLEXES ARE SEMI-SIMPLE

% \bibitem{haglundwise}
% SPECIAL CUBE COMPLEXES
% Frederic Haglund and Daniel T. Wise

% \bibitem{mattclay}
% baumslag paper

% \bibitem{hymanbass}
% Hyman Bass.
% \textit{Covering theory for graphs of groups}
% Journal, Volume, Number, Dates (???)

% \bibitem{bieribook}
% \bibitem{wisethesis}

% \bibitem{wisecsc}
% Wise complete square complexes

% \bibitem{guirardelcorepaper}
% Guirardel

% \bibitem{serretrees}
% Serre, Trees

% \bibitem{foresterdeformationrigidity}
% Forester deformation and rigidity

% \bibitem{wisethesis}
% Wise Thesis

% \bibitem{wisecompactcore}
% A Covering Space with No Compact Core

% \bibitem{boundingcomplexity}
% Bounding the complexity of simpliciail group actions
% on trees 


% \bibitem{latexcompanion} 
% Michel Goossens, Frank Mittelbach, and Alexander Samarin. 
% \textit{The \LaTeX\ Companion}. 
% Addison-Wesley, Reading, Massachusetts, 1993.

% \bibitem{einstein} 
% Albert Einstein. 
% \textit{Zur Elektrodynamik bewegter K{\"o}rper}. (German) 
% [\textit{On the electrodynamics of moving bodies}]. 
% Annalen der Physik, 322(10):891–921, 1905.

% \bibitem{knuthwebsite} 
% Knuth: Computers and Typesetting,
% \\\texttt{http://www-cs-faculty.stanford.edu/\~{}uno/abcde.html}

% \end{thebibliography}

\bibliographystyle{amsalpha}
\bibliography{tonybib}

%\textsc{Mathematics Department, University of Oklahoma, Norman, OK 73019, USA}

%Email: \texttt{\href{mailto:bwstucky@ou.edu}{bwstucky@ou.edu}}

%URL: \texttt{\href{http://benstuc.ky}{http://benstuc.ky}}

\end{document}


%%% Fun facts

immersion = locally injective
for trees immersions are injective
elementary collapse moves are quasi-isometries thanks to old forester paper


        \begin{rmk}
            Properly discontinuous (here we mean the version with finitely many group elements) free actions on Hausdorff spaces are covering space actions cf: ch1 ex23 in Hatcher
        \end{rmk}
        
        
( add this defn somewhere bc it's used once: total space = raelization of graph of psaces)


===

    (Somewhere in the Niblo or related papers this should be citable (affine equi map). Also from Nir Lazaravich:
    
        As part of the proof, Dunwoody introduced two key tools: patterns and resolutions. He observed
        that any action of an almost finitely presented group, G, on a tree could be resolved to a G-tree
        obtained from a geometric pattern on the universal cover of the presentation complex of G. This
        resolution is simpler in certain aspects, e.g, the edge stabilizers are finitely generated and one
        can bound the number of parallelism classes of edges in the resolution. This result is known as
        Dunwoody’s Lemma ([4, Lemma 4.4])
        
        I don't see Lemma 4.4 in the exact publication specified but some items look to be close


    )
    
    % just make this a remark when it comes up
    \begin{pro}
    [Invariant Cell Complex]
    \label{pro:cellinvariant}
    (cell(X) = cell(gX) bc X = gX... but cell(gX) = gcell(X) by observation ergo cell(X) g invariant )
\end{pro}


=== Meeting: 31-07-2022

==== Figuring out the part where we need to get a contradiction from our axes in K tilde by morally placing them in a product.

ax1 in K~ all H .... pt in the ax ... send over to Tree x v. 

ax2 ... v x Tree

....

in CAT(0) any two orbits of points will fellow travel (really bc we are acting by isometries)


...


product BS map to X x Y
consider all H axis, pick point in K~
that aixs goes to tree x y
..... x x Tree

==== Following up

{ talk about getting two axes in K~, one composed entirely of vertical edges and the other composed entirely of horizontal edges }

From the splittings we get BS-maps from K~ -> X and K~ -> Y. Take the product and think about K~ -> X x Y a G-map. Pick a vertex in the vertical axis, it's orbit is sent to a vertical factor in X x Y. Similarly for horizontal axis. In X x Y the points are some distance a part but we are acting by isometries so the distance is constant but the images of the orbits are infinite discrete points. This violates fellow traveling.



