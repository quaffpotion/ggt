\documentclass{article}
\usepackage{amsmath,amssymb,amsthm, fancyhdr, tikz-cd}
\swapnumbers

\pagestyle{fancy}
\renewcommand{\sectionmark}[1]{\markright{\thesubsection\ #1}}
\fancyhf{}
\lhead{Geometric Group Theory Notes}
\rhead{Tony Martino}

\newtheoremstyle{mystyle}
  {\topsep}
  {\topsep}
  {}
  {}
  {\scshape}
  {.}
  {.5em}
  {}


\theoremstyle{mystyle}
\newtheorem{lem}{Lemma}[section]
\newtheorem{pro}[lem]{Proposition}
\newtheorem{defn}[lem]{Definition}
\newtheorem*{defn*}{Definition}
\newtheorem*{claim*}{Claim}
\newtheorem*{lem*}{Lemma}

\theoremstyle{remark}
\newtheorem{rmk}[lem]{Remark}
\newtheorem{ex}[lem]{Example}
\newtheorem{nex}[lem]{Not Example}



\begin{document}
\section{Definition}
\begin{defn}[Graph - Abstract]
	An {\bfseries Abstract Graph} \(G\) is a set \((V,E, \partial, i)\) where \(V\) and \(E\) are non-empty sets and \(\partial: E \to V\) and \(i: E \to E\) are functions satisfying \(i^2(e)=e\) and \(i(e) \neq e\) for all \(e \in E\).
\end{defn}

We set \(o(e) := \partial(e)\) and \(t(e) := (\partial\circ i)(e)\) for origin and terminal vertices and put \(\overline{e} := i(e)\).

\begin{defn}[Fold]
	(placeholder)
\end{defn}
\end{document}
