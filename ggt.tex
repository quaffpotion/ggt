\documentclass{article}
\usepackage{amsmath,amssymb,amsthm, fancyhdr, tikz-cd, xcolor, mathrsfs,hyperref,cleveref}
\definecolor{mypink1}{rgb}{0.858, 0.188, 0.478}
\swapnumbers

\pagestyle{fancy}
\renewcommand{\sectionmark}[1]{\markright{\thesubsection\ #1}}
\fancyhf{}
\lhead{Research Outline}
\rhead{Tony Martino}

\newtheoremstyle{mystyle}
  {\topsep}
  {\topsep}
  {}
  {}
  {\scshape}
  {.}
  {.5em}
  {}

\newcommand{\sxyz}{((S_x)_y)_z}
\newcommand{\rxy}{(R_x)_y}

\theoremstyle{mystyle}
\newtheorem{thm}{Theorem}[section]
\newtheorem{thm*}{Theorem}
\newtheorem{lem}{Lemma}[section]
\newtheorem{pro}{Proposition}
\newtheorem{defn}{Definition}
\newtheorem*{defn*}{Definition}
\newtheorem*{claim*}{Claim}
\newtheorem*{lem*}{Lemma}

\theoremstyle{remark}
\newtheorem{rmk}{Remark}[section]
\newtheorem{ex}{Example}[section]
\newtheorem{nex}{Not Example}[section]



\begin{document}

\section{Problem Statement}


We want to show the following statement or similar: if a group is the fundamental group of a VH-complex, then there are at most two pairwise inequivalent actions on locally finite trees with FP vertex stabilizers. The following proof sketch uses or builds on VH-complexes introduced by Wise, Guirardel's core, and a theorem of Bieri. To start we have that \(G = \pi_1(K)\) where \(K\) is a VH-complex; according to Wise this comes with a vertical and horizontal splitting. For sake of a contradiction suppose there was a third tree as above. A generalization of Guirardel's core to three actions would imply that our \(G\) was the fundamental group of a VHD-complex (here ``D'' is for ``depth'') which would further imply, among other things, that our group splits along groups of cohomological dimension two. (This is analagous to Wise's work on VH-complexes) However, a theorem of Bieri and local finiteness forces the cohomological dimension of the resulting group to be dimension three which contradicts our original VH assumption so there are at most two such actions.

\subsection{Definitions}

\begin{defn}
	[Type \(FP\) ]
	A group is of type \(FP\) if it is (1) type \(FP_n\) for all \(n\) and (2) finite geometric cohomological dimension.
\end{defn}

\begin{defn}
	[Finite Type] 
	An action of {\em finite type} is one on a locally finite tree where  vertex stabilizers are of type FP.
\end{defn}

\begin{defn}[open Direction] An open direction is a connected component of an \(\mathbb{R}\)-tree minus a point. 
\end{defn}

\begin{defn}[Closed Direction] A closed directon is a connected component of an \(\mathbb{R}\)-tree minus a point, union that point.
\end{defn}
\begin{defn}[Open Halfspace] An open halfspace is an open direction obtained from deleting the midpoint of an edge.
\end{defn}
\begin{defn}[Closed Halfspace] A closed halfspace is a closed direction obtained from deleting the midpoint of an edge.
\end{defn}
\begin{defn}[Halfspaces of a product] An open (resp. closed) halfspace of a product (at a certain index) is a subset where one projection is an open (resp. closed)  halfspace in it's factor and the others are onto.
\end{defn}
\begin{defn}[Generalized quadrants] A generalized open (resp. closed) quadrant with respect to a product of \(k\) spaces is an intersection of \(k\) open (resp. closed) product halfspaces where each one is at a different index.
\end{defn}
\begin{defn}[cellular-product-convex] We say that \(K \subset X\) is cellular-product-convex if it's complement is the open cellular neighborhood of a union of generalized closed quadrants.
\end{defn}

\begin{defn}
	[Filling]
	Let \(\{X_{k}\}_{k}\) be a family of spaces where one can take convex hulls. Given \(S \subseteq X := \prod X_{k}\) define \(S_{k}\) via: \[p \in S_{k} \iff \exists \,q,r \in S: \forall j \neq k: p_{j} = q_{j} = r_{j} \text{ and } p_{k} \in \pi_{k} (\{q_{k} , r_{k}\}).\] 
\end{defn}

\subsection{Misc Statements}

\begin{lem}
    [Filling Lemma in \(\mathbb{R}^{2}\)]
	\label{lem:fillr2}
	If \(R \subseteq \mathbb{R}^{2}\) is connected, then \(\rxy\) is 1-dim fiber convex.
\end{lem}
\begin{proof}
    (sketch) This one is miai with a line that's left out
\end{proof}
\begin{lem}
    [Filling Lemma in \(T_{1} \times T_{2}\)]
	\label{lem:fillt2}
\end{lem}
\begin{proof}
    (sketch) find replace the proof in \(\mathbb{R}^{2}\) with corresponding words for trees e.g. connecting becomes convex hull, left becomes inside a half space etc.
\end{proof}


\section{Outline and notation for the construction of the core}

	From Guirardel we get a map from the universal cover of our VH complex to the product of three trees by taking inclusion in the first two factors and Guirardel's map in the last factor. We obtain \(S = \text{Im}(f)\) and will show it's cocompact. Then we put \(K = \text{hull}{S}\) and show it's still cocompact. Finally we fill \(K\) in all three directions obtaining the core \(C\). Again, this needs to be cocompact. The main goal is to show that \(C\) is actually QC as well so that we get a legitimate graph of spaces decomposition of \(G\).

\section{Octant Convexity of the core}
\begin{thm}
    [\label{thm:QCOC}Slices QC implies OC for trees]
	Let \(X=T_{1} \times T_{2} \times T_{3}\) where each \(T_{i}\) is a locally finite tree. Let \(K\) be a subcomplex of \(X\) that satisfies: 
	\begin{itemize}
		\item connected
		\item has connected hyperplanes
		\item has cellular-product-convex hyperplanes 
	\end{itemize}
Then \(K\) is also cellular-product-convex.
\end{thm}
We need four lemmas, the last two give the implication after taking intersections:

\begin{lem}
	[\label{lem:cubeOC}Cube OC] Let \(S \subseteq \mathbb{E}  \) be a closed set such that \(S\) satisfies: 
	\begin{enumerate}
		\item \((0,0,0) \not\in S\) 
		\item \(S\) connected
		\item \(S \cap (\text{coordinate-plane})\) is connected.
		\item \(S \cap (\text{coordinate-plane})\) is disjoint from some closed quadrant in that plane.
	\end{enumerate}
	Then \(S\) is disjoint from some closed octant of \(\mathbb{E}\). 
	\begin{proof}
		(todo, should follow from notes on the \(\mathbb{R}^{3}\) case.)
	\end{proof}
\end{lem}

\begin{lem}
	[\label{lem:Xtocube}X to cube]
	If \(S\) satisfies the hypotheses of \ref{thm:QCOC} then \(\rho S\) satisfies the hypotheses of \ref{lem:cubeOC} provided \(\rho\) is a projection map to a cube determined by a midpoint disjoint from \(S\).
\begin{proof}
	(todo)	
\end{proof}
\end{lem}

\begin{lem}
	[Connecting logic]
	If \(\rho S\) satisfies the conclusion of lemma \ref{lem:cubeOC} for each cube disjoint from \(S\), then \(S\) satisfies the conclusion of theorem \ref{thm:QCOC}.
\end{lem}


\section{Proving core is hyperplane QC}
One path is to show that \(C\) is 1-dimensional fiberwise connected and then apply Guirardel's lemma in each hyperplane to conclude that they're QC as needed.

\begin{lem}
	[Guirardel Lemma 5.4, Corollary 5.5]
	Let \(T_{1} , T_{2}\) be two \(\mathbb{R}\)-trees and let \(F\) be a nonempty connected subset of \(T_{1} \times T_{2}\) with convex fibers. Then the complement of \(\overline{F}\) is a union of quadrants. That is, \(\overline{F}\) is also nonempty, connected, and has convex fibers.
\end{lem}

\begin{lem}
	[Planar path argument in \(\mathbb{R}^{3}\) ]
	\label{lem:path}

	If \(S \subseteq \mathbb{R}^{3}\) is connected in all coordinate planes then \(S_x\), \(S_{y}\), and \(S_{z}\) are as well. 
	

\end{lem}

\begin{lem}
    [Filling Lemma in \(\mathbb{R}^{3}\)]
	\label{lem:fillr3}
	If \(S \subseteq \mathbb{R}^{3}\) is itself connected and connected in all coordinate planes, then \(\sxyz\) is 1-dim fiber convex.
\end{lem}
\begin{proof}
    (sketch) Need to use the Guirardel lemma to get two ways of writing the quadrant-convex hull of a set. This allows switching. Then assuming things are connected in all planes get that one filling is still connected and so you can apply the switching again. (There is a planar path argument to make) Then you get the result.
\end{proof}
\begin{lem}
    [Filling Lemma in \(T_{1} \times T_{2} \times T_{3}\)]
	\label{lem:fillt3}
    (Here we're taking convex hulls in the tree factors)
\begin{proof}
    (todo)
\end{proof}
\end{lem}


\section{Recovering Tree actions from VHD-complex}


\end{document}
words to search for: clarification, question, technicality, concern



#justVimThings
surround a selection of text: v to select some text, press "S", then a delimiter
	- works with (, {, [ and <p>, <body>, etc.
delete cursor to beginning of word: d/<type word here>
	- works on multiple lines
move down by display lines: prefix with g e.g. gj, gk, g0, g$
start search backwards via ?


