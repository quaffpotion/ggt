\documentclass{article}
\usepackage{amsmath,amssymb,amsthm, fancyhdr, tikz-cd, xcolor, mathrsfs,hyperref,cleveref}
\definecolor{error}{rgb}{1,0 , 0}
\swapnumbers

\pagestyle{fancy}
\renewcommand{\sectionmark}[1]{\markright{\thesubsection\ #1}}
\fancyhf{}
\lhead{Research Outline}
\rhead{Tony Martino}

%Commands
\newtheoremstyle{mystyle}
  {\topsep}
  {\topsep}
  {}
  {}
  {\scshape}
  {.}
  {.5em}
  {}

\newcommand{\sxyz}{((K_x)_y)_z}
\newcommand{\rxy}{(R_x)_y}
\newcommand{\ryx}{(R_y)_x}
\newcommand{\Sxyz}[3]{((K_#1)_#2)_#3}

\theoremstyle{mystyle}
\newtheorem{thm}{Theorem}[section]
\newtheorem{thm*}{Theorem}
\newtheorem{lem}{Lemma}[section]
\newtheorem{pro}{Proposition}
\newtheorem{defn}{Definition}
\newtheorem*{defn*}{Definition}
\newtheorem*{claim*}{Claim}
\newtheorem*{lem*}{Lemma}
\newtheorem*{cor*}{Corollary}


\theoremstyle{remark}
\newtheorem{rmk}{Remark}[section]
\newtheorem{ex}{Example}[section]
\newtheorem{nex}{Not Example}[section]



\begin{document}
\tableofcontents

\section{Toy Version}
\subsection{Cocompactness Argument}

\begin{lem} 
[Filling preserves cocompactness]
\label{lem:fillingcocompact}
We have that $(S_{23})_x$ is cocompact.
\begin{proof}
    Recall that $S_{23}$ is itself cocompact and is contained in a set that can be written as $A\times B$ where $A\subset T_1$ and $B\subset T_2\times T_3$ with $B$ cocompact as well. Also remember that $G$ acts freely on $T_2\times T_3$. For each vertex orbit of $B$ choose a particular vertex. Call them $\{b_1,\ldots,b_n\}$; by cocompactness this list is finite. By freeness in the second factor, the stabilizer of $S' := S_{23} \cap (T_1 \times \{b_1\})$ is trivial. This means that $S'$ injects into $S_{23}/G$ which is compact by assumption. Since we are dealing with cell complexes, $S'$ is also compact. Repeating this argument a finite number of times we have that $S_{23}\cap (T_1\times \{b_k\})$ is compact for each \(k\). Compact items have a well defined diameter, taking the maximum diamater and noting we're acting by isometries gives that there is a bound on the diameters of \(S_{23} \cap (T_1\times \{v\})\) as \(v\) ranges over vertices of \(B\). Lastly, because \(S_{23}\) is contained in a product, \textcolor{error}{[warning] if \(S_{23}\) were an actual product and we had a bounded slice for a single vertex \(v\) in the second factor then we could be done right away because we could move \(v\) around without changing the projection - in fact we could skip the multiple orbit argument } \textcolor{blue}{[fix] Are we using the productness of A x B? In which case, we need A to be bounded ... which seems correct(??) by construction with how we map to T1 }


    this argument applies if $v$ is replaced by a general point in the projection of \(S_{23}\) onto $T_2 \times T_3$.
    
    Consider a point $p \in (S_{23})_x$. By the definition of filling, this point $p$ is contained in a slice of $S_{23}$, that is $p \in S_{23}\cap (T_1 \times \{r\})$ for some $r$ in $T_2\times T_3$. 
    
        Because there is a universal bound on the diameter of $S_{23}\cap (T_1 \times \{r\})$ over all possible $r$ we have that $(S_{23})_x$ is contained in an $R$ neighborhood of $S_{23}$. Together with the local finiteness of $T_1 \times T_2 \times T_3$ we conclude that $(S_{23})_x$ is cocompact as needed. 
    
    
    
    
    
    
\end{proof}
\end{lem}

\subsection{Promote elliptic containment to equality}
\subsubsection{Definitions}
\begin{defn}[morphism]
A map $X\to Y$ between two trees is called a morphism if it sends vertices to vertices and edges to edges. 

(Formally, there is something to say, however if you think of your edges as intervals of the real line then you might as well take the homeomorphisms to be linear maps. For our purposes, only combinatorial information is relevant. If the maps stretch or squish parts of edges we wouldn't care)
\end{defn}

\begin{lem}[invariant map from elliptic inclusion]
Given invariant trees $X$ and $Y$ such that $\mathcal{E}(X)\subset\mathcal{E}(Y)$, there exists an equivariant map $X\to Y$.
 
 \begin{proof}
 Consider the vertex orbits in $X$. Pick a vertex from each orbit. Consider how the stabilizers of these vertices in $X$ act on $Y$. Because of the elliptic subgroup containment, each $G_{x_i}$ we picked out fixes a non-empty set of vertices in $Y$. Begin to define a map on the 0-skeleton by sending $x_i$ to something in $Y$ fixed by $G_{x_i}$. There are several choices, but a fixed set of choices plus the invariant condition defines a map on the 0-skeleton.
 
The containment says such a map is well-defined. Indeed, pick $x \in X^{(0)}$. Suppose $x=gx_0=hx_0$, then $(h^{-1}g)x_0 = x_0$ and so $h^{-1}g \in G_{x_0}$. Then $f(x_0)=f( (h^{-1}g)x_0)$ which gives $h f(x_0) = f(gx_0)$ and so $f(hx_0)=f(gx_0)$ as needed.
 
 Once we have a map on the 0-skeleton we can extend it to the entire tree by drawing unique geodesics in the trees.
\end{proof}
\end{lem}

\begin{thm}[Factoring as folds, from Bestvina paper]
Let $G$ be a finitely generated group. Suppose that $\alpha: T'\to T$ is a simplicial equivariant map from a $G$-tree $T'$ to a minimal $G$-tree $T$ such that no edge in $T'$ is mapped to a point by $\alpha$. If all edge stabilizers of $T$ are finitely generated and if $T'/G$ is finite, then $\alpha$ can be represented as a finite composition of folds.
\end{thm}

\subsubsection{Result}
\begin{lem}[Promote Elliptic Containment]
Given an invariant morphism $X \to Y$ where $Y$ is locally finite and both actions are minimal we show that $X$ is necessarily locally finite and that $X$ and $Y$ have the same elliptic subgroups.
\end{lem}

\begin{pro}[Holds for Folds]
Given a fold $X\to Y$ where $Y$ is locally finite we have the following:
\begin{enumerate}
    \item $X$ is locally finite
    \item If an element acts hyperbolically on $X$ then it acts hyperbolically on $Y$ as well.
\end{enumerate}

\end{pro}

\begin{pro}[Holds for collapsing]
Given a map $X\to Y$ that's given by collapsing trees to vertices and where $X$ and $Y$ are locally finite we have the following. If an element acts hyperbolically on $X$ then it acts hyperbolically on $Y$ as well.

\end{pro}

\subsection{Toy Version}

\subsubsection{Cited Theorems}

\begin{thm}[F, ``On uniqueness...'', Thm 3.2]
\label{thm:forester}
Let $G$ be a group and let $X$ and $Y$ be cocompact $G$-trees. Then $X$ and $Y$ are related by an elementary deformation if and only if they have the same elliptic subgroups.
\end{thm}

\subsubsection{Definitions}

\begin{defn}[Finite Type]
We say a tree action is of \emph{finite type} if the action is non-trivial, the tree is locally finite, and the vertex stabilizers are FP, with finite quotient graph.
\end{defn}

\begin{rmk}
    We will assume a group of dimension 2 and in that case Bieri gives that actions of finite type have vertex and edge stabilizers that are finitely generated free groups.
\end{rmk}

\begin{rmk}
    Bieri gives us that if we have an action of finite type then the group is FP.
\end{rmk}

\begin{defn}[Transverse]
    We say that two tree actions $X$ and $Y$ are \emph{transverse} if they are not in the same deformation space and there exist two stabilizers, one for each tree, such that their intersection is FP.
\end{defn}

\begin{rmk}
    The definition of transverse does not depend on the vertices chosen and remains unchanged up to deformation spaces
\end{rmk}

\subsubsection{Result}

\begin{pro}
 Let $G$ a group of cohomological dimension 2. If $X$ and $Y$ are $G$-trees of finite type that are in different deformation spaces then the following are equivalent:
\begin{enumerate}
    \item $X$, $Y$ transverse
    \item $x \in V(X), y\in V(Y)\Longrightarrow G_x\cap G_y = \{1\}$
    \item There exists a cocompact VH-complex $K$ with $\pi_1(K) = G$ whose horizontal and vertical splittings are $X$ and $Y$.
\end{enumerate}
\begin{proof}
\begin{enumerate}
    \item $1\Rightarrow 2$: Fix $x_0 \in V(X)$ Let $y\in V(Y)$. Then $G_{x_0} \cap G_y = (G_{x_0})_y$. By (1) $X$ is transverse to $Y$ hence $G_{x_0}\cap G_y$ is FP. Since the choice of $y\in V(Y)$ was arbitrary, the vertex groups of the $G_{x_0}$ action on $Y$ are FP. Note, $Y$ locally finite implies it's edge groups are finite index subgroups of it's vertex groups. Hence the edge groups are also FP. We claim that the action of $G_{x_0}$ on $Y$ is non-trivial. Given this we apply Bieri to get:
    \begin{align*}
        2 &= dG\\
          &= dG_{x_0}+1\\
          &= d( G_{x_0} )_y+1+1\\
          &= d(G_{x_0}\cap G_y)+2
    \end{align*}
    
    Hence, $d(G_{x_0}\cap G_y)=0$ so $G_{x_0}\cap G_y$ is trivial.
    
    \begin{claim*}
    The action of $G_{x_0}$ on $Y$ is non-trivial.
    \begin{proof}
    Suppose the action were trivial. That is, there exists some $y\in V(Y)$ such that $(G_{x_0})_y=G_{x_0}$. Hence, $G_{x_0}$ is elliptic for the action of $G$ on $Y$. By the local finiteness of $Y$, for all $x\in V(X)$ $G_x$ acts elliptically on $Y$. Hence, $\mathcal{E}(X)\subset \mathcal{E}(Y)$. Again by local finiteness we can promote this to $\mathcal{E}(X) = \mathcal{E}(Y)$ which by theorem \ref{thm:forester} gives $X \sim Y$ which contradicts the fact that $X$ and $Y$ were assumed to be in different deformation spaces.
    \end{proof}
    \end{claim*}
    \item $2\Rightarrow 1$: Trivial groups are FP.
    \item $2\Rightarrow 3$: Take $X \times Y$ and give it the VH-structure where $X$ and $Y$ correspond to horizontal and vertical edges respectively. Condition (2) says that $G$ acts freely on $X\times Y$. Guirardel gives us a simply-connected cocompact core $C \subset X \times Y$. The action is also free on subsets of $X \times Y$. Since the action is cellular and free it's a covering space action. (We avoid situations like irrational rotations on a circle that are free but not covering space actions) 
    
    It remains to show that $C/G$ is VH. Is it enough to say that the action respects the tree factors. (The edge partition on the cover descends to a well-defined edge partition on the quotient and attaching maps constructed in the standard way for the quotient alternate between vertical and horizontal edges)
    
    \item $3\Rightarrow 2$: Suppose $1\neq g \in G$ is an element of $G_x\cap G_y$ where $x \in V(X)$ and $y\in V(Y)$. Since $K$ is a VH-complex it has a decomposition as a graph of groups where the vertex spaces are the connected components of the set of vertical edges in the 1-skeleton of $K$. Each $x \in V(X)$ is in correspondence with the inclusion of a vertex space (composed entirely of vertical edges) into $K$. The inclusion of vertex spaces is always injective on fundamental groups. Lastly, after picking basepoints the image of the induced map is the stabilizer of $x$.
    
    By Wise, we have that the universal cover of $K$ is contained in $X \times Y$. Because the action respects the product structure, and because vertex spaces (in our case, graphs) are covered by embedded copies of their own universal covers we have that the non-trivial element $g$ when represented by a loop in a vertex space lifts to a path with distinct endpoints in a tree consisting entirely of vertical edges in $X \times Y$. In fact, $G_x$ acts on $\{x\}\times Y$ and freely on $\widetilde{K} \cap (\{x\}\times Y)$ because the action on $\widetilde{K}$ is a covering space action. This means that $g$ acts hyperbolically on $Y$ and we get an axis in $Y$. This remains an axis in $X \times Y$.
    
    Thus from looking at the vertical splitting (where vertex spaces were made of vertical edges) we obtained an axis consisting entirely of vertical edges. Similarly, after looking at the horizontal splitting we obtain an axis consisting entirely of horizontal edges. 
    
    Finally, since $X \times Y$ is CAT(0) these axes would have to be parallel. This is a contradiction.
\end{enumerate}
\end{proof}
\end{pro}

\subsection{Combining into main argument}
We need ... \ref{lem:fillingcocompact}.

\section{Problem Statement}
We want to show the following statement or similar: if a group is the fundamental group of a VH-complex, then there are at most two pairwise inequivalent actions on locally finite trees with FP vertex stabilizers. The following proof sketch uses or builds on VH-complexes introduced by Wise, Guirardel's core, and a theorem of Bieri. To start we have that \(G = \pi_1(K)\) where \(K\) is a VH-complex; according to Wise this comes with a vertical and horizontal splitting. For sake of a contradiction suppose there was a third tree as above. A generalization of Guirardel's core to three actions would imply that our \(G\) was the fundamental group of a VHD-complex (here ``D'' is for ``depth'') which would further imply, among other things, that our group splits along groups of cohomological dimension two. (This is analagous to Wise's work on VH-complexes) However, a theorem of Bieri and local finiteness forces the cohomological dimension of the resulting group to be dimension three which contradicts our original VH assumption so there are at most two such actions.

\subsection{Dramatis Person\ae}
\begin{itemize}
    \item \(X\) a compact connected VH-complex
    \item \(G = \pi_{1} (X)\) 
    \item \(T_{1}\) the tree from the horizontal splitting of \(X\)
    \item \(T_{2}\) the tree from the vertical splitting of \(X\) 
    \item \(T_{3}\) an interloping locally finite \(G\)-tree with FP vertex stabilizers
    \item \(\mathscr{T} = T_{1} \times T_{2} \times T_{3} \) 
    \item \(X^{+}\) a certain complex containing \(X\)
    \item \(f: X^{+} \to \mathscr{T}\) an equivariant map
    \item \(\Gamma \) a certain compact connected subgraph of \(X^{(1)}\) the 1-skeleton of \(X\)
    \item \(J = \text{Im}(f)\) 
    \item \(K = \text{cell}(J)\) 
    \item \(C = \text{fill}(K)\) the hero of our story, the simply-connected core on which \(G\) acts
\end{itemize}

\subsection{Definitions}

\begin{defn}
    [Split Maps]
    \label{def:splitmaps} 
    Given a graph of spaces we obtain an invariant map as follows ...
\end{defn}

\begin{defn}
    [Setup]
    \label{def:setup} 
    We say that \(K = (T, x_{0},\sigma=\sigma_{1}\cdots\sigma_{N},\lambda)\) satisfies property {\em setup} if:
    \begin{enumerate}
        \item \(T\) is a simplicial tree
        \item We have that \(x_{0}\) is a valence one vertex
        \item \(\sigma\) is a concatenation of non-degenerate edgepaths
        \item \(\sigma (t)=x_{0} \implies t \in \{0,1\}\) 
        \item \(\lambda (1) \neq \lambda (N)\) 
    \end{enumerate}
\end{defn}


\begin{defn}
    [Bad]
    Given \(K = (T, x_{0},\sigma=\sigma_{1}\cdots\sigma_{N},\lambda)\) we say that an edge is {\em bad} if:
    \begin{enumerate}
        \item The edge \(e\) separates the basepoint from some endpoint. i.e. \[e^{+} \cap \text{endpoints} \neq \varnothing\] where \(e^{+}\) is the halfspace of \(e\) not containing \(x_{0}\).
        \item The edge \(e\) always has the same color. i.e. \(\left| \{\lambda (k) \mid e \subseteq \text{Im} \sigma_{k}\} \right|=1\).
    \end{enumerate}
\end{defn}

\begin{defn}
	[bad edge snipping]
    Let \(K\) satisfy property setup with \(e\) a bad edge. Let \(\sigma_{i}\) and \(\sigma_{j}\) be the first and second subpaths of \(\sigma\) that use \(e\). Take \(\sigma '\) to be the concatenation of \(\sigma_{k}\) for \(k < i\) with the subpath denoted \(\sigma '_{ij}\) that is the concatentation of the largest initial subpath of \(\sigma_{i}\) not using \(e\) and the longest tail of \(\sigma_j\) not using \(e\) with \(\sigma_{k}\) for \(k > j\). The unmodified subpaths of \(\sigma\) in \(\sigma '\) receive the same colors as before and we take \(\sigma '_{ij}\) to have the color that \(\sigma_{i}\) and \(\sigma_{j}\) shared.
\end{defn}
\subsubsection{Miscellaneous}
\begin{lem}
    [Geometric Condition]
    \label{lem:simpgeo} 
    (Theorem 0.6 in LP97)
    A minimal simplicial action of a finitely generated group is geometric if and only if all edge groups are finitely generated.
\end{lem}
\begin{defn}
    [Fiberwise Connected]
    \label{def:connfibers} 
    Let \(S \subseteq X \times Y \times Z\). If \(S \cap \{\text{pt}_1\} \times \{\text{pt}_2\} \times Z\) and all similar sets as well as permutations are connected then we say \(S\) is one dimensional fiberwise connected.
\end{defn}
\begin{lem}
    [Extension]
    \label{lem:guirardel-extension}
    (Lemma 8.9 in Guirardel)
    Consider a geometric action of a finitely generated group \(G\) on an \(\mathbb{R}\)-tree \(T\), and let \(X\) be a 2-complex endowed with a free properly discontinuous cocompact action of \(G\). Let \(\mathscr{F}\) be a \(G\)-invariant measured foliation on \(X\). Consider a map \(f: X \to T\) which is constant on leaves of \(\mathscr{F}\), and isometric in restriction to transverse edges of \(X\). Then there exists a 2-complex \( X'\) containing \(X\), endowed with a free properly discontinuous cocompact action of \(G\), a measured foliation \(\mathscr{F} '\) extending \(\mathscr{F}\), and which induces an isometry between \(X'/ \mathscr{F}'\) and \(T\). Moreover, the inclusion \(X \subseteq X'\) induces an epimorphism of fundamental groups.
\end{lem}
\begin{lem}
    [Affine Equivariant Map]
    \label{lem:equivariant} 
    Suppose that \(G\) acts freely on a simplicial complex \(K\) and acts on a simplicial tree \(T\). Then there exists an equivariant map \(f: K \to T\) where \(\mathscr{F}\) the connected components of the fibration from \(f\) is a measured foliation and \(f\) is an isometry on edges transverse to \(\mathscr{F}\).
    \begin{proof}
        Part 1: Construct an equivariant map.

        We start by defining \(f\) on \(K^{(0)}\) the 1-skeleton. By equivariance it is enough to define the map on a single vertex in each vertex orbit. These choices can be arbitrary. Next we check that the resulting map is well-defined. Indeed, if \(gv=hv\) then \(g^{-1} h = 1\) by freeness and so \[ f(v)=g^{-1}h f(v)=g^{-1} f(hv) \] but then \[ f(gv)=gf(v)=f(hv).\] 
        
        Next we define the map on edges. If \(vw\) is an edge, map it to the geodesic \([f(v),g(w)]\). 
        
        Lastly, for 2-cells we use the standard fibration from mapping triangles to tripods.
        
        Part 2: Fibration details.
    \end{proof}
\end{lem}
\begin{lem}
	[Guirardel Lemma 5.4, Corollary 5.5]
 \label{lem:guirardel} 
	Let \(T_{1} , T_{2}\) be two \(\mathbb{R}\)-trees and let \(F\) be a nonempty connected subset of \(T_{1} \times T_{2}\) with convex fibers. Then the complement of \(\overline{F}\) is a union of quadrants. That is, \(\overline{F}\) is also nonempty, connected, and has convex fibers.
\end{lem}
\begin{claim*}
    If \(K\) satisfies property setup then the first and last edge of \(\sigma\) are equal.
\end{claim*}
\begin{defn}
	[Filling]
    Let \(\{X_{k}\}_{k \in K}\) be a family of spaces where one can take convex hulls. Given \(S \subseteq X := \prod X_{k}\) define \(S_{k}\) for \(k \in K\) via: \[p \in S_{k} \iff \exists \,q,r \in S: \forall j \neq k: p_{j} = q_{j} = r_{j} \text{ and } p_{k} \in \text{cvxhull}_k (\{q_{k} , r_{k}\}).\] 
\end{defn}
\begin{defn}
	[Type \(FP\) ]
	A group is of type \(FP\) if it is (1) type \(FP_n\) for all \(n\) and (2) finite geometric cohomological dimension.
\end{defn}

\begin{defn}
	[Finite Type] 
	An action of {\em finite type} is one on a locally finite tree where  vertex stabilizers are of type FP.
\end{defn}

\begin{defn}[Open Direction] An open direction is a connected component of an \(\mathbb{R}\)-tree minus a point. 
\end{defn}

\begin{defn}[Closed Direction] A closed directon is a connected component of an \(\mathbb{R}\)-tree minus a point, union that point.
\end{defn}
\begin{defn}[Open Halfspace] An open halfspace is an open direction obtained from deleting the midpoint of an edge.
\end{defn}
\begin{defn}[Closed Halfspace] A closed halfspace is a closed direction obtained from deleting the midpoint of an edge.
\end{defn}
\begin{defn}[Halfspaces of a product] An open (resp. closed) halfspace of a product (at a certain index) is a subset where exactly one projection is an open (resp. closed)  halfspace in it's factor and the others are onto.
\end{defn}
\begin{defn}[Generalized quadrants] A generalized open (resp. closed) quadrant with respect to a product of \(k\) spaces is an intersection of \(k\) open (resp. closed) product halfspaces where each one is at a different index.
\end{defn}
\begin{defn}[cellular-product-convex] We say that \(K \subset X\) is cellular-product-convex if it's complement is the open cellular neighborhood of a union of generalized closed quadrants.
\end{defn}



\section{Outlines}
Our goal is to show that when we have three different actions there exists a well chosen cube complex that \(G\) acts on which has a quotient that is a graph of spaces decomposition for \(G\). Using properties of this splitting we will conclude that our group has dimension at least three which is impossible because we assumed it was VH.

    In terms of properties of the core we need three things: (1) hyperplanes are simply connected, (2) cocompact, (3) hierarchy of splittings. We will take the edge spaces of our graph of spaces to be a subset of the corridors of hyperplanes and the vertex spaces to be the components of the complement of the edge spaces.


    Our main result is the following:

    \begin{cor*}
        Let \(G\) be the fundamental group of a compact VH-complex that doesn't satisfy {\em property A}. Let \(T_{v}\), and \(T_{h}\) be the vertical and horizontal splittings. Considering {\em nice} actions up to deformation these are the only two such actions. 
    \end{cor*}

\subsection{Misc lemmas}

\begin{lem}
    [Invariant to bounded distance]
    Let \(A\) and \(B\) be invariant subcomplexes of \(X\) with \(G\) acting cocompactly on both \(A\) and \(B\) after restricting the action on \(X\). Then \(A\) and \(B\) are Hausdorff equivalent. 
    \begin{proof}
        \textcolor{error}{[warning]Because we are dealing with cocompact actions on cell complexes for the action on \(A\) there exists a subcomplex [we need subcomplex because we need closed because we declare a distance later from a set to a point]} \(F \subseteq A\) such that \(F\) is bounded with orbit \(A\). We call \(F\) a fundamental domain. Let \(D_{1}\) be the diameter of \(F\). Pick some \(b_{0} \in B\) and let \(D_{2}\) be the distance from \(F\) to \(b_{0}\). Pick an arbitrary \(a \in A\). Because the orbit of \(F\) is \(A\) there exists some \(g \in G\) such that \(a \in gF\). Then \(a\) is within \(D_{1} + D_{2}\) of \(g y\). Hence, \(A\) is contained in the \(D_{1} + D_{2}\) neighborhood of \(B\). Switching \(A\) and \(B\) in this argument and taking the maximum of the distances shows that \(A\) and \(B\) are both contained in \(R\) neighborhoods of each other for some \(R\); that is they are Hausdorff equivalent.
    \end{proof}
\end{lem}


\subsection{Main Equivariant Map Construction}
Let \(X\) be a compact VH-complex and set \(G= \pi_{1} X\). Form the cover \(\widetilde{X} \to X\). Note that \(G\) acts on \(\widetilde X\) freely and PDC. Since \(X\) is a VH-complex we get two actions of \(G\) on trees \(T_{1}\) and \(T_{2}\) along with invariant \ref{def:splitmaps} maps \(f_{1}\) and \(f_{2}\) from the splitting. Suppose we had a third action of \(G\) on a tree \(T_{3}\) with property {\em nice}. Given the covering space action and the action on \(T_{3}\) we use the affine construction \ref{lem:equivariant} to get an equivariant map \(f_{3} : \widetilde X \to T_{3}\). Using Guirardel 8.9 \ref{lem:guirardel-extension} we extend \(f_{3}\) to \(\widehat f_{3}\) a map with connected fibers. We also need the proof of lemma \ref{lem:guirardel-extension} to ensure certain properties hold. Then, in order to extend the \(f_{1}\) and \(f_{2}\) maps we use the coning off construction. The product of these extensions gives \(f\).
    
    \subsection{Simply connected hyperplanes}
    To be a valid decomposition we need that the edge spaces are injective on fundamental groups. For this we need the corollary below.
\begin{cor*}
    The hyperplanes of \(C\) are simply connected. 
\end{cor*}

Which will follow from:

\begin{thm}
    The hyperplanes of \(C\) are quadrant convex.    
\end{thm}

One way to prove the theorem is to first show that \(C\) is one dimensional fiberwise connected \ref{def:connfibers} and then apply Guirardel's lemma \ref{lem:guirardel} in each hyperplane to conclude that they're quadrant convex as needed.

\begin{thm}
    [Connected fibers]
    The one dimensional fibers of \(C\) are connected.
\end{thm}


To achieve this, we'll build \(C\) from a suitable \(G\)-invariant subcomplex \(K \subseteq \mathscr{T} \) by taking \(C = \sxyz\).  Here \(K\) is the cellular hull of \(J\), the image of \(f\). 




\subsection{Cocompactness of the Core}
The second requirement that \(C\) be cocompact will follow from an argument that starts by bounding the size of fibers. 
 \begin{lem}
     [Finite fibers]
     ...need to latex...
 \end{lem}

 \begin{lem}
     [Universal bound on fibers]
     ...need to latex...
 \end{lem}


\subsection{Required splitting properties}
Note first we establish that \(C\) is itself simply connected.
\begin{lem}
    The core \(C\) is simply connected.
\end{lem}

Finally, having enough dimension will follow from some assumptions about our actions.
With this splitting in hand we want to verify that it's made of successive graphs of groups of items of a certain dimension so we can apply Bieri.

\begin{lem}
    [Directional splittings are non-trivial]
\end{lem}

\begin{thm}
    [Not All Trees]
\end{thm}

\begin{lem}
    The action of \(G\) on \(T_1 \times T_{2} \) is free.
    \begin{proof}
        ...look in section II.6 of BH... Need enough facts to avoid non-proper spaces, get a semisimple action, and axes.
    \end{proof}
\end{lem}

\section{Core has quadrant convex hyperplanes}

\begin{lem}
    [Connected Fibers]
    \label{lem:confib} 
    Let \(p\) in \(X\) our 2-complex, let \(p'\) be the unique projection. Let \(\Gamma\) be a compact subgraph in our 2-complex \(X\).  Take \(T\) to be either \(T_{1}\) or \(T_{2}\). For now suppose we have defined a map \(f: \Gamma \to T\). We will define a map \(F: \Gamma \times I \to T\). Choose an arbitrary \(t_{0} \in f(\Gamma) \) and define: 
    \[ F(x,s) = 
        \begin{cases} 
            f(x)& s=0\\ 
            t_{0} & s=1\\
            \gamma_{f(x),t_{0}} (s | \gamma |)a & \text{else}
        \end{cases}
    \]
    Let \(k_{0} \in \Gamma \cap F^{-1} (F(x,s))\) be the nearest point to \(p'\). Put \(\gamma_{p',k_{0}}\). Then define \(g: \text{Im} (\gamma_{p',k_{0}}) \to I\) by \(g(t) = \frac{a(t)}{a(t)+b}\) where \(a(t) = d(f(t), F(x,s))\) and \(b = d(F(x,s), t_{0}\). Then we compute \(F(t,g(t))=F(x,s)\) so \(\text{Graph}(g) \subseteq F^{-1} (x,s)\). Now, \(g\) is continuous so \(\text{Graph}(g)\) is connected. Hence, \((x,s)\) is connected to \((k_{0} ,0)\) in \(F^{-1} (F(x,s))\) as needed.
\end{lem}
\begin{lem}
    [Fibers homeomorphic to Coordinate planes]



    \textcolor{error}{[warning] Changed letters, need to fix
    Put \(f = \hat f_{1} \times \hat f_{2} \times \hat f_{3}: X^{+} \to T_{1} \times T_{2} \times T_{3} \) and \(J = \text{Im}(f)\). We claim that \(J \cap T_{1} \times T_{2} \times \{z\} = \text{Im}_{f} ({\hat f_{3}}^{-1}(z))\). Let \(p = (p_{1}, p_{2} , p_{3}) \in T_{1} \times T_{2} \times T_{3}\) then we have the following. }
    \begin{align*}
        p \in \text { LHS } & \Longleftrightarrow p \in \text{Im}(f) \wedge p_{3} = z \\ 
        & \Longleftrightarrow \exists x \in X^{+} (f(x)=p \wedge \hat f_3(x)=z)\\
        & \Longleftrightarrow \exists x \in X^{+} (f(x)=p \wedge x \in \hat f_3^{-1} (z))\\
        & \Longleftrightarrow p \in \text{Im}_f(\hat f_3^{-1} (z))
    \end{align*}
\end{lem}

\subsection{Wrap up lemmas}
The following is a list of statements, the goal is to prove enough of them to arrive at item number 1 for a suitably chosen core \(C\).
\begin{enumerate}
    \item \(C \subseteq \mathscr{T}\) has simply connected hyperplanes
    \item \(C\) has hyperplanes that are (1) connected (2) quadrant convex
    \item The hyperplanes and one dimensional fibers of \(C\) are connected
    \item If \(S \subseteq \mathscr{T}\) connected in all coordinate planes, then so are \(S_{x}, S_{y},\) and \(S_{z}\).
    \item If \(R \subseteq T_{1} \times T_{2}\) has convex fibers then \( \left( R_{x} \right)_{y} = \left( R_{y} \right)_{x}\).
    \item If \(S \subseteq \mathscr{T}\) is connected in all coordinate planes then \(\sxyz\) is one dimensional fiberwise convex. 
    \item \label{state:7} If \(S \subseteq \mathscr{T}\) and is connected in all coordinate planes then \( \left( S_{x} \right)_{y} = \left( S_{y} \right)_{x}\).
    \item Guirardel's Lemma
\end{enumerate}

\subsubsection{Using Wrap-up lemmas to prove main statement}
The implications are as follows:
\begin{alignat}{3}
    (4),(7) &\fbox{\(\label{eqn:filling}\Rightarrow\)}  (6) \Rightarrow (3) \Rightarrow (2)\Rightarrow (1) \\
    (8) &\Rightarrow (4)\\
    (5) &\Rightarrow (7)
\end{alignat}
\begin{lem}
    [Reduction to vertical subpath]
    \label{lem:verticalsubpath} 
    Let \(S\) be a subset of \(T_{1} \times T_{2}\) and let \(p,q,r \in S\) satisfy: 
    \begin{enumerate}
        \item \(r \not\in S\) 
        \item \(p,q \in S\) 
        \item \(p_{2} = q_{2} = r_{2}\) 
        \item \(r_{1} \in \text{cvxhull}_{T_{1}} (\{p_{1}, q_{1} \} )\) 
    \end{enumerate}
    Let \(\sigma : [0,1] \to S\) be a path from \(p\) to \(q\), then there exists a path \(\sigma '\) such that taking \(p,q:= \sigma '(0), \sigma '(1)\) satisfies properties (2)-(4) and \(\pi_{2} (\text{Im}( \sigma '\mid_{(0,1)} ))\) is contained in an open direction of \(T_{2}\) at \(r_{2}\).
    \begin{proof}
        Consider the inverse image of \(T_{1} \times \{r_{2}\} \) under \(\sigma\). After subdividing \([0,1]\) and noting that we're taking edge paths we get that this set is a finite union of closed intervals in \([0,1]\). We also have that it contains 0 and 1. Because \(r_{1}\) is separating in \(T_{1}\) we have that each closed interval lies within an open direction in \(T_{1}\) at \(r_{1}\). Also, because \(r_{1}\) lies in the convex hull of \(p_{1}\) and \(q_{1}\) we get that the number of interals is at least 2. Consider pairs of endpoints of the closed intervals making up the closure of the complement of the inverse image. If each pair is in the same direction in \(T_{1}\) at \(r_{1}\) then because closed intervals map to horizontal directions we would have that 0 and 1 were in the same direction which contradicts our assumptions; hence there exists a subpath \(\sigma '\) where only the endpoints map into \(T_{1} \times \{r_{2}\}\). Consider \(\sigma ' \mid_{(0,1)}\), it's image lies in a product of a finite subtree of \(T_{1}\) and a disjoint union of directions of \(T_{2}\) at \(r_{2}\). Because \(\sigma '\) is continuous it must lie in a single connected component and so projects to a single verticle direction of \(T_{2}\) at \(r_{2}\).

    \end{proof}
\end{lem}
\begin{lem}
    [Statement 4]
    Let \(S \subseteq \mathscr{T}\) be a subcomplex that is connected in all coordinate planes. Then \(S_{x}, S_{y},\) and \(S_{z}\) are as well. 
    \begin{proof}
        Without loss of generality, consider \(S_{ x}\), note that \(S_{x}\) will be connected in all \(xy\) and \(xz\) planes because \(S\) was. Consider the \(yz\)-planes in \(S_{x}\), if there were no new points added then the planes are connected and we are done. Suppose that \( p \in (S_{x} \smallsetminus S ) \cap \pi^{-1} _{1} (p_{1} )\), we will show that there is a path in \(S_{x} \cap \pi_{2}^{-1} (p_{2} )\) between \(p\) and some point in \(S\).
       
        Since \(p\) isn't in \(S\) there exist distinct points \(r\) and \(s\) in \(S \) that agree in all coordinates except the first where we have that \(p_{1} \in \text{cvxhull}_{T_{1}}  (\{r_{1} , s_{1}\})\). Now, because \(S\) is connected in all coordinate planes there is a path \(\sigma\) from \(r\) to \(s\) that lies in \(S \cap \pi_{2}^{-1} (p_{2})\). In fact, we can take \(\sigma\) to be a path that begins and ends at \(p\) and \(r\) but otherwise has \(T_{3}\) coordinates lying in exactly one direction of \(T_{3}\) at \(p_{3}\). We have factored out this situation into claim \ref{lem:verticalsubpath}.  

        Take \(\sigma\) as in the claim \ref{lem:verticalsubpath}. It remains to show that there is a path connecting \(p\) to another point in \(S\). Considering closed quadrants, there exists a sequence \(t_{1},\ldots,t_{n}\) such that \(v_i:=\sigma (t_i)\) are vertices in \(p_{1} \times p_{2} \times \overline\delta\)  (here \(\delta\) is the distinguished \(T_{3}\)  direction) with the property that \(\sigma (t_{1}) =p\), \(\sigma (t_{n}) = q\), and \(\sigma_{i} := \sigma \mid_{[t_{i} , t_{i + 1} ]}\) lie in quadrant \(i\) for \(0 < i < n\). (Here, quadrant \(i\) is determined by specifying a direction in \(T_{1}\) at \(p_{1}\) since we've already chosen \(\delta\) above.) Projecting, we obtain a sequence of subpaths in \(\overline\delta\) a direction in \(T_{3}\) that are colored based on their quadrant. Note that \(\overline \delta\) has a valence one vertex at \(p_{3}\) because our tree is simplicial and subpaths don't end at \(p_{3}\) unless it's the first or last because the claim has that the interior of the path maps into \(\delta\). This situation satisfies the conditions for our coloring lemma \ref{lem:coloring}.
        
        Multicolored paths in this vertical direction \(\delta\)  give paths contained in \(S_{x}\), applying the coloring lemma \ref{lem:coloring} gives the required multicolored path. This completes the proof.
    \end{proof}
\end{lem}

\begin{lem}
    [Statement 5]
    If \(R \subseteq T_{1} \times T_{2}\) is a connected subcomplex then \( \left( R_{x} \right)_{y} = \left( R_{y} \right)_{x}\).
    \begin{proof}
        We will first show that \(\rxy\) has connected fibers. Note, it already has connected \(y\)-fibers, so it remains to show that it has connected \(x\)-fibers. Suppose that \(\rxy \cap ( T_{1} \times \{y_{0}\} )\) were a disconnected \(x\)-fiber. Then because \(\rxy\) is a subcomplex we have that the fiber is separated by some edge. Let \(x_{0}\) be the midpoint of this edge. Let \(\ell\) and \(r\) denote the left and right closed halfspaces of \(T_{1} \times \{y_{0}\}\) at the midpoint \(x_{0}\). We will show that either \(\ell \cap \rxy = \varnothing\) or \(r \cap \rxy = \varnothing\), applying this to all such \(x_{0}\) will show \(\rxy\) has connected \(x\)-fibers.
        
        Suppose this were false, and that \(\ell \cap \rxy \neq \varnothing\) and \(r \cap \rxy \neq \varnothing\). We consider three cases: (1) \(\ell\) and \(r\) intersect \(R_{x}\) nontrivially (2) \(\ell\) and \(r\) don't intersect \(R_{x}\), and (3) exactly one of \(\ell\) or \(r\) intersects \(R_{x}\) nontrivially.
        \begin{enumerate}
            \item Case 1: If both \(\ell\) and \(r\) intersect \(R_{x}\) then both contain points of \(R\) because we filled in the \(x\)-fiber, but then \(x_{0} \times y_{0}  \in R_{x}\) a contradiction.
            \item Case 2: Pick a point \(x^+\) in \(r \cap \rxy \smallsetminus R_{x} \). Let \(q^{+}\) and \(q^-\) be points above and below \(x^+\) in \(R_{x}\). These points are either in \(R\) already, or because we filled in the \(x\)-fiber there exist points above and below the line \(\ell \cup r\) but this is a contradiction since \(R\) is connected.
            \item Case 3: Without loss of generality, suppose \(\ell \cap R_{x} = \varnothing\) and \(r \cap R_{x} \neq \varnothing\). This implies there is some \(s \in R \cap r\), we will get a contradiction by separating this point from another point in \(R\). 
                
                Let \(u\) and \(d\) be directions of \(x_{0} \times T_{2}\) at \(x_{0} \times y_{0}\). If both intersected \(R_{x}\) then \(x_{0} \times y_{0} \in R \) a contradiction so without loss of generality, suppose \(d \cap R_{x} \neq \varnothing \). Now let \(x^-\) be a point in \(\ell\), by assumption \(x^- \in \rxy \smallsetminus R_{x} \). This gives a point \(q^-\) below \(\ell\) in \(R_{x}\) which is already in \(R\) or there exists a point \(t^- \in R\) to the left of \(q^-\); but this is a contradiction since \(\ell \cup d\) separates these points from \(s \in R\).
        \end{enumerate}

                Hence, either \(\ell \cap \rxy = \varnothing\) or \(r \cap \rxy = \varnothing\) as needed, so \(\rxy\) has connected \(x\)-fibers. Similarly, \(\ryx\) also has connected fibers. Hence by \ref{lem:guirardel} their complements are unions of quadrants and so they contain \(QH(R)\). It remains to show that they are contained within \(QH(R)\). 

                \begin{claim*}
                    \(\rxy \subseteq QH(R)\) 
                    \begin{proof}
                        (sketch) The idea is to show that for every point \(p \in \rxy\) that for every quadrant \(Q\) containing \(p\) (i.e. that would be attempting to remove it) we can find a point \(r \in R \cap Q\). Picking an open quadrant containing \(p\) amounts to picking a point \(q = (q_{1}, q_2)\) with halfspaces at each coordinate that contain the corresponding \(p_{i}\). The hard case is where \(p \in \rxy \smallsetminus R_x\) - so you pick a point in a vertical direction at \(p\) pointing towards \(p\), because this point is only there due to filling there's another vertical direction that you're grabbing that contains some \(R_x\). Then you pick a horizintal place and point towards \(p_1\), this must contain at least one of the directions with \(R\) in it.
                    \end{proof}

                    \begin{proof}
                        We will show for every point \(p \in \rxy\) and every open quadrant \(Q\) with \(p \in Q\) that there exists some \(r \in R \cap Q\). Let \(q=(q_1,q_2)\) be the point where \(Q\) is based and label the halfspaces so that \(Q=q_1^{+} \times q_{2}^{+}\).

                        Case 1 \(p \in R_x \smallsetminus R\): Let \(p \in R_x \smallsetminus R\): Let \(\ell = (\ell_1,p_2)\) and \(u=(u_1,p_2)\) be points in \(R\) that cause the vertical filling. Now, \(q_1^+\) contains all but one direction at \(p_1\) and so must contain either \(\ell_1\) or \(u_1\). Since \(q_2^+\) must contain \(p_2\) we get that \(Q=q_1^+ \times q_2^+\) contains a point of \(R\).


                        Case 2 \(p \in \rxy \smallsetminus R_x\): Let \(u=(p_1,u_2)\) and \(d=(p_1,d_2)\) be points that cause the vertical filling. Now, \(q_2^+\) contains all but one direction at \(p_2\) and so must contain either \(u_2\) or \(d_2\). Without loss of generality, suppose it contains \(u_2\). If \(u \in R\) then we are done. Suppose \(u \in R_x \smallsetminus R\). Then we can find \(u'=(p',u_2)\) and \(u''=(p'',u_2)\) with \(p'\) and \(p''\) in different directions at \(p_1\). Now, \(q_1^+\) contains all but one direction at \(p_1\) and so must contain either \(p'\) or \(p''\). Suppose without loss of generality that it contains \(p'\), then \(Q\) contains \(p' \times u_2 \in R\).
                    \end{proof}
                \end{claim*}


    \end{proof}

\end{lem}

\begin{lem}
    [Boxed Implication]
 \label{lem:boximp} 
    Repeatedly applying statement \ref{state:7} to both \(S\) and \(S_{\alpha}\) where \(\alpha\) is one of \(x,y,\) or \(z\)  and noticing that \(\langle (12),(23) \rangle = S_{3}\) we get that \(\sxyz\) is equal to any of the permutations of the indices. In particular, \(\Sxyz{x}{y}{z}=\Sxyz{y}{z}{x}=\Sxyz{z}{x}{y}\) which shows that \(\sxyz\) has connected one dimensional fibers.
\end{lem}
\subsection{Planar Path Argument}

 \label{lem:ppa} 
\begin{lem}[Coloring Lemma]
    \label{lem:coloring} 
    Let \(K\) satisfy property setup. Let \(A\) be the set of endpoints minus the basepoint \(x_{ 0}\). Then there exists some \(a \in A\) such that for all edges \(e\) in the geodesic \([x_{0} , a]\) there exist some \(i, j\) such that \(\sigma_{i}\) and \(\sigma_{j}\) use \(e\) and \(\lambda ( i ) \neq \lambda ( j )\).
    \begin{proof}
        Suppose the lemma were false. Then there exist counterexamples \(K\) satisfying property setup such that for all \(a \in A\) there exists an edge \(e\) on \([x_{0} , a]\) such that \(\lambda (i)= \lambda (j)\) whenever \(\sigma_{i}\) and \(\sigma_{j}\) use \(e\). These \(K\) have bad edges and \(| \sigma | > 0 \). Now take a \(K\) such that \(| \sigma |\) is minimized. Let \(K'\) be the result of snipping a bad edge. By lemma \ref{lem:preservesetup}, \(K'\) has property setup so \(| \sigma ' | > 0\). If \(K'\) contained a multicolored path between \(x_{0}\) and some \(a'\) in \(A'\) then because the path \(\sigma\) in \(K\) an be obtained by inserting subpaths into \(\sigma '\), we have that \(A' \subseteq A\) and any edge that was multicolored stays multicolored. Therefore \(K'\) has no multicolored path, because that would force \(K\) to have one. Then the fact \(| \sigma' | < | \sigma |\) gives a contradiction because \(K\) was chosen to be minimal. 
        
        
    
        
    \end{proof}
\end{lem}


\begin{lem}
    [Snip invariant]
 \label{lem:preservesetup} 
    If \(K\) satisfies property setup then \(K'\) obtained from snipping a bad edge also satisfies property setup.
    \begin{proof}
        Let \(e\) be a bad edge of \(K\), since \(\sigma_{1}\) and \(\sigma_{N}\) have different colors the edge \(e\) in at most one of them. Because \(x_{0}\) has valence one, \(\sigma_{1}\) and \(\sigma_{N}\) share the edge containing \(x_{0}\); therefore \(e\) cannot be the common edge. The rest follows.
    \end{proof}
\end{lem}

\end{document}

#justVimThings
select last search: //, go to end of search upon finding: //e
move to beginning/end of visual selection: o,O
surround a selection of text: v to select some text, press "S", then a delimiter
	- works with (, {, [ and <p>, <body>, etc.
delete cursor to beginning of word: d/<type word here>
	- works on multiple lines
move down by display lines: prefix with g e.g. gj, gk, g0, g$
start search backwards via ?



