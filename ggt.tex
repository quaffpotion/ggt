\documentclass{article}
\usepackage{amsmath,amssymb,amsthm, fancyhdr, tikz-cd, xcolor, mathrsfs,hyperref,cleveref}
\definecolor{mypink1}{rgb}{0.858, 0.188, 0.478}
\swapnumbers

\pagestyle{fancy}
\renewcommand{\sectionmark}[1]{\markright{\thesubsection\ #1}}
\fancyhf{}
\lhead{Research Outline}
\rhead{Tony Martino}

\newtheoremstyle{mystyle}
  {\topsep}
  {\topsep}
  {}
  {}
  {\scshape}
  {.}
  {.5em}
  {}

\newcommand{\sxyz}{((S_x)_y)_z}
\newcommand{\rxy}{(R_x)_y}

\theoremstyle{mystyle}
\newtheorem{thm}{Theorem}[section]
\newtheorem{thm*}{Theorem}
\newtheorem{lem}{Lemma}[section]
\newtheorem{pro}{Proposition}
\newtheorem{defn}{Definition}
\newtheorem*{defn*}{Definition}
\newtheorem*{claim*}{Claim}
\newtheorem*{lem*}{Lemma}
\newtheorem*{cor*}{Corollary}


\theoremstyle{remark}
\newtheorem{rmk}{Remark}[section]
\newtheorem{ex}{Example}[section]
\newtheorem{nex}{Not Example}[section]



\begin{document}

\section{Problem Statement}


We want to show the following statement or similar: if a group is the fundamental group of a VH-complex, then there are at most two pairwise inequivalent actions on locally finite trees with FP vertex stabilizers. The following proof sketch uses or builds on VH-complexes introduced by Wise, Guirardel's core, and a theorem of Bieri. To start we have that \(G = \pi_1(K)\) where \(K\) is a VH-complex; according to Wise this comes with a vertical and horizontal splitting. For sake of a contradiction suppose there was a third tree as above. A generalization of Guirardel's core to three actions would imply that our \(G\) was the fundamental group of a VHD-complex (here ``D'' is for ``depth'') which would further imply, among other things, that our group splits along groups of cohomological dimension two. (This is analagous to Wise's work on VH-complexes) However, a theorem of Bieri and local finiteness forces the cohomological dimension of the resulting group to be dimension three which contradicts our original VH assumption so there are at most two such actions.

\subsection{Definitions}

\begin{defn}
	[Type \(FP\) ]
	A group is of type \(FP\) if it is (1) type \(FP_n\) for all \(n\) and (2) finite geometric cohomological dimension.
\end{defn}

\begin{defn}
	[Finite Type] 
	An action of {\em finite type} is one on a locally finite tree where  vertex stabilizers are of type FP.
\end{defn}

\begin{defn}[Open Direction] An open direction is a connected component of an \(\mathbb{R}\)-tree minus a point. 
\end{defn}

\begin{defn}[Closed Direction] A closed directon is a connected component of an \(\mathbb{R}\)-tree minus a point, union that point.
\end{defn}
\begin{defn}[Open Halfspace] An open halfspace is an open direction obtained from deleting the midpoint of an edge.
\end{defn}
\begin{defn}[Closed Halfspace] A closed halfspace is a closed direction obtained from deleting the midpoint of an edge.
\end{defn}
\begin{defn}[Halfspaces of a product] An open (resp. closed) halfspace of a product (at a certain index) is a subset where exactly one projection is an open (resp. closed)  halfspace in it's factor and the others are onto.
\end{defn}
\begin{defn}[Generalized quadrants] A generalized open (resp. closed) quadrant with respect to a product of \(k\) spaces is an intersection of \(k\) open (resp. closed) product halfspaces where each one is at a different index.
\end{defn}
\begin{defn}[cellular-product-convex] We say that \(K \subset X\) is cellular-product-convex if it's complement is the open cellular neighborhood of a union of generalized closed quadrants.
\end{defn}

\begin{defn}
	[Filling]
    Let \(\{X_{k}\}_{k \in K}\) be a family of spaces where one can take convex hulls. Given \(S \subseteq X := \prod X_{k}\) define \(S_{k}\) for \(k \in K\) via: \[p \in S_{k} \iff \exists \,q,r \in S: \forall j \neq k: p_{j} = q_{j} = r_{j} \text{ and } p_{k} \in \text{cvxhull}_k (\{q_{k} , r_{k}\}).\] 
\end{defn}


\section{Outline and notation for the construction of the core}

    From Guirardel we get a map \(f\) from the universal cover of our VH complex to the product of three trees by taking inclusion in the first two factors and Guirardel's map in the last factor. We obtain \(S = \text{Im}(f)\) and will show it's cocompact. Then we put \(K = \text{cell}{(S)}\) and show it's still cocompact. Finally we fill \(K\) in all three directions obtaining the core \(C\). The main goal is to show that \(C\) is a cocompact cube complex with quadrant-convex hyperplanes so that we get a legitimate graph of spaces decomposition for \(G\).

\section{Proving core is hyperplane Quadrant Convex}
One path is to show that \(C\) is 1-dimensional fiberwise connected and then apply Guirardel's lemma in each hyperplane to conclude that they're QC as needed.

\begin{lem}
	[Guirardel Lemma 5.4, Corollary 5.5]
	Let \(T_{1} , T_{2}\) be two \(\mathbb{R}\)-trees and let \(F\) be a nonempty connected subset of \(T_{1} \times T_{2}\) with convex fibers. Then the complement of \(\overline{F}\) is a union of quadrants. That is, \(\overline{F}\) is also nonempty, connected, and has convex fibers.
\end{lem}
\subsection{Planar Path Argument}
\begin{lem}[Coloring Lemma]%{{{
	Let \(T\) be a finite combinatorial tree with a valence one vertex \(x_{0}\). Let \((\sigma_{i} )_{i=1}^{N}\) be a sequence of non-degenerate edgepaths and let \(p = *_{i=1}^{N} \sigma_{i}\) be a loop based at \(x_{0}\). Let \(\lambda\) be a function from \(\{1,\ldots,N\}\) to a non-empty finite set such that \(\lambda (1) \neq \lambda (N)\). Suppose that \(x_{0} \in \{\delta_{0} \sigma_{i} , \delta_{1} \sigma_{i} \} \implies i \in \{1,N\}\). Let \(A\) be the set of endpoints minus the basepoint \(x_{ 0}\). Then there exists some \(a \in A\) such that for all edges \(e\) in the geodesic \([x_{0} , a]\) there exist some \(i, j\) such that \(\sigma_{i}\) and \(\sigma_{j}\) intersect \(\{e,\overline{e}\}\) and \(\lambda ( i ) \neq \lambda ( j )\).
\end{lem}%}}}

\subsubsection{Definitions for combinatorial lemma}%{{{
\begin{defn}
	[straddle]
	A path \(p\) in a tree \(T\) {\em straddles} an edge \(e\) if
	\begin{enumerate}
		\item There exists \(k\) such that \(p(k) \in  \{e, \overline{e}\}\). (i.e. the path uses the geometric edge \(\{e , \overline{e}\}\) corresponding to  \(e\))
		\item Both halfspaces of \(e\) contain an endpoint of \(p\).
	\end{enumerate}
\end{defn}
\begin{defn}
	[complexity]
	Given \(J=(T, \sigma_{1} \cdots \sigma_{N} , \{e_{1} , \ldots, e_{m}\})\) we define the {\em complexity} of \(J\) to be the sum of the number of times each edge is straddled.
\end{defn}
\begin{defn}
	[path snipping]
	Let \(p\) be a path and let \(e=p(i), f=p(j)\) be edges such that \(\partial_{0}  e = \partial_{1} f \). Then \(p'\) is the restriction of \(p\) to \(\{k\mid 1 \leq k < i\} \cup \{k \mid j < k < \infty\}\). If \(p\) has a subpath structure then the subpath structure for \(p'\) is given by restricting each subpath as well. (Any resulting subpaths that are empty may or may not be deleted based on context)
	
\end{defn}
\begin{defn}
	[outer]
	Let \(T\) be a tree, \(x \in T\), and \(A\) be a subset of edges. Let \(e^{+}\) denote the halfspace of \(e\) that doesn't contain \(x\). An edge \(e\) of \(T\) is {\em outer} with respect to \((x,A)\) if \(e^{+} \cap A=\varnothing\).  
\end{defn}
\begin{lem}
	If there exists \(K=(T,x_{0} ,\sigma_{1} \cdots \sigma_{N} , \{ e_{ 1} ,\ldots, e_{m}  \}, \lambda )\) with \(m > 1\) satisfying the below properties with complexity \(c > 2\) then there exists \(K'\) also satisfying those properties with \(c' = c - 2\).
	\begin{enumerate}
		\item \(\lambda_{1} \neq \lambda_{N} \) 
		\item The first edge of \(\sigma_{1}\) is dual to the last edge of \(\sigma_{N}\).
		\item \(\forall 1 \leq i \leq N \exists k\) \(e_{i}\) is {\em straddled} by \(\sigma_k\).
		\item 	 The coloring \(\lambda\) is constant on the set of all subpaths using some element of the geometric edge corresponding to \(e_{k}\).
		\item 	 The set \(\{e_{1},\ldots, e_{m}\}\) separates \(x_{0}\) from \(A\). (as geometric edges e.g. \(\{e, \overline{e}\}\))

		\item If the endpoints of \(\sigma_{k}\) contain \(x_{0}\) then \(k\) is either 1 or \(N\).
	\end{enumerate}
\end{lem}

\begin{lem}
	Let \(K\) satisfy the above properties, then \(c \neq 2\).
\end{lem}

\begin{lem}
	Given the SETUP (taking T to be the middle factor etc.) we obtain \(K\) as in above ... (list the properties we get from setup)
\end{lem}


\begin{cor*}
	Let \(K > 0\) be as in SETUP. Let \(K'\) be the result of performing an outer snip on \(K\). Then \(K' > 0\).
	\begin{proof}
        (Here we only need to show that \(K > K'\) and then apply the edges lemma)
		
	\end{proof}
\end{cor*}
\begin{lem}
	[Multipath]
	 \label{lem:multi} 
	Let \(K\) be as in SETUP. Let \(K'\) be the result of performing an outer snip on \(K\). Let \(p\) be a path in \(K'\) between \(x_{0}\) and some \(a \in A'\) that is {\em multicolored} relative to \(\lambda'\). Then \(p\) is multicolored relative to \(\lambda\) in \(K\) and goes between \(x_{0}\) and some \(a \in A\).
	\begin{proof}
		Check that if an edge is monocolored in \(K\) then it is monocolored in \(K'\) as well. Given this, the multicolored path \(p\) in \(K'\) cannot use \(e\) and so is entirely contained in \(e^{+}\). However, because the snip only affects the coloring of edges outside \(e^{+}\) and doesn't change the set of endpoints in \(e^{+}\) we have that the multicolored path exists in \(K\) between \(x_{0}\) and \(a:=a' \in A\).
	\end{proof}
\end{lem}
\begin{lem}
	[Edges]
	 \label{edges} 
	  Let \(K=0\) be as in SETUP. Then \(K\) has a multicolored path between \(x_{0}\) and some \(a \in A\).
\end{lem}%}}}

\subsection{Wrap up lemmas}
The following is a list of statements, the goal is to prove enough of them to arrive at item number 1 for a suitably chosen core \(C\).
\begin{enumerate}
    \item \(C \subseteq \mathscr{T}\) has simply connected hyperplanes
    \item \(C\) has hyperplanes that are (1) connected (2) quadrant convex
    \item The hyperplanes and one dimensional fibers of \(C\) are connected
    \item If \(S \subseteq \mathscr{T}\) connected in all coordinate planes, then so are \(S_{x}, S_{y},\) and \(S_{z}\).
    \item If \(R \subseteq T_{1} \times T_{2}\) is connected in all \(xy\)-planes then \( \left( R_{x} \right)_{y} = \left( R_{y} \right)_{x}\).
    \item If \(S \subseteq \mathscr{T}\) is connected and connected in all coordinate planes then \(\sxyz\) is one dimensional fiberwise convex. 
    \item If \(S \subseteq \mathscr{T}\) and is connected in all coordinate planes then \( \left( S_{x} \right)_{y} = \left( S_{y} \right)_{x}\).
    \item Planar Path Argument
\end{enumerate}

\begin{lem}
    [Planar Path Argument]
    Given SETUP, let \(S\) be a non-empty closed subset of \(T_{1} \times T_{2}\) and suppose \(p \not\in S_{x}\smallsetminus S\), then there exists \(q \in S \cap \left( \{p_{1} \} \times T_{2} \right)\) such that \([p,q] \subseteq \{p_{1} \} \times T_{2}\).
    \begin{proof}
        Given SETUP, we apply the combinatorial lemma.
    \end{proof}
\end{lem}

\subsubsection{Using Wrap-up lemmas to prove main statement}
The implications are as follows:
\begin{alignat}{3}
    (4),(7) &\fbox{\(\Rightarrow\)}  (6) \Rightarrow (3) \Rightarrow (2)\Rightarrow (1)\\
    (8) &\Rightarrow (4)\\
    (5) &\Rightarrow (7)
\end{alignat}

\begin{lem}
    [Statement 4]
    Let \(S \subseteq \mathscr{T}\) be a subcomplex that is connected in all coordinate planes. Then \(S_{x}, S_{y},\) and \(S_{z}\) are as well. 
\end{lem}
\end{document}


words to search for: clarification, question, technicality, concern

#justVimThings
surround a selection of text: v to select some text, press "S", then a delimiter
	- works with (, {, [ and <p>, <body>, etc.
delete cursor to beginning of word: d/<type word here>
	- works on multiple lines
move down by display lines: prefix with g e.g. gj, gk, g0, g$
start search backwards via ?
workaround for 


