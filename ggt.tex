\documentclass{article}
\usepackage{amsmath,amssymb,amsthm, fancyhdr, tikz-cd, xcolor, mathrsfs,hyperref,cleveref}
\definecolor{mypink1}{rgb}{0.858, 0.188, 0.478}
\swapnumbers

\pagestyle{fancy}
\renewcommand{\sectionmark}[1]{\markright{\thesubsection\ #1}}
\fancyhf{}
\lhead{Research Outline}
\rhead{Tony Martino}

\newtheoremstyle{mystyle}
  {\topsep}
  {\topsep}
  {}
  {}
  {\scshape}
  {.}
  {.5em}
  {}


\theoremstyle{mystyle}
\newtheorem{thm}{Theorem}[section]
\newtheorem{thm*}{Theorem}
\newtheorem{lem}{Lemma}[section]
\newtheorem{pro}{Proposition}
\newtheorem{defn}{Definition}
\newtheorem*{defn*}{Definition}
\newtheorem*{claim*}{Claim}
\newtheorem*{lem*}{Lemma}

\theoremstyle{remark}
\newtheorem{rmk}{Remark}[section]
\newtheorem{ex}{Example}[section]
\newtheorem{nex}{Not Example}[section]



\begin{document}
\section{Background}

\begin{enumerate}
    \item Group Splittings
    
        Classical combinatorial group theory gives an algebraic notion of gluing groups along subgroups. Using suitable spaces and Siefert-van Kampen's theorem one can interpret the gluing as taking the quotient of a topological space. In the definitions below, amalgamation and HNN extension from the point of view of topology correspond to the disconnected and connected case. (e.g. An HNN extension along an automorphism is the fundamental group of a mapping torus) For a different kind of intuition note that an amalgamated free product is the pushout in the category of groups.
        
        Given \(A\leq G\) and \(B\leq H\) and an isomorphism \(\phi:A\to B\) and another isomorphism \(\psi: C\to C\) where \(C\leq G\) we can form the {\em amalgamated free product of \(G\) and \(H\) along \(\phi\)} denoted \(A*_\phi B\) with presentation \(\langle G, H\mid \phi(a) = b, a\in A\rangle\) and the {\em HNN extension of \(G\) along \(\psi\)} denoted \(G*_\psi\) with presentation \(\langle G, t\mid \psi(c)=tct^{-1}, c\in C\rangle\). 
        
        We will use results from Bass-Serre theory which gives a correspondence between splittings and \(G\)-trees. Roughly speaking, the groups in the amalgamation or HNN extension appear as vertex and edge stabilizers and vice versa.
    \item Daniel Wise VH-complexes
    
        In his thesis Wise considered {\em VH-complexes}, these are square complexes with the edge set split into two disjoint sets such that attaching maps alternate between them. Among other items, Wise showed that from a VH-complex you can extract two splittings (vertical and horizontal) of it's fundamental group.
    \item Guirardel Core
    
        A particularly nice example comes from interpreting the algebraic definition of a group splitting topologically. First, specialize to the case of cyclic splittings of a fixed surface group. In this special case splittings correspond to homotopy classes of simply closed curves on a surface. The splittings further correspond to actions on trees and taking the diagonal action we get an action on a product of trees. Guirardel's work applies to more general \(\mathbb{R}\)-trees but in this case the {\em Guirardel core} is a contractible cocompact square complex - in fact, the quotient recovers the surface as a square complex with a number of squares equal to the intersection number of the curves.
\end{enumerate}

\section{Problem Statement}

\subsection{VH implies at most two actions:}

We want to show the following statement or similar: if a group is the fundamental group of a VH-complex, then there are at most two pairwise inequivalent actions on locally finite trees with FP vertex stabilizers. The following proof sketch uses or builds on VH-complexes introduced by Wise, Guirardel's core, and a theorem of Bieri. To start we have that \(G = \pi_1(K)\) where \(K\) is a VH-complex; according to Wise this comes with a vertical and horizontal splitting. For sake of a contradiction suppose there was a third tree as above. A generalization of Guirardel's core to three actions would imply that our \(G\) was the fundamental group of a VHD-complex (here ``D'' is for ``depth'') which would further imply, among other things, that our group splits along groups of cohomological dimension two. (This is analagous to Wise's work on VH-complexes) However, a theorem of Bieri and local finiteness forces the cohomological dimension of the resulting group to be dimension three which contradicts our original VH assumption so there are at most two such actions.


\section{Timeline}

The precise details of the lemmas and definitions still need to be worked out in full. (e.g. crafting a version of Guirardel's core for three instead of two trees requires work) That said, I plan to begin writing my thesis in the Summer with the intent of graduating in the Spring of 2020.

\section{Career}
I've decided that the most important factor in my next career opportunity is location. Given this, I've been focusing on industry and government jobs. I have a resume and a GitHub account with examples of my work. I've also talked to  friends and family - including some OU gradautes - who work in areas I would enjoy.

\section{Progress}

\subsection{Definitions}

\begin{defn}[Directions] A {\em direction} is a connected component of an \(\mathbb{R}\)-tree minus a point. 
\end{defn}
\begin{defn}[Halfspaces] A halfspace is a direction obtained from deleting the midpoint of an edge.
\end{defn}
\begin{defn}[closed Halfspaces] A component of tree minus a point, union the point.
\end{defn}
\begin{defn}[Halfspaces of a product] A halfspace of a product is a subset where one projection is a halfspace in it's factor and the others are identity
\end{defn}
\begin{defn}[Generalized open quadrants] A generalized quadrant with respect to a product of \(k\) spaces is an intersection of \(k\) product halfspaces where each one lies in a separate factor 
\end{defn}
\begin{defn}
    [Generalized closed quadrant]
\end{defn}
\begin{defn}
    [Guirardel Quadrant]
    Product of two directions. (This coincides with our definition of generalized-open-quadrants where \(k=2\) remembering to allow products of directions not just halfspaces)
\end{defn}
\begin{defn}[product-convex] We say that \(S\subset X\) is product-convex if it's complement is the union of generalized closed Guirardel quadrants.
\end{defn}
\begin{defn}[cellular-product-convex] We say that \(K \subset X\) is cellular-product-convex if it's complement is the open cellular neighborhood of a union of generalized closed quadrants.
\end{defn}

\subsection{Statements}

\begin{lem}
    [Filling Lemma in \(\mathbb{R}^{2}\) with Guirardel-qudrants]
\end{lem}
\begin{proof}
    (sketch) This one is miai with a line that's left out
\end{proof}
\begin{lem}
    [Filling Lemma in \(T_{1} \times T_{2}\) with Guirardel-quadrants]
\end{lem}
\begin{proof}
    (sketch) find replace the proof in \(\mathbb{R}^{2}\) with corresponding words for trees e.g. connecting becomes convex hull, left becomes inside a half space etc.
\end{proof}
\begin{lem}
    [Filling Lemma in \(\mathbb{R}^{3}\)]
\end{lem}
\begin{proof}
    (sketch) Need to use the Guirardel lemma to get two ways of writing the quadrant-convex hull of a set. This allows switching. Then assuming things are connected in all planes get that one filling is still connected and so you can apply the switching again. (There is a planar path argument to make) Then you get the result.
\end{proof}
\begin{lem}
    [Filling Lemma in \(T_{1} \times T_{2} \times T_{3}\)]
    (Here we're taking convex hulls in the tree factors)
\begin{proof}
    (todo)
\end{proof}
\end{lem}

\subsubsection{Transition from old-convex to new-convex}

In \(\mathbb{R}^{3} \) no point is a branch point so we can view every point as a midpoint hence we can separate. The idea is the separation is kept track of in a cube and this cube appears in \(X\) as well. 

In the below use the following notation:
\begin{itemize}
	\item \(X=T_{1} \times T_{2} \times T_{3}\)
	\item \(K \subseteq X\) always as subcomplex
\end{itemize}

\begin{lem}
    [retract separation]
    You can pull back a separation of a retract to a separation of the whole space.
\end{lem}
\begin{lem}
	[\label{lem:midpointsep}Separating at midpoints implies cell-prod-cvx for subcomplexes]
	If \(K \subseteq X \) has the property that for every midpoint disjoint from \(K\) we can find a generalized-closed-quadrant containing the midpoint that's disjoint from \(K\) then \(K\) is cellular-product-convex.
\end{lem}

\subsubsection{Reduction to cube}

Some observations:
\begin{itemize}
	\item We care about subcomplexes so removing a midpoint removes an entire cube - this has three consequences:
		\begin{enumerate}
			\item The subcomplex will retract to a cube, in fact the image will lie on the boundary
			\item No need to worry about ``plucking out a point'' to make the retraction work, no points of our set lie ``inside'' the octahedron
			\item only need to worry about what hyperplanes are doing as opposed to general fibers
		\end{enumerate}
	\item Keep the square picture in mind when doing the retraction - octants retract to vertices, etc. Things contract in sheets if you like.
\end{itemize}

\subsection{Main Goals}
\begin{thm}
	\label{thm:QCOC}
    [Slices QC implies OC for trees]
	Let \(X=T_{1} \times T_{2} \times T_{3}\) where each \(T_{i}\) is a locally finite tree. Let \(K\) be a subcomplex of \(X\) that satisfies: 
	\begin{itemize}
		\item connected
		\item has connected hyperplanes
		\item has cellular-product-convex hyperplanes 
	\end{itemize}
Then \(K\) is also cellular-product-convex.
\end{thm}
We need four lemmas, the last two give the implication after taking intersections:

\begin{lem}
	\label{lem:cubeOC}
	[Cube OC] Let \(S \subseteq \mathbb{E}  \) be a closed set such that \(S\) satisfies: 
	\begin{enumerate}
		\item \((0,0,0) \not\in S\) 
		\item \(S\) connected
		\item \(S \cap (\text{coordinate-plane})\) is connected.
		\item \(S \cap (\text{coordinate-plane})\) is disjoint from some closed quadrant in that plane.
	\end{enumerate}
	Then \(S\) is disjoint from some closed octant of \(\mathbb{E}\). 
	\begin{proof}
		(todo, should follow from notes on the \(\mathbb{R}^{3}\) case.)
	\end{proof}
\end{lem}

\begin{lem}
	[\label{lem:Xtocube}X to cube]
	If \(S\) satisfies the hypotheses of \ref{thm:QCOC} then \(\pi S\) satisfies the hypotheses of \ref{lem:cubeOC} provided \(\pi\) is a projection map to a cube determined by a midpoint disjoint from \(S\).
\begin{proof}
	(todo)	
\end{proof}
\end{lem}

\begin{lem}
	[Connecting logic]
	If \(S\) satisfies the conclusion of lemma \ref{lem:cubeOC}, then the inverse image of \(S\) under \(\pi\) satisfies the conclusion of theorem \ref{thm:QCOC}.
\end{lem}

\end{document}

#justVimThings
surround a selection of text: v to select some text, press "S", then a delimiter
	- works with (, {, [ and <p>, <body>, etc.
delete cursor to beginning of word: d/<type word here>
	- works on multiple lines
move down by display lines: prefix with g e.g. gj, gk, g0, g$
