\documentclass{article}
\usepackage{amsmath,amssymb,amsthm, fancyhdr, tikz-cd, xcolor, mathrsfs,hyperref,cleveref}
\definecolor{mypink1}{rgb}{0.858, 0.188, 0.478}
\swapnumbers

\pagestyle{fancy}
\renewcommand{\sectionmark}[1]{\markright{\thesubsection\ #1}}
\fancyhf{}
\lhead{Research Outline}
\rhead{Tony Martino}

%Commands
\newtheoremstyle{mystyle}
  {\topsep}
  {\topsep}
  {}
  {}
  {\scshape}
  {.}
  {.5em}
  {}

\newcommand{\sxyz}{((S_x)_y)_z}
\newcommand{\rxy}{(R_x)_y}
\newcommand{\ryx}{(R_y)_x}
\newcommand{\Sxyz}[3]{((S_#1)_#2)_#3}

\theoremstyle{mystyle}
\newtheorem{thm}{Theorem}[section]
\newtheorem{thm*}{Theorem}
\newtheorem{lem}{Lemma}[section]
\newtheorem{pro}{Proposition}
\newtheorem{defn}{Definition}
\newtheorem*{defn*}{Definition}
\newtheorem*{claim*}{Claim}
\newtheorem*{lem*}{Lemma}
\newtheorem*{cor*}{Corollary}


\theoremstyle{remark}
\newtheorem{rmk}{Remark}[section]
\newtheorem{ex}{Example}[section]
\newtheorem{nex}{Not Example}[section]



\begin{document}

\section{Problem Statement}
We want to show the following statement or similar: if a group is the fundamental group of a VH-complex, then there are at most two pairwise inequivalent actions on locally finite trees with FP vertex stabilizers. The following proof sketch uses or builds on VH-complexes introduced by Wise, Guirardel's core, and a theorem of Bieri. To start we have that \(G = \pi_1(K)\) where \(K\) is a VH-complex; according to Wise this comes with a vertical and horizontal splitting. For sake of a contradiction suppose there was a third tree as above. A generalization of Guirardel's core to three actions would imply that our \(G\) was the fundamental group of a VHD-complex (here ``D'' is for ``depth'') which would further imply, among other things, that our group splits along groups of cohomological dimension two. (This is analagous to Wise's work on VH-complexes) However, a theorem of Bieri and local finiteness forces the cohomological dimension of the resulting group to be dimension three which contradicts our original VH assumption so there are at most two such actions.

\subsection{Definitions}

\begin{defn}
    [Setup]
    \label{def:setup} 
    We say that \(K = (T, x_{0},\sigma=\sigma_{1}\cdots\sigma_{N},\lambda)\) satisfies property {\em setup} if:
    \begin{enumerate}
        \item \(T\) is a simplicial tree
        \item We have that \(x_{0}\) is a valence one vertex
        \item \(\sigma\) is a concatenation of non-degenerate edgepaths
        \item \(\sigma (t)=x_{0} \implies t \in \{0,1\}\) 
        \item \(\lambda (1) \neq \lambda (N)\) 
    \end{enumerate}
\end{defn}


\begin{defn}
    [Bad]
    Given \(K = (T, x_{0},\sigma=\sigma_{1}\cdots\sigma_{N},\lambda)\) we say that an edge is {\em bad} if:
    \begin{enumerate}
        \item The edge \(e\) separates the basepoint from some endpoint. i.e. \[e^{+} \cap \text{endpoints} \neq \varnothing\] where \(e^{+}\) is the halfspace of \(e\) not containing \(x_{0}\).
        \item The edge \(e\) always has the same color. i.e. \(\left| \{\lambda (k) \mid e \subseteq \text{Im} \sigma_{k}\} \right|=1\).
    \end{enumerate}
\end{defn}

\begin{defn}
	[bad edge snipping]
    Let \(K\) satisfy property setup with \(e\) a bad edge. Let \(\sigma_{i}\) and \(\sigma_{j}\) be the first and second subpaths of \(\sigma\) that use \(e\). Take \(\sigma '\) to be the concatenation of \(\sigma_{k}\) for \(k < i\) with the subpath denoted \(\sigma '_{ij}\) that is the concatentation of the largest initial subpath of \(\sigma_{i}\) not using \(e\) and the longest tail of \(\sigma_j\) not using \(e\) with \(\sigma_{k}\) for \(k > j\). The unmodified subpaths of \(\sigma\) in \(\sigma '\) receive the same colors as before and we take \(\sigma '_{ij}\) to have the color that \(\sigma_{i}\) and \(\sigma_{j}\) shared.
\end{defn}
\subsubsection{Miscellaneous}
\begin{lem}
	[Guirardel Lemma 5.4, Corollary 5.5]
 \label{lem:guirardel} 
	Let \(T_{1} , T_{2}\) be two \(\mathbb{R}\)-trees and let \(F\) be a nonempty connected subset of \(T_{1} \times T_{2}\) with convex fibers. Then the complement of \(\overline{F}\) is a union of quadrants. That is, \(\overline{F}\) is also nonempty, connected, and has convex fibers.
\end{lem}
\begin{claim*}
    If \(K\) satisfies property setup then the first and last edge of \(\sigma\) are equal.
\end{claim*}
\begin{defn}
	[Filling]
    Let \(\{X_{k}\}_{k \in K}\) be a family of spaces where one can take convex hulls. Given \(S \subseteq X := \prod X_{k}\) define \(S_{k}\) for \(k \in K\) via: \[p \in S_{k} \iff \exists \,q,r \in S: \forall j \neq k: p_{j} = q_{j} = r_{j} \text{ and } p_{k} \in \text{cvxhull}_k (\{q_{k} , r_{k}\}).\] 
\end{defn}
\begin{defn}
	[Type \(FP\) ]
	A group is of type \(FP\) if it is (1) type \(FP_n\) for all \(n\) and (2) finite geometric cohomological dimension.
\end{defn}

\begin{defn}
	[Finite Type] 
	An action of {\em finite type} is one on a locally finite tree where  vertex stabilizers are of type FP.
\end{defn}

\begin{defn}[Open Direction] An open direction is a connected component of an \(\mathbb{R}\)-tree minus a point. 
\end{defn}

\begin{defn}[Closed Direction] A closed directon is a connected component of an \(\mathbb{R}\)-tree minus a point, union that point.
\end{defn}
\begin{defn}[Open Halfspace] An open halfspace is an open direction obtained from deleting the midpoint of an edge.
\end{defn}
\begin{defn}[Closed Halfspace] A closed halfspace is a closed direction obtained from deleting the midpoint of an edge.
\end{defn}
\begin{defn}[Halfspaces of a product] An open (resp. closed) halfspace of a product (at a certain index) is a subset where exactly one projection is an open (resp. closed)  halfspace in it's factor and the others are onto.
\end{defn}
\begin{defn}[Generalized quadrants] A generalized open (resp. closed) quadrant with respect to a product of \(k\) spaces is an intersection of \(k\) open (resp. closed) product halfspaces where each one is at a different index.
\end{defn}
\begin{defn}[cellular-product-convex] We say that \(K \subset X\) is cellular-product-convex if it's complement is the open cellular neighborhood of a union of generalized closed quadrants.
\end{defn}



\section{Outline and notation for the construction of the core}

    From Guirardel we get a map \(f\) from the universal cover of our VH complex to the product of three trees by taking inclusion in the first two factors and Guirardel's map in the last factor. We obtain \(S = \text{Im}(f)\) and will show it's cocompact. Then we put \(K = \text{cell}{(S)}\) and show it's still cocompact. Finally we fill \(K\) in all three directions obtaining the core \(C\). The main goal is to show that \(C\) is a cocompact cube complex with quadrant-convex hyperplanes so that we get a legitimate graph of spaces decomposition for \(G\) over the hyperplanes.


\begin{cor*}
    Let \(G\) be the fundamental group of a compact VH-complex. Let \(T_{v}\), and \(T_{h}\) be the vertical and horizontal splittings. Considering {\em nice} actions up to their deformation spaces these are the only two such actions.
\end{cor*}
\begin{thm}
    Let \(G\) be the fundamental group of a compact VH-complex. Then \(G\) acts geometrically on \(C\) cube complex with simply connected hyperplanes where \(C = \text{fill}( \text{cell}( \text{Im}(f) ) )\).
\end{thm}
\section{Core has quadrant convex hyperplanes}
Our goal is to show that the core is a graph of groups decomposition for \(G\), this follows from the corollary below which we will conclude from the theorem.
\begin{cor*}
    The core \(C \subseteq \mathscr{T}\) has simply connected hyperplanes.
\end{cor*}

\begin{thm}
    The core \(C\) has hyperplanes that are (1) connected and (2) quadrant convex.
\end{thm}

One way to prove the theorem is to first show that \(C\) is one dimensional fiberwise connected and then apply Guirardel's lemma in each hyperplane to conclude that they're quadrant convex as needed.

\begin{lem}
    [Connected fibers]
    The core \(C\) has connected one dimensional fibers.
\end{lem}



\subsection{Wrap up lemmas}
The following is a list of statements, the goal is to prove enough of them to arrive at item number 1 for a suitably chosen core \(C\).
\begin{enumerate}
    \item \(C \subseteq \mathscr{T}\) has simply connected hyperplanes
    \item \(C\) has hyperplanes that are (1) connected (2) quadrant convex
    \item The hyperplanes and one dimensional fibers of \(C\) are connected
    \item If \(S \subseteq \mathscr{T}\) connected in all coordinate planes, then so are \(S_{x}, S_{y},\) and \(S_{z}\).
    \item If \(R \subseteq T_{1} \times T_{2}\) has convex fibers then \( \left( R_{x} \right)_{y} = \left( R_{y} \right)_{x}\).
    \item If \(S \subseteq \mathscr{T}\) is connected in all coordinate planes then \(\sxyz\) is one dimensional fiberwise convex. 
    \item \label{state:7} If \(S \subseteq \mathscr{T}\) and is connected in all coordinate planes then \( \left( S_{x} \right)_{y} = \left( S_{y} \right)_{x}\).
    \item Guirardel's Lemma
\end{enumerate}

\subsubsection{Using Wrap-up lemmas to prove main statement}
The implications are as follows:
\begin{alignat}{3}
    (4),(7) &\fbox{\(\label{eqn:filling}\Rightarrow\)}  (6) \Rightarrow (3) \Rightarrow (2)\Rightarrow (1) \\
    (8) &\Rightarrow (4)\\
    (5) &\Rightarrow (7)
\end{alignat}
\begin{lem}
    [Reduction to vertical subpath]
    \label{lem:verticalsubpath} 
    Let \(S\) be a subset of \(T_{1} \times T_{2}\) and let \(p,q,r \in S\) satisfy: 
    \begin{enumerate}
        \item \(r \not\in S\) 
        \item \(p,q \in S\) 
        \item \(p_{2} = q_{2} = r_{2}\) 
        \item \(r_{1} \in \text{cvxhull}_{T_{1}} (\{p_{1}, q_{1} \} )\) 
    \end{enumerate}
    Let \(\sigma : [0,1] \to S\) be a path from \(p\) to \(q\), then there exists a path \(\sigma '\) such that taking \(p,q:= \sigma '(0), \sigma '(1)\) satisfies properties (2)-(4) and \(\pi_{2} (\text{Im}( \sigma '\mid_{(0,1)} ))\) is contained in an open direction of \(T_{2}\) at \(r_{2}\).
    \begin{proof}
        Consider the inverse image of \(T_{1} \times \{r_{2}\} \) under \(\sigma\). After subdividing \([0,1]\) and noting that we're taking edge paths we get that this set is a finite union of closed intervals in \([0,1]\). We also have that it contains 0 and 1. Because \(r_{1}\) is separating in \(T_{1}\) we have that each closed interval lies within an open direction in \(T_{1}\) at \(r_{1}\). Also, because \(r_{1}\) lies in the convex hull of \(p_{1}\) and \(q_{1}\) we get that the number of interals is at least 2. Consider pairs of endpoints of the closed intervals making up the closure of the complement of the inverse image. If each pair is in the same direction in \(T_{1}\) at \(r_{1}\) then because closed intervals map to horizontal directions we would have that 0 and 1 were in the same direction which contradicts our assumptions; hence there exists a subpath \(\sigma '\) where only the endpoints map into \(T_{1} \times \{r_{2}\}\). Consider \(\sigma ' \mid_{(0,1)}\), it's image lies in a product of a finite subtree of \(T_{1}\) and a disjoint union of directions of \(T_{2}\) at \(r_{2}\). Because \(\sigma '\) is continuous it must lie in a single connected component and so projects to a single verticle direction of \(T_{2}\) at \(r_{2}\).

    \end{proof}
\end{lem}
\begin{lem}
    [Statement 4]
    Let \(S \subseteq \mathscr{T}\) be a subcomplex that is connected in all coordinate planes. Then \(S_{x}, S_{y},\) and \(S_{z}\) are as well. 
    \begin{proof}
        Without loss of generality, consider \(S_{ x}\), note that \(S_{x}\) will be connected in all \(xy\) and \(xz\) planes because \(S\) was. Consider the \(yz\)-planes in \(S_{x}\), if there were no new points added then the planes are connected and we are done. Suppose that \( p \in (S_{x} \smallsetminus S ) \cap \pi^{-1} _{1} (p_{1} )\), we will show that there is a path in \(S_{x} \cap \pi_{2}^{-1} (p_{2} )\) between \(p\) and some point in \(S\).
       
        Since \(p\) isn't in \(S\) there exist distinct points \(r\) and \(s\) in \(S \) that agree in all coordinates except the first where we have that \(p_{1} \in \text{cvxhull}_{T_{1}}  (\{r_{1} , s_{1}\})\). Now, because \(S\) is connected in all coordinate planes there is a path \(\sigma\) from \(r\) to \(s\) that lies in \(S \cap \pi_{2}^{-1} (p_{2})\). In fact, we can take \(\sigma\) to be a path that begins and ends at \(p\) and \(r\) but otherwise has \(T_{3}\) coordinates lying in exactly one direction of \(T_{3}\) at \(p_{3}\). We have factored out this situation into claim \ref{lem:verticalsubpath}.  

        Take \(\sigma\) as in the claim \ref{lem:verticalsubpath}. It remains to show that there is a path connecting \(p\) to another point in \(S\). Considering closed quadrants, there exists a sequence \(t_{1},\ldots,t_{n}\) such that \(v_i:=\sigma (t_i)\) are vertices in \(p_{1} \times p_{2} \times \overline\delta\)  (here \(\delta\) is the distinguished \(T_{3}\)  direction) with the property that \(\sigma (t_{1}) =p\), \(\sigma (t_{n}) = q\), and \(\sigma_{i} := \sigma \mid_{[t_{i} , t_{i + 1} ]}\) lie in quadrant \(i\) for \(0 < i < n\). (Here, quadrant \(i\) is determined by specifying a direction in \(T_{1}\) at \(p_{1}\) since we've already chosen \(\delta\) above.) Projecting, we obtain a sequence of subpaths in \(\overline\delta\) a direction in \(T_{3}\) that are colored based on their quadrant. Note that \(\overline \delta\) has a valence one vertex at \(p_{3}\) because our tree is simplicial and subpaths don't end at \(p_{3}\) unless it's the first or last because the claim has that the interior of the path maps into \(\delta\). This situation satisfies the conditions for our coloring lemma \ref{lem:coloring}.
        
        Multicolored paths in this vertical direction \(\delta\)  give paths contained in \(S_{x}\), applying the coloring lemma \ref{lem:coloring} gives the required multicolored path. This completes the proof.
    \end{proof}
\end{lem}

\begin{lem}
    [Statement 5]
    If \(R \subseteq T_{1} \times T_{2}\) is a connected subcomplex then \( \left( R_{x} \right)_{y} = \left( R_{y} \right)_{x}\).
    \begin{proof}
        We will first show that \(\rxy\) has connected fibers. Note, it already has connected \(y\)-fibers, so it remains to show that it has connected \(x\)-fibers. Suppose that \(\rxy \cap ( T_{1} \times \{y_{0}\} )\) were a disconnected \(x\)-fiber. Then because \(\rxy\) is a subcomplex we have that the fiber is separated by some edge. Let \(x_{0}\) be the midpoint of this edge. Let \(\ell\) and \(r\) denote the left and right closed halfspaces of \(T_{1} \times \{y_{0}\}\) at the midpoint \(x_{0}\). 

        We will show that either \(\ell \cap \rxy = \varnothing\) or \(r \cap \rxy = \varnothing\). Suppose this were false, and that \(\ell \cap \rxy \neq \varnothing\) and \(r \cap \rxy \neq \varnothing\). We consider three cases: (1) \(\ell\) and \(r\) intersect \(R_{x}\) nontrivially (2) \(\ell\) and \(r\) don't intersect \(R_{x}\), and (3) Exactly one of \(\ell\) or \(r\) intersects \(R_{x}\) nontrivially.
        \begin{enumerate}
            \item Case 1: If both \(\ell\) and \(r\) intersect \(R_{x}\) then both contain points of \(R\) because we filled in the \(x\)-fiber, but then \(x_{0} \times y_{0}  \in R_{x}\) a contradiction.
            \item Case 2: Pick a point \(x^+\) in \(r \cap \rxy \smallsetminus R_{x} \). Let \(q^{+}\) and \(q^-\) be points above and below \(x^+\) in \(R_{x}\). These points are either in \(R\) already, or because we filled in the \(x\)-fiber there exist points above and below the line \(\ell \cup r\) but this is a contradiction since \(R\) is connected.
            \item Case 3: Without loss of generality, suppose \(\ell \cap R_{x} = \varnothing\) and \(r \cap R_{x} \neq \varnothing\). This implies there is some \(s \in R \cap r\), we will get a contradiction by separating this point from another point in \(R\). 
                
                Let \(u\) and \(d\) be directions of \(x_{0} \times T_{2}\) at \(x_{0} \times y_{0}\). If both intersected \(R_{x}\) then \(x_{0} \times y_{0} \in R \) a contradiction so without loss of generality, suppose \(d \cap R_{x} \neq \varnothing \). Now let \(x^-\) be a point in \(\ell\), by assumption \(x^- \in \rxy \smallsetminus R_{x} \). This gives a point \(q^-\) below \(\ell\) in \(R_{x}\) which is already in \(R\) or there exists a point \(t^- \in R\) to the left of \(q^-\); but this is a contradiction since \(\ell \cup d\) separates these points from \(s \in R\).

                Hence, either \(\ell \cap \rxy = \varnothing\) or \(r \cap \rxy = \varnothing\) where the rays are taken at midpoints missing from \(R\) and as a result \(\rxy\) has connected fibers. Similarly, \(\ryx\) also has connected fibers. 


                (...)
             
        \end{enumerate}
      
    \end{proof}

\end{lem}

\begin{lem}
    [Boxed Implication]
 \label{lem:boximp} 
    Repeatedly applying statement \ref{state:7} to both \(S\) and \(S_{\alpha}\) where \(\alpha\) is one of \(x,y,\) or \(z\)  and noticing that \(\langle (12),(23) \rangle = S_{3}\) we get that \(\sxyz\) is equal to any of the permutations of the indices. In particular, \(\Sxyz{x}{y}{z}=\Sxyz{y}{z}{x}=\Sxyz{z}{x}{y}\) which shows that \(\sxyz\) has connected one dimensional fibers.
\end{lem}
\subsection{Planar Path Argument}

 \label{lem:ppa} 
\begin{lem}[Coloring Lemma]
    \label{lem:coloring} 
    Let \(K\) satisfy property setup. Let \(A\) be the set of endpoints minus the basepoint \(x_{ 0}\). Then there exists some \(a \in A\) such that for all edges \(e\) in the geodesic \([x_{0} , a]\) there exist some \(i, j\) such that \(\sigma_{i}\) and \(\sigma_{j}\) use \(e\) and \(\lambda ( i ) \neq \lambda ( j )\).
    \begin{proof}
        Suppose the lemma were false. Then there exist counterexamples \(K\) satisfying property setup such that for all \(a \in A\) there exists an edge \(e\) on \([x_{0} , a]\) such that \(\lambda (i)= \lambda (j)\) whenever \(\sigma_{i}\) and \(\sigma_{j}\) use \(e\). These \(K\) have bad edges and \(| \sigma | > 0 \). Now take a \(K\) such that \(| \sigma |\) is minimized. Let \(K'\) be the result of snipping a bad edge. By lemma \ref{lem:preservesetup}, \(K'\) has property setup so \(| \sigma ' | > 0\). If \(K'\) contained a multicolored path between \(x_{0}\) and some \(a'\) in \(A'\) then because the path \(\sigma\) in \(K\) an be obtained by inserting subpaths into \(\sigma '\), we have that \(A' \subseteq A\) and any edge that was multicolored stays multicolored. Therefore \(K'\) has no multicolored path, because that would force \(K\) to have one. Then the fact \(| \sigma' | < | \sigma |\) gives a contradiction because \(K\) was chosen to be minimal. 
        
        
    
        
    \end{proof}
\end{lem}


\begin{lem}
    [Snip invariant]
 \label{lem:preservesetup} 
    If \(K\) satisfies property setup then \(K'\) obtained from snipping a bad edge also satisfies property setup.
    \begin{proof}
        Let \(e\) be a bad edge of \(K\), since \(\sigma_{1}\) and \(\sigma_{N}\) have different colors the edge \(e\) in at most one of them. Because \(x_{0}\) has valence one, \(\sigma_{1}\) and \(\sigma_{N}\) share the edge containing \(x_{0}\); therefore \(e\) cannot be the common edge. The rest follows.
    \end{proof}
\end{lem}

\end{document}

#justVimThings
select last search: //, go to end of search upon finding: //e
move to beginning/end of visual selection: o,O
surround a selection of text: v to select some text, press "S", then a delimiter
	- works with (, {, [ and <p>, <body>, etc.
delete cursor to beginning of word: d/<type word here>
	- works on multiple lines
move down by display lines: prefix with g e.g. gj, gk, g0, g$
start search backwards via ?



