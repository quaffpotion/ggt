\documentclass{article}
\usepackage{amsmath,amssymb,amsthm, fancyhdr, tikz-cd, xcolor}
\definecolor{mypink1}{rgb}{0.858, 0.188, 0.478}
\swapnumbers

\pagestyle{fancy}
\renewcommand{\sectionmark}[1]{\markright{\thesubsection\ #1}}
\fancyhf{}
\lhead{Research Outline}
\rhead{Tony Martino}

\newtheoremstyle{mystyle}
  {\topsep}
  {\topsep}
  {}
  {}
  {\scshape}
  {.}
  {.5em}
  {}


\theoremstyle{mystyle}
\newtheorem{thm}{Theorem}[section]
\newtheorem{thm*}{Theorem}[section]
\newtheorem{lem}{Lemma}[section]
\newtheorem{pro}[lem]{Proposition}
\newtheorem{defn}[lem]{Definition}
\newtheorem*{defn*}{Definition}
\newtheorem*{claim*}{Claim}
\newtheorem*{lem*}{Lemma}

\theoremstyle{remark}
\newtheorem{rmk}[lem]{Remark}
\newtheorem{ex}[lem]{Example}
\newtheorem{nex}[lem]{Not Example}



\begin{document}

\section{Background}

\begin{enumerate}
    \item cohomological and geometric dimension
    \item group splittings
    \item Danny Wise VH-complexes
    \item Guirardel Core
\end{enumerate}

\section{Problem Statement}

Statement for Dimension 2: 

Best case scenario: Given \(N \geq 2\) tree actions (i.e. splittings) that are pairwise not in the same deformation space obtain a generalized-VH complex, called the core (e.g. VHD-complex for the case of three pairwise inequivalent actions) by deleting light quadrants from the tree product.

The core is generalized-VH, simply connected, and cocompact. The quotient modulo \(G\) is a compact, generalized-VH, graph-of-spaces decomposition of \(G\) along its (two-sided) hyperplanes. (i.e. in particular edge maps are injective on fundamental groups) The hyperplanes are also generalized-VH complexes. Using a theorem of Bieri we get that cd \(G\) is strictly greater than that of it's hyperplanes. Continuing inductively, we get that \(G\) has cd \(\geq N\).

\section{Progress}

\begin{enumerate}
    \item Freeness of product action:

	    We're going to walk through the setup in the two dimensional case. Let \(G\) be a group with two locally finite type \(FP\) actions \(T_{1}\) and \(T_{2}\) lying in different deformation spaces.  Consider the diagonal action on \(T_{1} \times T_{2}\), we will show that this action is free. {\color{mypink1}We will show that for a given \(g \in G\) that if \(g\) fixed a point in one tree then it doesn't fix a point in the other. By symmetry suppose \(g\) fixes a point in \(T_{1}\). Consider \(G_{v}\) where \(v\) lies in \(T_{1}\), as a subgroup of \(G\) it also acts on \(T_{2}\). Suppose that \(G_{v}\) acts without a global fixed point on \(T_{2}\) then }
    \item Hyperplane separability:
        Note that hyperplanes in the tree product are dual to edges in its one-skeleton and we have a diagonal action without edge inversions.
    \item Using Bieri: If a group \(G\) has geometric dimension \(n\) and acts on a locally finite tree with \(FP_\infty\) (i.e. finite dimensional \(K(G,1)\)) stabilizers then the stabilizers have geometric dimension \(\leq n-1\).
    \item Using Howsen
    \item Using ``Quadrant-Convex'' lemma of Guirardel 
    \item Using injectivity lemma for maps of NPC complexes
    \item other: relevant facts on locally finite trees
\end{enumerate};

\end{document}
\section{Guirardel Core}

\section{VH-complexes}

Let \(X\) be our complex and \(X_{V}\) be the set of vertical edges. We say gates are connected components of \(X_{V}\) and corridors are cc of \(X - X_{V}\). The complex \(X\) is clean if the attaching maps of the corridors are injective.

\begin{thm}[Wise]
	If \(X\) a complete, VH, clean, connected complex then there is a finite sheeted cover that's a graph product.
\end{thm}

\begin{thm}
	If \(X\) is NPC, VH, compact, nice(???) then \(X\) is complete XOR contains a locally geodesic VH pair of lollipops
\end{thm}

\section{Definitions}
\begin{defn}[Graph - Abstract]
	An {\em Abstract Graph} \(\Gamma\) is a set \((V,E, \partial, i)\) where \(V\) and \(E\) are non-empty sets and \(\partial: E \to V\) and \(i: E \to E\) are functions satisfying \(i^2(e)=e\) and \(i(e) \neq e\) for all \(e \in E\). (i.e. the function \(i\) is a fixed-point free involution)
	
\end{defn}

We set \(o(e) := \partial(e)\) and \(t(e) := (\partial\circ i)(e)\) for origin and terminal vertices and put \(\overline{e} := i(e)\). We call the orbits of \(i\) the {\em undirected edges} of \(G\). (i.e. the set \(\{ e, \overline{e}\}\) is an undirected edge) 

\begin{defn}
	A tree action \((G,T)\) is {\em minimal} if it contains no proper invariant subtree. A graph of groups \(\mathcal{G}\) is minimal if there is no proper subgroup carrying the entire group.
	
\end{defn}


\begin{defn}[Collapse Move]
	If we have a group \(G\) acting on a tree \(T\) with an edge \(e\) such that \(G_{e} = G_{v}\) where \(v=o(e)\) and \(o(e)\) and \(t(e)\) are in different \(G\)-orbits then we can form \(T_{e}\) a new tree with \(V(T_{e} )=V(T)\smallsetminus G t(e)\) and \(E(T_{e} )=E(T) \smallsetminus Ge\). Then for all edges \(f\) with \(o(f) = t(ge)\) for some \(g \in G\) we define \(o(f)=gv\) in \(T_{e}\). 
	
	(Alternatively, if \(q\) was the map that paired the inital and terminal vertices of \(g e\) and left the others alone then we could define the new tree by taking \(E(T_{e} ):=E(T)\smallsetminus Ge \) and \(V(T_{e} ):=q(V)\) with attaching map \(q \circ \partial\).

	If \(\mathcal{G}\) is a graph of groups decomposition of \(G\) with \(\varphi_{e}: G_{e} \to G_{\partial e}\) an isomorphism then define \(\mathcal{G}_{e}\) by removing \(e\), \(\overline{e}\) and \(\partial e\) and for every edge \(f\) with \(\partial(f) = \partial(e)\) replace \(\varphi_{f}\) with \(\varphi_{\overline{e}} \circ \varphi_{e}^{-1} \circ \varphi_{f}\) and set \(\partial f = \partial \overline{e}\). This corresponds to taking the edge \(e\) from the tree description above and folding \([e]\) in the graph.

\end{defn}
\begin{defn}[Fold]
	Given a tree with \(\partial e = \partial f\) identify \(e\) with \(f\) as well as \(\overline{e}\) and \(\overline{f}\) and \(\overline{\partial} e\) with \(\overline{\partial} f\) and do so equivariantly. (In \(\mathcal{G}\) this corresponds to moves of type A or B with subtype I, II, or III.)
\end{defn}

\begin{claim*} A reduced not locally finite tree remains not locally finite after folding.
\end{claim*}

\hrulefill

\section{Claims}



\end{document}

